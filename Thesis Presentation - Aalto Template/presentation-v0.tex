\documentclass[handout,first=blue,second=blue,logo=yellowexcELEC,usenames,dvipsnames, aspectratio=169]{aaltoslides169}
%\documentclass[first=blue,second=blue,logo=blueexc]{aaltoslides}
%\documentclass[handout]{beamer}        % to compile 2x2 handouts

%\usepackage{beamerthemesplit}

\usepackage[orientation=landscape,size=custom,width=16,height=9,debug]{beamerposter} 

\usefonttheme[onlymath]{serif}

\setbeamercovered{transparent}
\setbeamertemplate{caption}[numbered]

\usepackage{amsmath,amssymb}
\usepackage{biblatex}
\addbibresource{bibliography_admission_ecc2018.bib}
\usepackage[ruled,vlined,linesnumbered]{algorithm2e}
\usepackage{arydshln}
\usepackage{geometry}
\usepackage{graphicx}
\usepackage{lmodern}
\usepackage{pifont}
\usepackage{scrextend}
\usepackage{bm}
\usepackage{amsfonts}
\usepackage{booktabs}
\usepackage{siunitx}
\usepackage{gensymb}
\usepackage{subfig}
\usepackage{makecell}
\usepackage{xfrac}
\usepackage{cancel}
\usepackage{dsfont}

\usepackage{environ}
\usepackage{tikz}
\usetikzlibrary{arrows}
\usetikzlibrary{fadings}
\usetikzlibrary{matrix}
\usetikzlibrary{chains}
\usetikzlibrary{positioning}
\usetikzlibrary{fit,shapes.geometric}
\usetikzlibrary{arrows, patterns}
\usepackage{tcolorbox}
\usepackage{multimedia}
\usepackage{hyperref}
\usepackage{tcolorbox}


\usepackage[utf8]{inputenc}
%Load useful packages
\usepackage{graphicx} % Allows including images
\usepackage{booktabs} % Allows the use of \toprule, \midrule and \bottomrule in tables
%\usepackage{subcaption}
\usepackage{subfiles}
\usepackage{url}
\usepackage{amsmath}
\usepackage{amssymb}
\usepackage{mathtools}
\newtheorem{proposition}{Proposition}
%\usetheme{AnnArbor}
\setbeamertemplate{caption}[numbered]
\usepackage{lipsum}
\newcommand{\ie}{\emph{i.e., }}
\newcommand{\blue}{\textcolor{blue}}
\newcommand\xdownarrow[1][2ex]{%
   \mathrel{\rotatebox{90}{$\xleftarrow{\rule{#1}{0pt}}$}}
}
%\usepackage[usenames,dvipsnames]{color}
%\pgfplotsset{compat=1.13}

\usepackage{makecell}
\usepackage{colortbl} % for cell coloring 
\usepackage{multirow}





% packages necessary for the danger sign
\usepackage{stackengine}
\usepackage{scalerel}
\newcommand\dangersign[1][2ex]{%
  \renewcommand\stacktype{L}%
  \scaleto{\stackon[1.3pt]{\color{red}$\triangle$}{\tiny\bfseries !}}{#1}%
}


\tikzstyle{observation} = [rectangle, rounded corners, minimum width=3cm, minimum height=0.5cm,text centered]
\tikzstyle{binG} = [rectangle, rounded corners, minimum width=3cm, minimum height=0.5cm,text centered, draw=green,thick, fill=green!10]
\tikzstyle{binB} = [rectangle, rounded corners, minimum width=3cm, minimum height=0.5cm,text centered, draw=red,thick, fill=red!10]
\tikzstyle{arrow} = [thick,->,>=stealth]
\tikzstyle{crossout} = [rectangle, rounded corners, minimum width=3cm, minimum height=0.5cm,text centered, pattern=north west lines]

\tikzset{
	invisible/.style={opacity=0},
	visible on/.style={alt={#1{}{invisible}}},
	alt/.code args={<#1>#2#3}{%
		\alt<#1>{\pgfkeysalso{#2}}{\pgfkeysalso{#3}} % \pgfkeysalso doesn't change the path
	},
}

\usetikzlibrary{shapes.arrows}

\tikzset{
    myarrow/.style={
        draw=aaltoblue,
        fill=aaltoblue!70!white,
        single arrow,
        minimum height=5.5ex,
        single arrow head extend=1ex
    }
}
\newcommand{\arrowup}{%
\tikz [baseline=-0.5ex]{\node [myarrow,rotate=90] {};}
}
\newcommand{\arrowdown}{%
\tikz [baseline=-1ex]{\node [myarrow,rotate=-90] {};}
}

\newcounter{nodemarkers}
\newcommand<>\circletext[1]{%
	\tikz[overlay,remember picture] 
	\node (marker-\arabic{nodemarkers}-a) at (0,1.5ex) {};%
	#1%
	\tikz[overlay,remember picture]
	\node (marker-\arabic{nodemarkers}-b) at (0,0){};%
	\tikz[overlay,remember picture,inner sep=2pt]
	\node#2[draw,green,ellipse,fit=(marker-\arabic{nodemarkers}-a.center) (marker-\arabic{nodemarkers}-b.center)] {};%
	\stepcounter{nodemarkers}%
}

\everymath{\displaystyle}

\changefontsizes{8pt}

\usepackage{mathtools}
\DeclarePairedDelimiter\ceil{\lceil}{\rceil}
\DeclareMathOperator*{\argmax}{arg\,max}

\setlength{\leftmargini}{0.45cm}
\setlength{\leftmarginii}{0.35cm}

 \def\tr{\mathop{\mathrm{tr}}}
  
\setbeamertemplate{section in toc}[square]
\setbeamertemplate{subsection in toc}[square]
\setbeamerfont{section number projected}{size=\large}
\setbeamercolor{section number projected}{bg=aaltoblue,fg=white}

\newcommand\aaltofootertext[3]{\def\footfrow{#1}\def\footsrow{#2}\def\foottrow{#3}}
\aaltofootertext{Omar Boufous}{\today}{\insertframenumber/\inserttotalframenumber}

% Syntax: \colorboxed[<color model>]{<color specification>}{<math formula>}
\newcommand*{\colorboxed}{}
\def\colorboxed#1#{%
  \colorboxedAux{#1}%
}
\newcommand*{\colorboxedAux}[3]{%
  % #1: optional argument for color model
  % #2: color specification
  % #3: formula
  \begingroup
    \colorlet{cb@saved}{.}%
    \color#1{#2}%
    \boxed{%
      \color{cb@saved}%
      #3%
    }%
  \endgroup
}

\newtcolorbox{adbox}[1][\hspace{-0.3cm} \textbf{Advertisement}]{
colback=white,
colbacktitle=aaltoblue!10!white,
coltitle=aaltoblue,
colframe=aaltoblue,
boxrule=1pt,
titlerule=0pt,
arc=5pt,
title={\strut#1}
}

\usepackage{pgfplots}

%\mode<presentation>{\usetheme{Warsaw}}

%\makeatletter
%\def\th@mystyle{%
%    \normalfont % body font
%    \setbeamercolor{block title example}{bg=orange,fg=white}
%    \setbeamercolor{block body example}{bg=orange!20,fg=black}
%    \def\inserttheoremblockenv{exampleblock}
%  }
%\makeatother
%\theoremstyle{mystyle}
\newtheorem*{remark}{Remark}




%List of packages
%\usepackage{amsmath}

%%%%%%%%%%%%%%%%%%%%%%%%%%%%%% Metadata %%%%%%%%%%%%%%%%%%%%%%%%%%%%%
\hypersetup
{
	%Separate multiple authors by comma
	pdfauthor={Omar Boufous},
	pdftitle={PhD presentation},
	pdfsubject={PhD presentation},
	pdfkeywords={PhD presentation},
	colorlinks=false
}

%%%%%%%%%%%%%%%%%%%%%%%%%%%%%% Title related %%%%%%%%%%%%%%%%%%%%%%%%%%%%%%
\setbeamertemplate{subsection in toc}[default]

%\vspace{1cm}
%The contact for one of the authors MUST be embedded on the title (see below)
\title[]{Correlated Equilibria \& Learning}
%Subtitle (if needed)
\subtitle{}
%For LICENSE, we suggest CC-BY-SA, but you are free to choose your own as long
%as the LICENSE you choose is AT LEAST as permissive as CC-BY-SA
\date[2020]{\today\\ \vspace{1cm} {PhD Thesis Defense\\}}
\author[Boufous \hspace{0.2cm} {\includegraphics[height=0.2cm, keepaspectratio]{8015.png}}]{\texorpdfstring{Omar Boufous \vspace{0.5cm}\\ {\footnotesize Supervised by} \\ \vspace{0.2cm} {\footnotesize Rachid El-Azouzi$^2$, Mikael Touati$^1$, Eitan Altman$^{2,3}$, Mustapha Bouhtou$^1$}}{}}
\institute[Orange]{$^1$Orange, Châtillon, France\\$^2$CERI/LIA, Université d’Avignon, Avignon, France\\$^3$INRIA Sophia Antipolis, France}







%%%%%%%%%%%%%%%%%%%%%%%%% Presentation begins here %%%%%%%%%%%%%%%%%%%%%%%%%
\begin{document}

\begin{frame}
	\titlepage
\end{frame}




 

%% new slide ===========================================================
%\begin{frame}
%\frametitle{Table of Content}
%
%\begin{itemize}
%	\item \textbf{Introduction}
%	\item \textbf{Related Work}
%	\item \textbf{Part 1 : Learning Correlated Equilibria}
%	\begin{itemize}
%		\item The Learning Problem
%		\item CPRM Algorithm
%		\item Convergence Properties
%		\item Simulation Results
%	\end{itemize}
%	\item \textbf{Part 2 : Constrained Correlated Equilibria}
%	\begin{itemize}
%		\item Definition of Constrained Correlated Equilibria
%		\item Properties of Constrained Correlated Equilibrium Strategies
%		\item Constrained Correlated Equilibrium Distributions
%		\item Existence of Constrained Correlated Equilibria
%		\item Simulation Results
%	\end{itemize}
%	\item \textbf{Conclusion}
%\end{itemize}
%
%\end{frame}


% new slide ===========================================================
\begin{frame}
\frametitle{Presentation Outline}
\pause
\begin{enumerate}
    \item \textbf{Background \& Preliminaries}
    \begin{enumerate}\setlength{\itemsep}{-0.05cm}
    	\item Introduction \pause
        \item Related Work \pause
        %\item Highlight of contributions
        \item Notations \& Model \pause
    \end{enumerate}
    \item \textbf{Learning Correlated Equilibria}
    \begin{enumerate}\setlength{\itemsep}{-0.05cm}
        \item Online Learning with Regret 
        \item Algorithm Description
        \item Stochastic Model
        \item Simulation Results
        \item Time-varying Game
        \item Application to a Congestion Game
    \end{enumerate}\pause
    \item \textbf{Constrained Correlated Equilibria}
    \begin{enumerate}\setlength{\itemsep}{-0.05cm}
        \item Definition \& Example
        	\item Properties of Constrained Correlated Equilibrium Strategies
	\item Canonical Representation and Existence of Constrained Correlated Equilibria
	\item Computation of Constrained Correlated Equilibria
	\item Simulation Results
    \end{enumerate}\pause
     \item \textbf{Conclusions \& Possible Directions of Research}
\end{enumerate}

%    \item \textbf{Correlated Equilibria \& Properties}
%    \begin{enumerate}\setlength{\itemsep}{-0.05cm}
%        \item Extended Game \& Correlated Strategies
%        \item Coupled Constraints in the Extended Game
%    \end{enumerate}

\end{frame}


% new slide ===========================================================
\begin{frame}[t]
\frametitle{Introduction}


\begin{columns}
\begin{column}{0.56\textwidth}
\pause
\begin{itemize} \setlength{\itemindent}{.2in}
    \item A \textbf{generalization} of Nash equilibria with strong connections to Bayesian rationality introduced by Aumann \pause
    \item A solution concept with appealing \textbf{computational properties}  \pause
    \item Basic \textbf{learning dynamics} naturally lead to correlated equilibria \pause
    %\item A second proof of existence and a generalization to infinite games in 
    \item \textbf{Canonical} form characterizing the set of correlated equilibrium distributions \pause
    \item Simple \textbf{implementation} using a correlation device  \pause
    \item Many \textbf{applications} in engineering, economics, etc.  \pause
\end{itemize}

\end{column}
\begin{column}{0.46\textwidth}  %%<--- here
\visible<2->{\begin{figure}[!ht]
  \centering
  \includegraphics[scale=0.55]{"./figures/robert.jpg}
  \caption{\visible<2->{Robert John Aumann}}
\end{figure}}
\end{column}
\end{columns}


\end{frame}
% new slide ===========================================================
\begin{frame}
\frametitle{Related Work}

\textbf{Correlated Equilibria} \pause
%%%%%%%%%%%%%%%%%%%%%%%%%%%%%%%%%%%%%%%%%%%%%%%%%%%%%%%%%%%
\begin{itemize}\setlength{\itemsep}{-0.1cm}
    \setlength{\itemindent}{.2in}
    \item Defined in (Aumann, 1974) and (Aumann, 1987) \pause
    \item A proof of existence and a generalization to infinite games in (Hart \& Schmeidler, 1989) \pause
    \item Extensive form correlated equilibrium defined in (Von Stengel \& Forges, 2008) \pause
    \item Other extensions in (Forges, 2020), (Brandenburger \& Dekel, 1992) and (Grant \& Stauber, 2022) \pause
\end{itemize}

%%%%%%%%%%%%%%%%%%%%%%%%%%%%%%%%%%%%%%%%%%%%%%%%%%%%%%%%%%%
\textbf{Learning Correlated Equilibria} \pause
\begin{itemize}\setlength{\itemsep}{-0.1cm}
    \setlength{\itemindent}{.2in}
    %\item Regret Matching (Hart \& Mas-Colell, 2000) converging to the set of correlated equilibrium distributions 
	\item Regret-matching (Hart \& Mas-Colell, 2000) implying the convergence to the \underline{set} of correlated equilibria \pause
	\item Correlated Q-learning for selection of specific correlated equilibrium points (Greenwald \& Hall, 2003) \pause
	\item Convergence to the social welfare maximizing correlated equilibrium (Borowski et al., 2015) \pause
	\item Learning algorithms for specific applications e.g., resource allocation game (Cigler \& Faltings, 2011) \pause
    %under appropriate assumptions on utilities and the feasible set of joint strategies
\end{itemize}

%%%%%%%%%%%%%%%%%%%%%%%%%%%%%%%%%%%%%%%%%%%%%%%%%%%%%%%%%%%
\textbf{Constraints \& Correlated Equilibria} \pause
\begin{itemize}\setlength{\itemsep}{-0.1cm}
    \setlength{\itemindent}{.2in}
    \item (Debreu, 1952) defines the concept of generalized equilibrium \pause
    \item (Arrow \& Debreu, 1954) shows a proof of existence \pause
    \item (Rosen, 1965) shows existence and uniqueness of equilibria in concave games with coupled constraints \pause
    \item (Bernasconi et al., 23) introduces the concept of constrained Phi-equilibrium \pause
    %under appropriate assumptions on utilities and the feasible set of joint strategies
\end{itemize}

\textbf{Many Other Theoretical and Applied Contributions} \pause
\begin{itemize}\setlength{\itemsep}{-0.1cm}
    \setlength{\itemindent}{.2in}
    \item Applications in energy systems and smart grids (Kulkarni, 2017), resource allocation problems (Pang et al., 2008) (Nowak et al., 2018)
\end{itemize}


\end{frame}


% new slide ===========================================================
\begin{frame}
\frametitle{Model \& Definitions}
\vspace{0.3cm}

\pause

\begin{columns}
\begin{column}{0.56\textwidth}
\; \quad Finite non-cooperative game  $G  = (\mathcal{N}, (\mathcal{A}_i)_{i \in \mathcal{N}}, (u_i)_{i \in \mathcal{N}})$ 
\begin{itemize}
\pause
\setlength{\itemindent}{2.5em}
\setlength\itemsep{-0.35em}
\item Set of players $\mathcal{N}$  \pause
\item Action set $\mathcal{A}_i$ for each $i \in \mathcal{N}$   \pause
\item Utility function $u_i : {\scriptscriptstyle \prod\limits_{i \in \mathcal{N}}}\mathcal{A}_i = \mathcal{A} \rightarrow \mathbb{R}$  \pause 
\end{itemize}
\end{column}
\begin{column}{0.46\textwidth}  %%<--- here
Correlation device  $d = (\Omega, (\mathcal{P}_i)_{i \in \mathcal{N}}, \bm{q})$  \pause
\begin{itemize}
\setlength{\itemindent}{1.2em}
\setlength\itemsep{-0.35em}
\item A sample space $\Omega$ \pause
\item Partition $\mathcal{P}_i$ of $\Omega$  for each $i \in \mathcal{N}$    \pause
\item Probability distribution $\bm{q}$ over $\Omega$  $\textcolor{white}{{\scriptscriptstyle \prod\limits_{i \in \mathcal{N}}}}$  \pause
\end{itemize}
\end{column}
\end{columns}
\vspace{0.2cm}
Player \textbf{$i$'s set of strategies} is $\mathcal{S}_{i, d} = \{ f_i : \Omega \rightarrow \mathcal{A}_i \text{  s.t. }  f_i \text{ is } \mathcal{P}_i\text{-measurable} \}$ and the \textbf{set of strategy profiles} is $\mathcal{S}_{d} = \mathcal{S}_{1, d} \times  ... \times \mathcal{S}_{n, d}$. \pause
\vspace{-0cm}
\begin{block}{Definition 1 -- Correlated equilibrium (Aumann, 1974)\pause} 
\vspace{-0.1cm}
A correlated equilibrium of $G$ is a pair $(d, \bm{\alpha}^*)$ where $\bm{\alpha}^* : \Omega \rightarrow \mathcal{A}$ is a correlated strategy profile such that  
\vspace{-0.2cm}
\begin{align}
\forall i \in \mathcal{N}, \forall \alpha_i^\prime \in \mathcal{S}_{i, d} \quad  \sum_{\omega\in \Omega} \bm{q}(\omega) u_i(\bm{\alpha}^*_i(\omega), \bm{\alpha}^*_{-i}(\omega)) \geq  \sum_{\omega\in \Omega} \bm{q}(\omega) u_i(\alpha^\prime_i(\omega), \bm{\alpha}^*_{-i}(\omega))
\end{align}\pause
\end{block}
\vspace{-0.1cm} 
%Given a \textbf{constraint set} $\mathcal{R}_d \subseteq \mathcal{S}_d$, we define a constrained correlated equilibrium,
\vspace{-0cm}
A correlated equilibrium distribution of $G$ (Forges, 2020) is a probability distribution $\bm{p} \in \Delta(\mathcal{A})$ such that,
\vspace{-0.15cm}
\begin{align}
\forall i \in \mathcal{N},
\forall \beta_i : \mathcal{A}_i \rightarrow \mathcal{A}_i
\quad
\sum_{\bm{a} \in \mathcal{A}} \bm{p}(\bm{a}) u_i(\bm{a}_i, \bm{a}_{-i}) \geq  \sum_{\bm{a} \in \mathcal{A}} \bm{p}(\bm{a}) u_i(\beta_i(a_i), \bm{a}_{-i})
\end{align} 
\vspace{-0.1cm}
%\customfootnotetext{$1$}{Aumann, R. J. (1974). {''\emph{Subjectivity and correlation in randomized strategies}''}. Journal of mathematical Economics.}
%\customfootnotetext{$2$}{Boufous, O., El-Azouzi, R., Touati, M., Altman, E., Bouhtou, M., (2023). {''\emph{Constrained Correlated Equilibria}''}. ArXiv (to appear).}
\end{frame}


% new slide ===========================================================
\begin{frame}[t]

\vspace{1.7cm}
\vspace*{\fill}
\begin{center}
	{ \huge Part 1 : Learning Correlated Equilibria }
\end{center}
\vspace*{\fill}

\end{frame}


% new slide ===========================================================
\begin{frame}
\frametitle{Online Learning with Regret}
\pause

\vspace{0.3cm}
The game $G$ is played repeatedly through time $t= 1, 2, 3, ...$. Each step $t$, action profile $\bm{a}^t = (a_1^t, ..., a_n^)$ is played \pause

\vspace{0.1cm}

\begin{block}{Definition 3 -- Regret (Hart \& Mas‐Colell, 2000)}
The regret player $i$ experiences at time $t$ for any pair of actions $j,k \in \mathcal{A}_i$ is
\vspace{-0.15cm}
\begin{align}
\scriptstyle  R_{i}^{t}(j, k) = max\Bigg(0, \frac{1}{t} \sum\limits_{\tau \leq t : a_i^\tau = j}\left[u_{i}\left(k, \bm{a}^{\tau}_{-i}\right)-u_{i}\left(\bm{a}^{\tau}\right)\right]\Bigg) = max\Bigg(0, D_i^t(j,k)\Bigg)
\end{align}
\end{block}
\vspace{-0.2cm}
\pause

\textbf{Regret-Matching algorithm} (Hart \& Mas‐Colell, 2000)
\pause
$$\left\{\begin{array}{l}
p^{t+1}_i(k) = \frac{1}{\mu} R_i^t(j, k), \\
p^{t+1}_i(j) = 1-\sum_{k \in \mathcal{A}_i: k \neq j} p^{t+1}_i(k) .
\end{array}\right.$$

\pause

\begin{block}{Proposition 1 (Hart \& Mas‐Colell, 2000)}
Let $(\bm{a}^t)_{t=1,2, \ldots}$ be a sequence of plays (i.e., $\bm{a}^t \in \mathcal{A}$ for all $\left.t\right)$ and let $\varepsilon \geq 0$. Then: $\limsup _{t \rightarrow \infty} R^t_i(j, k) \leq \varepsilon$ for every $i \in \mathcal{N}$ and every $j, k \in \mathcal{A}_i$ with $j \neq k$, if and only if the sequence of empirical distributions $\bm{q}^t$ converges to the set of correlated $\varepsilon$-equilibria.
\end{block}
\vspace{-0.2cm}



\end{frame}

% new slide ===========================================================
\begin{frame}
\frametitle{Online Learning with Regret}

\vspace{0.3cm}
\textbf{Regret-Matching algorithm} (Hart \& Mas‐Colell, 2000)

$$\left\{\begin{array}{l}
p^{t+1}_i(k) = \frac{1}{\mu} R_i^t(j, k), \\
p^{t+1}_i(j) = 1-\sum_{k \in \mathcal{A}_i: k \neq j} p^{t+1}_i(k) .
\end{array}\right.$$
\pause
\vspace{0.15cm}
\textbf{Regret-Matching algorithm applied to a 3-by-2 game}

\vspace{-0.2cm}
\begin{columns}[c] % The "c" option specifies centered vertical alignment while the "t" option is used for top vertical alignment
\column{.4\textwidth}
\begin{table}[H]
\normalsize
\centering
\begin{tabular}{c|c|c}
 & \multicolumn{1}{c|}{\textbf{D}}  & \textbf{E} \\ \hline
\textbf{A} & \multicolumn{1}{c|}{(2, 29)} & (16, 7)  \\ \hline
\textbf{B} & \multicolumn{1}{c|}{(4, 7)} & (6, 13)  \\ \hline
\textbf{C} & \multicolumn{1}{c|}{(4, 4)} & (6, 6)
\end{tabular}
\caption{Utility matrix.}
\end{table}
\pause
\column{.6\textwidth}
%\hspace{-0.5cm}\visible<2->{\begin{figure}[!ht]
%  \centering
%  \hspace{-0.4cm}\includegraphics[scale=0.55]{"./figures/probabilities_RM".pdf}
%  \vspace{-0.3cm}
%  \caption{Evolution of the empirical probability distribution}
%\end{figure}}
\hspace{-0.5cm}\visible<3->{\begin{figure}[!ht]
  \centering
  \hspace{-0.4cm}\includegraphics[scale=0.55]{"./figures/probabilities_RM".pdf}
  \vspace{-0.3cm}
  \caption{Evolution of the empirical probability distribution.}
\end{figure}}
\end{columns}
\pause

\vspace{-0.1cm}
\begin{itemize}
\setlength{\itemindent}{2em}
\setlength\itemsep{-0.2em}
\item The empirical distribution does not seem to converge towards a correlated equilibrium point  \pause
\item We propose a new dynamic called Correlated Perturbed Regret Minimization (CPRM) 
\end{itemize}
\vspace{0.2cm}

%\footnotetext{Hart Sergiu, Andreu Mas‐Colell. (2000). \emph{"A simple adaptive procedure leading to correlated equilibrium"}. Econometrica.}



\end{frame}

%===============================================================================
\begin{frame}{CPRM Algorithm}

\pause

Inspired from the learning algorithm introduced in (Young, H. P., 2009) for learning pure Nash equilibria.
\pause

\vspace{0.3cm}

At time $t$, each player $i$ is characterized by a mood $m_i^t \in \{ \textcolor{red}{syn}, \; \textcolor{blue}{asyn} \}$ \pause
\vspace{1cm}

\begin{columns}[c] % The "c" option specifies centered vertical alignment while the "t" option is used for top vertical alignment
\column{.5\textwidth}
\begin{figure}[h!]
\hspace{-0.2cm} \scalebox{0.8}{\input{figures/learning/learning1.tex}}
%\caption{Agent $i$ state dynamics}
\end{figure}
\column{.5\textwidth}
\begin{figure}[h!]
\hspace{-0.2cm} \scalebox{0.8}{

\tikzset{every picture/.style={line width=0.75pt}} %set default line width to 0.75pt        

\begin{tikzpicture}[x=0.75pt,y=0.75pt,yscale=-1,xscale=1]
%uncomment if require: \path (0,316); %set diagram left start at 0, and has height of 316

%Shape: Ellipse [id:dp4138434097361441] 
\draw  [fill={rgb, 255:red, 208; green, 2; blue, 27 }  ,fill opacity=1 ] (369.58,56.19) .. controls (369.58,49.79) and (374.77,44.6) .. (381.17,44.6) .. controls (387.57,44.6) and (392.75,49.79) .. (392.75,56.19) .. controls (392.75,62.59) and (387.57,67.78) .. (381.17,67.78) .. controls (374.77,67.78) and (369.58,62.59) .. (369.58,56.19) -- cycle ;
%Straight Lines [id:da8006346956127226] 
\draw    (381.17,67.78) -- (381.17,97.92) ;
%Straight Lines [id:da5496336415878431] 
\draw    (381.17,97.38) -- (366.43,114.92) ;
%Straight Lines [id:da10380737242225169] 
\draw    (381.17,97.38) -- (396.63,114.92) ;
%Straight Lines [id:da8197675086482226] 
\draw    (381.17,74.35) -- (368.64,91.89) ;
%Straight Lines [id:da9395536462719025] 
\draw    (381.17,74.35) -- (393.69,91.89) ;
%Shape: Cloud [id:dp19370286388009572] 
\draw   (226.42,55.17) .. controls (225.53,51.12) and (229.25,47.33) .. (235.97,45.41) .. controls (242.7,43.49) and (251.27,43.77) .. (258.04,46.14) .. controls (260.58,43.93) and (265.11,42.54) .. (270.25,42.39) .. controls (275.4,42.23) and (280.55,43.33) .. (284.16,45.36) .. controls (286.33,43.35) and (290.47,42.13) .. (295.1,42.13) .. controls (299.73,42.13) and (304.2,43.35) .. (306.93,45.35) .. controls (310.78,43.3) and (316.78,42.64) .. (322.34,43.67) .. controls (327.89,44.7) and (332,47.22) .. (332.89,50.15) .. controls (337.45,51.01) and (341.19,52.8) .. (343.15,55.08) .. controls (345.12,57.35) and (345.1,59.88) .. (343.13,62.01) .. controls (347.5,65.2) and (348.37,69.22) .. (345.4,72.57) .. controls (342.44,75.92) and (336.09,78.08) .. (328.72,78.27) .. controls (328.51,81.53) and (324.84,84.36) .. (319.11,85.66) .. controls (313.38,86.96) and (306.5,86.53) .. (301.11,84.54) .. controls (298.59,88.34) and (291.9,90.91) .. (283.92,91.13) .. controls (275.93,91.35) and (268.1,89.19) .. (263.79,85.57) .. controls (258.23,86.99) and (251.64,87.15) .. (245.49,86.04) .. controls (239.34,84.93) and (234.17,82.62) .. (231.12,79.65) .. controls (225.51,79.71) and (220.17,77.98) .. (217.75,75.32) .. controls (215.33,72.65) and (216.35,69.62) .. (220.3,67.73) .. controls (215.38,66) and (212.97,62.92) .. (214.34,60.09) .. controls (215.7,57.27) and (220.53,55.34) .. (226.3,55.32) ; \draw   (220.3,67.72) .. controls (222.62,68.54) and (225.33,68.99) .. (228.06,69.01)(231.12,79.65) .. controls (232.3,79.64) and (233.45,79.54) .. (234.56,79.37)(263.79,85.57) .. controls (263,84.91) and (262.34,84.21) .. (261.83,83.48)(301.11,84.54) .. controls (301.57,83.85) and (301.88,83.13) .. (302.04,82.39)(328.72,78.27) .. controls (328.94,74.79) and (325.2,71.42) .. (319.1,69.6)(343.13,62.01) .. controls (342.07,63.14) and (340.49,64.12) .. (338.52,64.86)(332.89,50.15) .. controls (333.04,50.64) and (333.09,51.13) .. (333.05,51.61)(306.93,45.35) .. controls (305.97,45.87) and (305.17,46.45) .. (304.56,47.09)(284.16,45.36) .. controls (283.64,45.84) and (283.24,46.36) .. (282.98,46.9)(258.04,46.14) .. controls (259.47,46.64) and (260.79,47.23) .. (261.97,47.88)(226.42,55.17) .. controls (226.54,55.73) and (226.74,56.28) .. (227.04,56.83) ;
%Shape: Circle [id:dp03831063999469109] 
\draw   (348.98,60.51) .. controls (348.98,58.02) and (351,56) .. (353.49,56) .. controls (355.98,56) and (358,58.02) .. (358,60.51) .. controls (358,63) and (355.98,65.02) .. (353.49,65.02) .. controls (351,65.02) and (348.98,63) .. (348.98,60.51) -- cycle ;
%Shape: Circle [id:dp8075756454846346] 
\draw   (360.5,58.76) .. controls (360.5,57.51) and (361.51,56.5) .. (362.76,56.5) .. controls (364.01,56.5) and (365.02,57.51) .. (365.02,58.76) .. controls (365.02,60.01) and (364.01,61.02) .. (362.76,61.02) .. controls (361.51,61.02) and (360.5,60.01) .. (360.5,58.76) -- cycle ;

% Text Node
\draw (260.79,153.84) node [anchor=north west][inner sep=0.75pt]    {$ \begin{array}{l}
\textcolor[rgb]{0.82,0.01,0.11}{synchronous}\\
\ \ \ \ \ player
\end{array}$};
% Text Node
\draw (257,57.4) node [anchor=north west][inner sep=0.75pt]  [font=\small]  {$\textcolor[rgb]{0.82,0.01,0.11}{\ \ \ syn}$};


\end{tikzpicture}
}
%\caption{Agent $i$ state dynamics}
\end{figure}
\end{columns}

\vspace{0.8cm}
%\textcolor{red}{PRECISER L'ETAT}
%Young, H. P. (2009). \emph{''Learning by trial and error''}. Games and economic behavior.
\end{frame}
%===============================================================================
\begin{frame}{CPRM Algorithm}
\begin{figure}[h!]
\hspace{-0.2cm} \scalebox{0.8}{\input{figures/learning/learning7.tex}}
%\caption{Agent $i$ state dynamics}
\end{figure}

\begin{columns}

\begin{column}{0.5\textwidth}
\centering
Players implementing a\\regret-minimizing strategy
\end{column}
\begin{column}{0.5\textwidth}  %%<--- here
\centering
Players play the actions drawn\\from the empirical distribution
\end{column}

\end{columns}
\end{frame}
%===============================================================================
\begin{frame}{CPRM Algorithm}
\begin{figure}[h!]
\hspace{-0.2cm} \scalebox{0.8}{\begin{tikzpicture}[x=0.75pt,y=0.75pt,yscale=-1,xscale=1]
%uncomment if require: \path (0,341); %set diagram left start at 0, and has height of 341

%Shape: Ellipse [id:dp3005583251278674] 
\draw  [dash pattern={on 4.5pt off 4.5pt}] (13.72,205.29) .. controls (13.72,183.04) and (82.16,165) .. (166.57,165) .. controls (250.99,165) and (319.42,183.04) .. (319.42,205.29) .. controls (319.42,227.53) and (250.99,245.57) .. (166.57,245.57) .. controls (82.16,245.57) and (13.72,227.53) .. (13.72,205.29) -- cycle ;
%Shape: Ellipse [id:dp8969793505161732] 
\draw  [dash pattern={on 4.5pt off 4.5pt}] (344.3,205.29) .. controls (344.3,183.04) and (412.74,165) .. (497.15,165) .. controls (581.57,165) and (650,183.04) .. (650,205.29) .. controls (650,227.53) and (581.57,245.57) .. (497.15,245.57) .. controls (412.74,245.57) and (344.3,227.53) .. (344.3,205.29) -- cycle ;
%Curve Lines [id:da2811571509974371] 
\draw    (329.37,75.01) .. controls (433.58,76.55) and (434.1,75.93) .. (432.03,173.35) ;
\draw [shift={(432,174.82)}, rotate = 271.22] [color={rgb, 255:red, 0; green, 0; blue, 0 }  ][line width=0.75]    (10.93,-3.29) .. controls (6.95,-1.4) and (3.31,-0.3) .. (0,0) .. controls (3.31,0.3) and (6.95,1.4) .. (10.93,3.29)   ;
%Curve Lines [id:da7257100915393806] 
\draw    (329.37,75.01) .. controls (471.39,75.56) and (472.1,75.56) .. (472,172.86) ;
\draw [shift={(472,174.33)}, rotate = 270.06] [color={rgb, 255:red, 0; green, 0; blue, 0 }  ][line width=0.75]    (10.93,-3.29) .. controls (6.95,-1.4) and (3.31,-0.3) .. (0,0) .. controls (3.31,0.3) and (6.95,1.4) .. (10.93,3.29)   ;
%Curve Lines [id:da9281274979524605] 
\draw    (329.37,75.17) .. controls (510.2,76.56) and (511.1,75.93) .. (511.99,173.66) ;
\draw [shift={(512,175.14)}, rotate = 269.48] [color={rgb, 255:red, 0; green, 0; blue, 0 }  ][line width=0.75]    (10.93,-3.29) .. controls (6.95,-1.4) and (3.31,-0.3) .. (0,0) .. controls (3.31,0.3) and (6.95,1.4) .. (10.93,3.29)   ;
%Shape: Ellipse [id:dp5747949043348641] 
\draw  [fill={rgb, 255:red, 0; green, 0; blue, 0 }  ,fill opacity=1 ] (327,75.01) .. controls (327,73.82) and (328.06,72.85) .. (329.37,72.85) .. controls (330.69,72.85) and (331.75,73.82) .. (331.75,75.01) .. controls (331.75,76.21) and (330.69,77.17) .. (329.37,77.17) .. controls (328.06,77.17) and (327,76.21) .. (327,75.01) -- cycle ;
%Shape: Ellipse [id:dp9839795244213729] 
\draw  [fill={rgb, 255:red, 208; green, 2; blue, 27 }  ,fill opacity=1 ] (424.06,188.17) .. controls (424.06,183.8) and (427.59,180.27) .. (431.96,180.27) .. controls (436.32,180.27) and (439.85,183.8) .. (439.85,188.17) .. controls (439.85,192.53) and (436.32,196.06) .. (431.96,196.06) .. controls (427.59,196.06) and (424.06,192.53) .. (424.06,188.17) -- cycle ;
%Straight Lines [id:da42311535821122725] 
\draw    (431.96,196.06) -- (431.96,216.62) ;
%Straight Lines [id:da6182167576414703] 
\draw    (431.96,216.24) -- (421.91,228.2) ;
%Straight Lines [id:da3414399202354139] 
\draw    (431.96,216.24) -- (442.5,228.2) ;
%Straight Lines [id:da8812076043056545] 
\draw    (431.96,200.55) -- (423.42,212.51) ;
%Straight Lines [id:da24598080293534452] 
\draw    (431.96,200.55) -- (440.49,212.51) ;
%Shape: Ellipse [id:dp42077037172884646] 
\draw  [fill={rgb, 255:red, 208; green, 2; blue, 27 }  ,fill opacity=1 ] (464.06,188.17) .. controls (464.06,183.8) and (467.59,180.27) .. (471.96,180.27) .. controls (476.32,180.27) and (479.85,183.8) .. (479.85,188.17) .. controls (479.85,192.53) and (476.32,196.06) .. (471.96,196.06) .. controls (467.59,196.06) and (464.06,192.53) .. (464.06,188.17) -- cycle ;
%Straight Lines [id:da4649866978988264] 
\draw    (471.96,196.06) -- (471.96,216.62) ;
%Straight Lines [id:da20370020373133868] 
\draw    (471.96,216.24) -- (461.91,228.2) ;
%Straight Lines [id:da6428389436293447] 
\draw    (471.96,216.24) -- (482.5,228.2) ;
%Straight Lines [id:da026158760314009655] 
\draw    (471.96,200.55) -- (463.42,212.51) ;
%Straight Lines [id:da012714818764540947] 
\draw    (471.96,200.55) -- (480.49,212.51) ;
%Shape: Ellipse [id:dp9386780214229125] 
\draw  [fill={rgb, 255:red, 208; green, 2; blue, 27 }  ,fill opacity=1 ] (384.06,188.17) .. controls (384.06,183.8) and (387.59,180.27) .. (391.96,180.27) .. controls (396.32,180.27) and (399.85,183.8) .. (399.85,188.17) .. controls (399.85,192.53) and (396.32,196.06) .. (391.96,196.06) .. controls (387.59,196.06) and (384.06,192.53) .. (384.06,188.17) -- cycle ;
%Straight Lines [id:da8416342905135632] 
\draw    (391.96,196.06) -- (391.96,216.62) ;
%Straight Lines [id:da971411093876879] 
\draw    (391.96,216.24) -- (381.91,228.2) ;
%Straight Lines [id:da25510054563755324] 
\draw    (391.96,216.24) -- (402.5,228.2) ;
%Straight Lines [id:da5957049019216363] 
\draw    (391.96,200.55) -- (383.42,212.51) ;
%Straight Lines [id:da7516626238989643] 
\draw    (391.96,200.55) -- (400.49,212.51) ;
%Shape: Ellipse [id:dp31349873946910845] 
\draw  [fill={rgb, 255:red, 208; green, 2; blue, 27 }  ,fill opacity=1 ] (504.06,188.17) .. controls (504.06,183.8) and (507.59,180.27) .. (511.96,180.27) .. controls (516.32,180.27) and (519.85,183.8) .. (519.85,188.17) .. controls (519.85,192.53) and (516.32,196.06) .. (511.96,196.06) .. controls (507.59,196.06) and (504.06,192.53) .. (504.06,188.17) -- cycle ;
%Straight Lines [id:da9274690734516102] 
\draw    (511.96,196.06) -- (511.96,216.62) ;
%Straight Lines [id:da009461491957432733] 
\draw    (511.96,216.24) -- (501.91,228.2) ;
%Straight Lines [id:da7157941275449011] 
\draw    (511.96,216.24) -- (522.5,228.2) ;
%Straight Lines [id:da10712131031014893] 
\draw    (511.96,200.55) -- (503.42,212.51) ;
%Straight Lines [id:da7164052851513101] 
\draw    (511.96,200.55) -- (520.49,212.51) ;
%Shape: Ellipse [id:dp7955315984240003] 
\draw  [fill={rgb, 255:red, 208; green, 2; blue, 27 }  ,fill opacity=1 ] (544.06,188.17) .. controls (544.06,183.8) and (547.59,180.27) .. (551.96,180.27) .. controls (556.32,180.27) and (559.85,183.8) .. (559.85,188.17) .. controls (559.85,192.53) and (556.32,196.06) .. (551.96,196.06) .. controls (547.59,196.06) and (544.06,192.53) .. (544.06,188.17) -- cycle ;
%Straight Lines [id:da6098435626058583] 
\draw    (551.96,196.06) -- (551.96,216.62) ;
%Straight Lines [id:da5218065763701105] 
\draw    (551.96,216.24) -- (541.91,228.2) ;
%Straight Lines [id:da5919508403198068] 
\draw    (551.96,216.24) -- (562.5,228.2) ;
%Straight Lines [id:da34102824578192137] 
\draw    (551.96,200.55) -- (543.42,212.51) ;
%Straight Lines [id:da12215875066269422] 
\draw    (551.96,200.55) -- (560.49,212.51) ;
%Shape: Ellipse [id:dp8836820382228234] 
\draw  [fill={rgb, 255:red, 208; green, 2; blue, 27 }  ,fill opacity=1 ] (593.06,188.17) .. controls (593.06,183.8) and (596.59,180.27) .. (600.96,180.27) .. controls (605.32,180.27) and (608.85,183.8) .. (608.85,188.17) .. controls (608.85,192.53) and (605.32,196.06) .. (600.96,196.06) .. controls (596.59,196.06) and (593.06,192.53) .. (593.06,188.17) -- cycle ;
%Straight Lines [id:da9730808708372152] 
\draw    (600.96,196.06) -- (600.96,216.62) ;
%Straight Lines [id:da5119326823281014] 
\draw    (600.96,216.24) -- (590.91,228.2) ;
%Straight Lines [id:da013412651363815753] 
\draw    (600.96,216.24) -- (611.5,228.2) ;
%Straight Lines [id:da0010693670034451763] 
\draw    (600.96,200.55) -- (592.42,212.51) ;
%Straight Lines [id:da3253282682883363] 
\draw    (600.96,200.55) -- (609.49,212.51) ;
%Curve Lines [id:da12457767898807792] 
\draw    (329.37,75.17) .. controls (549,76.56) and (550.1,75.93) .. (550.99,174.64) ;
\draw [shift={(551,176.14)}, rotate = 269.49] [color={rgb, 255:red, 0; green, 0; blue, 0 }  ][line width=0.75]    (10.93,-3.29) .. controls (6.95,-1.4) and (3.31,-0.3) .. (0,0) .. controls (3.31,0.3) and (6.95,1.4) .. (10.93,3.29)   ;
%Curve Lines [id:da24506091904522598] 
\draw    (329.37,75.17) .. controls (598.75,76.56) and (600.1,78.54) .. (599.02,173.7) ;
\draw [shift={(599,175.14)}, rotate = 270.66] [color={rgb, 255:red, 0; green, 0; blue, 0 }  ][line width=0.75]    (10.93,-3.29) .. controls (6.95,-1.4) and (3.31,-0.3) .. (0,0) .. controls (3.31,0.3) and (6.95,1.4) .. (10.93,3.29)   ;
%Curve Lines [id:da5780300577298003] 
\draw    (329.37,75.17) .. controls (391.79,76.56) and (392.1,75.93) .. (391.02,172.67) ;
\draw [shift={(391,174.14)}, rotate = 270.64] [color={rgb, 255:red, 0; green, 0; blue, 0 }  ][line width=0.75]    (10.93,-3.29) .. controls (6.95,-1.4) and (3.31,-0.3) .. (0,0) .. controls (3.31,0.3) and (6.95,1.4) .. (10.93,3.29)   ;

% Text Node
\draw (121.03,263.61) node [anchor=north west][inner sep=0.75pt]    {$ \begin{array}{l}
\textcolor[rgb]{0.29,0.56,0.89}{asynchronous}\\
\ \ \ \ \ players
\end{array}$};
% Text Node
\draw (463.91,263.61) node [anchor=north west][inner sep=0.75pt]    {$ \begin{array}{l}
\textcolor[rgb]{0.82,0.01,0.11}{synchronous}\\
\ \ \ \ \ players
\end{array}$};
% Text Node
\draw (569.83,191.37) node [anchor=north west][inner sep=0.75pt]    {$...$};
% Text Node
\draw (374.49,137.4) node [anchor=north west][inner sep=0.75pt]  [font=\small]  {$a_{1}$};
% Text Node
\draw (416.59,137.4) node [anchor=north west][inner sep=0.75pt]  [font=\small]  {$a_{2}$};
% Text Node
\draw (581.83,137.4) node [anchor=north west][inner sep=0.75pt]  [font=\small]  {$a_{n}$};
% Text Node
\draw (255.42,19.4) node [anchor=north west][inner sep=0.75pt]  [font=\small]  {$ \begin{array}{l}
\bm{a}\ =\ ( a_{1} ,\ a_{2} ,\ a_{3} ,\ ...,\ a_{n})\\
\ \ \ \ \ \ \ drawn\ \ from\ \ \bm{q}^t
\end{array}$};
% Text Node
\draw (456.59,137.4) node [anchor=north west][inner sep=0.75pt]  [font=\small]  {$a_{3}$};
% Text Node
\draw (496.59,137.4) node [anchor=north west][inner sep=0.75pt]  [font=\small]  {$a_{4}$};
% Text Node
\draw (534.59,137.4) node [anchor=north west][inner sep=0.75pt]  [font=\small]  {$a_{5}$};


\end{tikzpicture}
}
%\caption{Agent $i$ state dynamics}
\end{figure}
\pause

\textbf{In the long run}\\
$\hookrightarrow$ We expect players to play the action profiles drawn from the current empirical distributions to stabilize the play at an equilibrium point.

\end{frame}
% new slide ===========================================================
\begin{frame}{Algorithm Analysis}
\vspace{0.4cm}
\textbf{Mood dynamics}
\begin{columns}[c] % The "c" option specifies centered vertical alignment while the "t" option is used for top vertical alignment
\column{.5\textwidth}
\begin{figure}[H]
\centering
\includegraphics[scale=0.8]{figures/process/state_dynamic_figure_1}
\end{figure}
\column{.5\textwidth} 
\begin{figure}[H]
\centering
\hspace{0.5cm}\includegraphics[scale=0.8]{figures//process/state_dynamic_figure_2}
\end{figure}
\end{columns}
\vspace{0.4cm}
\pause

\textbf{Characteristics of the stochastic process} \pause
\begin{itemize}
\setlength{\itemindent}{1.8em}
\setlength\itemsep{-0.1em}
\item Transition probabilities depend on the regrets and a perturbation parameter $\varepsilon > 0$ \pause
\item The stochastic process $\bm{X}^t = (\bm{X}^t_1, ..., \bm{X}^t_n)$ describing players' states is a perturbed non-homogeneous Markov chain \pause
\item The Markov chain is defined over a infinitely countable state spaces \pause
%\item Currently studying the long run behaviour of the process 
\item For a regularly perturbed homogenous Markov chain, the process spends almost all time in the stochastically stable states when $\varepsilon \to 0$ 
\end{itemize}
\vspace{0.4cm}

%\noindent\rule{2cm}{0.4pt}

%{$^9$} Young, H. P. (2009). \emph{''Learning by trial and error''}. Games and economic behavior.
\end{frame}

% new slide ===========================================================
\begin{frame}
\frametitle{Simulation Results}


\begin{table}[H]
\normalsize
\centering
\begin{tabular}{c|c|c}
 & \multicolumn{1}{c|}{\textbf{D}}  & \textbf{E} \\ \hline
\textbf{A} & \multicolumn{1}{c|}{(2, 29)} & (16, 7)  \\ \hline
\textbf{B} & \multicolumn{1}{c|}{(4, 7)} & (6, 13)  \\ \hline
\textbf{C} & \multicolumn{1}{c|}{(4, 4)} & (6, 6)
\end{tabular}
\caption{Utility matrix.}
\end{table}

\end{frame}


% new slide ===========================================================
\begin{frame}
\frametitle{Simulation Results}

\vspace{0.6cm}

\begin{figure}
\centering
\begin{minipage}{.33\textwidth}
  \centering
  \vspace{-0.2cm}\includegraphics[scale=0.55]{"./figures/probabilities_PRM".pdf}
  \vspace{0.28cm}
  \captionof*{figure}{\textcolor{aaltoblue}{(a)} CPRM algorithm}
\end{minipage}%
\begin{minipage}{.33\textwidth}
  \hspace{-1.9cm}\vspace{0.02cm}\includegraphics[scale=0.55]{"./figures/probabilities_Blackwell".pdf}
    \captionof*{figure}{\textcolor{aaltoblue}{(b)} Regret minimizing strategy}
\end{minipage}%
\begin{minipage}{.33\textwidth}
  \centering
  \includegraphics[scale=0.55]{"./figures/probabilities_RM".pdf}
  \vspace{0.28cm}
  \captionof*{figure}{\textcolor{aaltoblue}{(c)} Regret-matching}
\end{minipage}
\caption{Evolution of the empirical distribution over action profiles.}
\end{figure}

\end{frame}
% new slide ===========================================================
\begin{frame}[label=simulations]
\frametitle{Simulation Results}

\vspace{0.6cm}

\begin{figure}
\centering
\begin{minipage}{.33\textwidth}
  \centering
  \vspace{-0.2cm}\includegraphics[scale=0.55]{"./figures/regrets_PRM".pdf}
  \vspace{0.28cm}
    \captionof*{figure}{\textcolor{aaltoblue}{(a)} CPRM algorithm.}
\end{minipage}%
\begin{minipage}{.33\textwidth}
  \centering
  \vspace{0.0cm}\includegraphics[scale=0.55]{"./figures/regrets_Blackwell".pdf}
    \captionof*{figure}{\textcolor{aaltoblue}{(b)} Regret minimizing strategy.}
\end{minipage}%
\begin{minipage}{.33\textwidth}
  \centering
  \includegraphics[scale=0.55]{"./figures/regrets_RM".pdf}
  \vspace{0.28cm}
  \captionof*{figure}{\textcolor{aaltoblue}{(c)} Regret-matching.}
\end{minipage}
\caption{Evolution of players' regrets for the three algorithms.}
\end{figure}
\end{frame}
% new slide ===========================================================
\begin{frame}
\frametitle{Time-varying Game}

\begin{columns}
\begin{column}{.5\textwidth}

\begin{figure}
\begin{table}[ht]
\centering
\captionsetup{justification=centering}
\resizebox{6cm}{!}{%
\begin{tabular}{ c}
\begin{tabular}{c|c|c}
 & \multicolumn{1}{c|}{\textbf{D}}  & \textbf{E} \\ \hline
\textbf{A} & \multicolumn{1}{c|}{(2, 29)} & (16, 7)  \\ \hline
\textbf{B} & \multicolumn{1}{c|}{(4, 7)} & (6, 13)  \\ \hline
\textbf{C} & \multicolumn{1}{c|}{(4, 4)} & (6, 6)
\end{tabular}
 \\[20pt]
 $\downarrow$
 \\[-1pt]
\begin{tabular}{c|c|c}
 & \multicolumn{1}{c|}{\textbf{D}}  & \textbf{E} \\ \hline
\textbf{A} & \multicolumn{1}{c|}{(2, 29, 2)} & (16, 7, 8)  \\ \hline
\textbf{B} & \multicolumn{1}{c|}{(4, 7, 2)} & (6, 13, 0)  \\ \hline
\textbf{C} & \multicolumn{1}{c|}{(4, 4, 1)} & (6, 6, 5)
\end{tabular} \quad \begin{tabular}{c|c|c}
 & \multicolumn{1}{c|}{\textbf{D}}  & \textbf{E} \\ \hline
\textbf{A} & \multicolumn{1}{c|}{(9, 4, 0)} & (4, 1, 4)  \\ \hline
\textbf{B} & \multicolumn{1}{c|}{(8, 0, 1)} & (6, 7, 2)  \\ \hline
\textbf{C} & \multicolumn{1}{c|}{(11, 9, 3)} & (2, 0, 4)
\end{tabular}\\[15pt]
\textbf{X} \hspace{3.5cm} \textbf{Y}
\\[-7pt]
 $\downarrow$
 \\[5pt]
\begin{tabular}{c|c|c}
 & \multicolumn{1}{c|}{\textbf{D}}  & \textbf{E} \\ \hline
\textbf{A} & \multicolumn{1}{c|}{(2, 29)} & (16, 7)  \\ \hline
\textbf{B} & \multicolumn{1}{c|}{(4, 7)} & (6, 13)  \\ \hline
\textbf{C} & \multicolumn{1}{c|}{(4, 4)} & (6, 6)
\end{tabular}
\\[20pt]
 $\downarrow$
 \\[-3pt]
\begin{tabular}{c|c|c}
 & \multicolumn{1}{c|}{\textbf{D}}  & \textbf{E} \\ \hline
\textbf{A} & \multicolumn{1}{c|}{(2, 29, 2)} & (16, 7, 8)  \\ \hline
\textbf{B} & \multicolumn{1}{c|}{(4, 7, 2)} & (6, 13, 0)  \\ \hline
\textbf{C} & \multicolumn{1}{c|}{(4, 4, 1)} & (6, 6, 5)
\end{tabular} \quad \begin{tabular}{c|c|c}
 & \multicolumn{1}{c|}{\textbf{D}}  & \textbf{E} \\ \hline
\textbf{A} & \multicolumn{1}{c|}{(9, 4, 0)} & (4, 1, 4)  \\ \hline
\textbf{B} & \multicolumn{1}{c|}{(8, 0, 1)} & (6, 7, 2)  \\ \hline
\textbf{C} & \multicolumn{1}{c|}{(11, 9, 3)} & (2, 0, 4)
\end{tabular}
\\[15pt]
\textbf{X} \hspace{3.5cm} \textbf{Y}
\end{tabular}}
\end{table}
\captionsetup{justification=centering}
\caption{Sequence of stage games in the dynamic case.}
\end{figure}

\end{column}
\begin{column}{.5\textwidth}
 
\begin{figure}[!ht]
  \includegraphics[scale=0.6]{"./figures/regrets_dynamic".pdf}
  \vspace{-0.2cm}
  \caption{Evolution of maximal regrets}
\end{figure}
\vspace{-0.6cm}
\begin{figure}[!ht]
  \includegraphics[scale=0.6]{"./figures/probabilities_long".pdf}
  \vspace{-0.2cm}
  \caption{Evolution of the empirical distribution}
\end{figure}
 
\end{column}
\end{columns}

\end{frame}

% new slide ===========================================================
\begin{frame}[t]
\frametitle{Application to a Congestion Game}

\vspace{-0.2cm}

\begin{columns}[c] % The "c" option specifies centered vertical alignment while the "t" option is used for top vertical alignment
\column{.5\textwidth}

\begin{figure}[!ht]
  \includegraphics[scale=0.5]{"./figures/graph".pdf}
  \caption{Network graph}
\end{figure}
\begin{figure}
\begin{table}[]
        \resizebox{3.5cm}{!}{%
        \begin{tabular}{c|c|c}
            \multicolumn{1}{c|}{\textbf{Player}}  & \textbf{\makecell{Source\\node}} & \textbf{\makecell{Destination\\node}}  \\ \hline
            1 & \multicolumn{1}{c|}{B} & F  \\ \hline
            2 & \multicolumn{1}{c|}{B} & E  \\ \hline
            3 & \multicolumn{1}{c|}{D} & B  \\ \hline
            4 & \multicolumn{1}{c|}{F} & E
        \end{tabular}}
\end{table}
\caption{Source-destination nodes for each player}
\end{figure}

\column{.5\textwidth}

\begin{figure}[!ht]
  \includegraphics[scale=0.6]{"./figures/probas_congestion".pdf}
  \vspace{-0.2cm}
  \caption{Evolution of the empirical distribution}
\end{figure}
\vspace{-0.6cm}
\begin{figure}[!ht]
  \includegraphics[scale=0.6]{"./figures/regrets_congestion".pdf}
  \vspace{-0.2cm}
  \caption{Evolution of maximal regrets}
\end{figure}

\end{columns}


\end{frame}
% new slide ===========================================================
\begin{frame}[t]


\vspace{1.7cm}
\vspace*{\fill}
\begin{center}
	{ \huge Part 2 : Constrained Correlated Equilibria }
\end{center}
\vspace*{\fill}


\end{frame}

% new slide ===========================================================
\begin{frame}
\frametitle{Constrained Correlated Equilibrium Strategies}

\begin{block}{Definition 1 -- Correlated equilibrium (Aumann, 1974)}
\vspace{-0.1cm}
A correlated equilibrium of $G$ is a pair $(d, \bm{\alpha}^*)$ where $\bm{\alpha}^* : \Omega \rightarrow \mathcal{A}$ is a correlated strategy profile such that  
\vspace{-0.1cm}
\begin{align}
\forall i \in \mathcal{N}, \forall \alpha_i^\prime \in \mathcal{S}_{i,d} \quad  \sum_{\omega\in \Omega} \bm{q}(\omega) u_i(\bm{\alpha}^*_i(\omega), \bm{\alpha}^*_{-i}(\omega)) \geq  \sum_{\omega\in \Omega} \bm{q}(\omega) u_i(\alpha^\prime_i(\omega), \bm{\alpha}^*_{-i}(\omega))
\end{align}
\end{block}

\pause

\begin{columns}
\begin{column}{0.5\textwidth}
\begin{center}
%{\textbf{Two-player game $G$}}
\end{center}
\end{column}
\begin{column}{0.5\textwidth}  %%<--- here
\begin{center}
%{\textbf{Correlation device $d$}}
\end{center}
\end{column}
\end{columns}
\begin{columns}
\begin{column}{0.45\textwidth}
\begin{table}
\resizebox{3cm}{!}{%
\begin{tabular}{c|c|c}
 & \multicolumn{1}{c|}{$P$}  & $A$ \\ \hline
$P$ & \multicolumn{1}{c|}{8, 8} & 3, 10  \\ \hline
$A$ & \multicolumn{1}{c|}{10, 3} & 0, 0
\end{tabular}}
\caption{Utility matrix of the game of Chicken.}
\end{table}
\end{column}  
\begin{column}{0.55\textwidth}  %%<--- here
\vspace{-0.55cm}
\begin{itemize}
\setlength{\itemindent}{3em}
\setlength\itemsep{-0.2em}
    \item $\Omega = \{ H, M, L\}$
    \item $\mathcal{P}_1 = \{ \{H\}, \{M, L\}\}$ and $\mathcal{P}_2 = \{ \{H, M\}, \{L\}\}$
    \item $\bm{q}(H) = \bm{q}(M) = \bm{q}(L) = \sfrac{1}{3}$
\end{itemize}

\end{column}
\end{columns} 





%\customfootnotetext{$1$}{Aumann, R. J. (1974). {''\emph{Subjectivity and correlation in randomized strategies}''}. Journal of mathematical Economics.}
%\customfootnotetext{$2$}{Boufous, O., El-Azouzi, R., Touati, M., Altman, E., Bouhtou, M., (2023). {''\emph{Constrained Correlated Equilibria}''}. ArXiv (to appear).}
\end{frame}

% new slide ===========================================================
%===============================================================================
\begin{frame}
\frametitle{Constrained Correlated Equilibrium Strategies}

\begin{columns}
\begin{column}{0.5\textwidth}
\begin{center}
%{\textbf{Two-player game $G$}}
\end{center}
\end{column}
\begin{column}{0.5\textwidth}  %%<--- here
\begin{center}
%{\textbf{Correlation device $d$}}
\end{center}
\end{column}
\end{columns}
\begin{columns}
\begin{column}{0.45\textwidth}
\begin{table}
\resizebox{3cm}{!}{%
\begin{tabular}{c|c|c}
 & \multicolumn{1}{c|}{$P$}  & $A$ \\ \hline
$P$ & \multicolumn{1}{c|}{8, 8} & 3, 10  \\ \hline
$A$ & \multicolumn{1}{c|}{10, 3} & 0, 0
\end{tabular}}
\caption{Utility matrix of the game of Chicken.}
\end{table}
\end{column}  
\begin{column}{0.55\textwidth}  %%<--- here
\vspace{-0.55cm}
\begin{itemize}
\setlength{\itemindent}{3em}
\setlength\itemsep{-0.2em}
    \item $\Omega = \{ H, M, L\}$
    \item $\mathcal{P}_1 = \{ \{H\}, \{M, L\}\}$ and $\mathcal{P}_2 = \{ \{H, M\}, \{L\}\}$
    \item $\bm{q}(H) = \bm{q}(M) = \bm{q}(L) = \sfrac{1}{3}$
\end{itemize}

\end{column}
\end{columns} 
\vspace{-0.2cm}

\pause

\begin{columns}
\begin{column}{0.5\textwidth}
\begin{center}
%{\textbf{Game $G$ extended with correlation device $d$}}
\end{center}
\end{column}
\begin{column}{0.5\textwidth}  %%<--- here
\begin{center}
%\textbf{Constrained extended game}
\end{center}
\end{column}
\end{columns}
\vspace{-0.cm}
\begin{columns}
\begin{column}{0.5\textwidth}
\begin{center}
    \begin{figure}[H]
    \centering
    \scalebox{0.8}{
    \begin{tabular}{c|c|c|c|c}
                & \makecell{${L\textcolor{white}{^c} \mapsto P}$\\${L^c\mapsto P}$}
                & \makecell{${L\textcolor{white}{^c} \mapsto A}$\\${L^c \mapsto A}$}
                & \makecell{${L\textcolor{white}{^c} \mapsto A}$\\${L^c \mapsto P}$}
                & \makecell{${L\textcolor{white}{^c} \mapsto P}$\\${L^c \mapsto A}$}\\ \hline
    \makecell{${H\textcolor{white}{^c} \mapsto P}$\\${H^c \mapsto P}$} &       $8, 8$            &           $3, 10$            &    $6.33, 8.67$        &      $4.67, 9.33$       \\ \hline
    \makecell{${H\textcolor{white}{^c} \mapsto A}$\\${H^c \mapsto A}$} &   $10, 3$           &           $0, 0$           &       $6.67, 2$        &     $3.33, 1$ \\ \hline
    \makecell{${H\textcolor{white}{^c} \mapsto A}$\\${H^c \mapsto P}$} &   $8.67, 6.33$        &      $2, 6.67$                &       \cellcolor{green!15}$7,7$             &       $3.67, 6$        \\ \hline
    \makecell{${H\textcolor{white}{^c} \mapsto P}$\\${H^c \mapsto A}$}  &   $9.33, 4.67$        &       $1, 3.33$               &     $6, 3.67$          &     $4.33, 4.33$        \\
    %$\begin{cases} \bm{H \mapsto A}\\ \bm{H^c \mapsto P}\end{cases}$ & b & c & d & e
    \end{tabular}}
    \caption{Extension of the game of Chicken.}
  \label{fig:ConstrainedExtendedPayoff}
\end{figure}
\end{center}

\end{column}
\begin{column}{0.5\textwidth}  %%<--- here
\begin{center}
\begin{figure}[H]
    \centering
    \scalebox{0.8}{
    \begin{tabular}{c|c|c|c|c}
                & \makecell{${L\textcolor{white}{^c} \mapsto P}$\\${L^c\mapsto P}$}
                & \makecell{${L\textcolor{white}{^c} \mapsto A}$\\${L^c \mapsto A}$}
                & \makecell{${L\textcolor{white}{^c} \mapsto A}$\\${L^c \mapsto P}$}
                & \makecell{${L\textcolor{white}{^c} \mapsto P}$\\${L^c \mapsto A}$}\\ \hline
    \makecell{${H\textcolor{white}{^c} \mapsto P}$\\${H^c \mapsto P}$} &       $8, 8$            &           \cellcolor{green!15}$3, 10$            &    \cellcolor{gray!20}\textcolor{gray}{\xcancel{$6.33, 8.67$}}        &      \cellcolor{gray!20}\textcolor{gray}{\xcancel{$4.67, 9.33$}}       \\ \hline
    \makecell{${H\textcolor{white}{^c} \mapsto A}$\\${H^c \mapsto A}$} &   \cellcolor{gray!20}\textcolor{gray}{\xcancel{$10, 3$}}           &           $0, 0$           &       $6.67, 2$        &     $3.33, 1$ \\ \hline
    \makecell{${H\textcolor{white}{^c} \mapsto A}$\\${H^c \mapsto P}$} &   \cellcolor{gray!20}\textcolor{gray}{\xcancel{$8.67, 6.33$}}        &       \cellcolor{gray!20}\textcolor{gray}{\xcancel{$2, 6.67$}}                &       \cellcolor{green!15}$7,7$             &       $3.67, 6$        \\ \hline
    \makecell{${H\textcolor{white}{^c} \mapsto P}$\\${H^c \mapsto A}$}  &   \cellcolor{gray!20}\textcolor{gray}{\xcancel{$9.33, 4.67$}}        &       $1, 3.33$               &     \cellcolor{gray!20}\textcolor{gray}{\xcancel{$6, 3.67$}}          &     \cellcolor{green!15}$4.33, 4.33$        \\
    %$\begin{cases} \bm{H \mapsto A}\\ \bm{H^c \mapsto P}\end{cases}$ & b & c & d & e
    \end{tabular}}
    \caption*{\textcolor{aaltoblue}{Table 2:} Constrained extension of the game of Chicken.}
    \end{figure}
\end{center}
\end{column}
\end{columns}
\end{frame}

% new slide ===========================================================
\begin{frame}
\frametitle{Constrained Correlated Equilibrium Strategies}





\begin{columns}
\begin{column}{0.5\textwidth}
\begin{center}
%{\textbf{Game $G$ extended with correlation device $d$}}
\end{center}
\end{column}
\begin{column}{0.5\textwidth}  %%<--- here
\begin{center}
%\textbf{Constrained extended game}
\end{center}
\end{column}
\end{columns}
\vspace{-0.cm}
\begin{columns}
\begin{column}{0.5\textwidth}
\begin{center}
    \begin{figure}[H]
    \centering
    \scalebox{0.8}{
    \begin{tabular}{c|c|c|c|c}
                & \makecell{${L\textcolor{white}{^c} \mapsto P}$\\${L^c\mapsto P}$}
                & \makecell{${L\textcolor{white}{^c} \mapsto A}$\\${L^c \mapsto A}$}
                & \makecell{${L\textcolor{white}{^c} \mapsto A}$\\${L^c \mapsto P}$}
                & \makecell{${L\textcolor{white}{^c} \mapsto P}$\\${L^c \mapsto A}$}\\ \hline
    \makecell{${H\textcolor{white}{^c} \mapsto P}$\\${H^c \mapsto P}$} &       $8, 8$            &           $3, 10$            &    $6.33, 8.67$        &      $4.67, 9.33$       \\ \hline
    \makecell{${H\textcolor{white}{^c} \mapsto A}$\\${H^c \mapsto A}$} &   $10, 3$           &           $0, 0$           &       $6.67, 2$        &     $3.33, 1$ \\ \hline
    \makecell{${H\textcolor{white}{^c} \mapsto A}$\\${H^c \mapsto P}$} &   $8.67, 6.33$        &      $2, 6.67$                &       \cellcolor{green!15}$7,7$             &       $3.67, 6$        \\ \hline
    \makecell{${H\textcolor{white}{^c} \mapsto P}$\\${H^c \mapsto A}$}  &   $9.33, 4.67$        &       $1, 3.33$               &     $6, 3.67$          &     $4.33, 4.33$        \\
    %$\begin{cases} \bm{H \mapsto A}\\ \bm{H^c \mapsto P}\end{cases}$ & b & c & d & e
    \end{tabular}}
    \caption*{\textcolor{aaltoblue}{Figure 13:} Extension of the game of Chicken.}
  \label{fig:ConstrainedExtendedPayoff}
\end{figure}
\end{center}

\end{column}
\begin{column}{0.5\textwidth}  %%<--- here
\begin{center}
\begin{figure}[H]
    \centering
    \scalebox{0.8}{
    \begin{tabular}{c|c|c|c|c}
                & \makecell{${L\textcolor{white}{^c} \mapsto P}$\\${L^c\mapsto P}$}
                & \makecell{${L\textcolor{white}{^c} \mapsto A}$\\${L^c \mapsto A}$}
                & \makecell{${L\textcolor{white}{^c} \mapsto A}$\\${L^c \mapsto P}$}
                & \makecell{${L\textcolor{white}{^c} \mapsto P}$\\${L^c \mapsto A}$}\\ \hline
    \makecell{${H\textcolor{white}{^c} \mapsto P}$\\${H^c \mapsto P}$} &       $8, 8$            &           \cellcolor{green!15}$3, 10$            &    \cellcolor{gray!20}\textcolor{gray}{\xcancel{$6.33, 8.67$}}        &      \cellcolor{gray!20}\textcolor{gray}{\xcancel{$4.67, 9.33$}}       \\ \hline
    \makecell{${H\textcolor{white}{^c} \mapsto A}$\\${H^c \mapsto A}$} &   \cellcolor{gray!20}\textcolor{gray}{\xcancel{$10, 3$}}           &           $0, 0$           &       $6.67, 2$        &     $3.33, 1$ \\ \hline
    \makecell{${H\textcolor{white}{^c} \mapsto A}$\\${H^c \mapsto P}$} &   \cellcolor{gray!20}\textcolor{gray}{\xcancel{$8.67, 6.33$}}        &       \cellcolor{gray!20}\textcolor{gray}{\xcancel{$2, 6.67$}}                &       \cellcolor{green!15}$7,7$             &       $3.67, 6$        \\ \hline
    \makecell{${H\textcolor{white}{^c} \mapsto P}$\\${H^c \mapsto A}$}  &   \cellcolor{gray!20}\textcolor{gray}{\xcancel{$9.33, 4.67$}}        &       $1, 3.33$               &     \cellcolor{gray!20}\textcolor{gray}{\xcancel{$6, 3.67$}}          &     \cellcolor{green!15}$4.33, 4.33$        \\
    %$\begin{cases} \bm{H \mapsto A}\\ \bm{H^c \mapsto P}\end{cases}$ & b & c & d & e
    \end{tabular}}
    \caption*{\textcolor{aaltoblue}{Figure 14:} Constrained extension of the game of Chicken.}
    \end{figure}
\end{center}
\end{column}
\end{columns}

\pause

\vspace{0.6cm}

\begin{block}{Definition 4 -- Constrained Correlated equilibrium (Boufous et al., 2024)}
\vspace{-0.1cm}
A correlated equilibrium of $G$ is a pair $(d, \bm{\alpha}^*)$ where $\bm{\alpha}^* : \Omega \rightarrow \mathcal{A}$ is a correlated strategy profile such that  
\vspace{-0.2cm}
\begin{align}
\forall i \in \mathcal{N}, \forall \alpha_i^\prime : \Omega \rightarrow \mathcal{A}_i \quad  \sum_{\omega\in \Omega} \bm{q}(\omega) u_i(\bm{\alpha}^*_i(\omega), \bm{\alpha}^*_{-i}(\omega)) \geq  \sum_{\omega\in \Omega} \bm{q}(\omega) u_i(\alpha^\prime_i(\omega), \bm{\alpha}^*_{-i}(\omega))
\end{align}
\end{block}

\end{frame}


% new slide ===========================================================
\begin{frame}
\frametitle{Properties of Constrained Correlated Equilibrium Strategies}

\pause

%Given a finite non-cooperative game $G$, a correlation device $d$ and a set of feasible correlated strategies $\mathcal{R}_d$, we have,

\vspace{0.3cm}

Consider a finite non-cooperative game ${G}=(\mathcal{N},(\mathcal{A}_{i})_{i \in \mathcal{N}},(u_{i})_{i \in \mathcal{N}})$, a correlation device $d = (\Omega, (\mathcal{P}_i)_{i \in \mathcal{N}}, \bm{q})$ and a constraint set $\mathcal{R}_d \subset \mathcal{S}_d$.

\pause

\vspace{0.5cm}

\textcolor{aaltoblue}{\large Proposition 1} : If $(d, \bm{\alpha}^*)$ is a correlated equilibrium and $\bm{\alpha}^* \in \mathcal{R}_d$, then $(d,\mathcal{R}_d,\bm{\alpha}^*)$ is a constrained correlated equilibrium.

\pause

\vspace{0.8cm}


\textcolor{aaltoblue}{\large Proposition 2} :  If $\bm{\alpha}^* \in \mathcal{R}_d$ and for any $i \in \mathcal{N}$, for any $\alpha_i^\prime$ s.t. $(\alpha_i^\prime, \bm{\alpha}_{-i}) \in \mathcal{R}_d$, for any $\omega \in \Omega$
    \vspace{0cm}
    %%%%%%%%%%%%%%%%%%%%%
    \begin{equation}\label{eq:perOutcomeStabilityCondition}
        \sum_{\omega^{\prime} \in P_{i}(\omega)} \bm{q}(\omega^{\prime}) \left[u_{i}({\alpha}_i^*(\omega), \bm{\alpha}_{-i}^*(\omega)) - u_{i}(\alpha_i^\prime(\omega^\prime), \bm{\alpha}^*_{-i}(\omega^{\prime}))\right] \geq 0
        \vspace{-0.1cm}
    \end{equation}
    %%%%%%%%%%%%%%%%%%%%%
    then $(d,\mathcal{R}_d,\bm{\alpha}^*)$ is a constrained correlated equilibrium.

\pause

\vspace{0.8cm}

\textcolor{aaltoblue}{\large Proposition 3} :
    The triplet $(d,\mathcal{R}_d,\bm{\alpha}^*)$ is a constrained correlated equilibrium if and only if $\bm{\alpha}^* \in \mathcal{R}_d$ and for any 
    $i \in \mathcal{N}$, for any $\alpha^\prime_{i} \in \mathcal{S}_{i,d}$,
    \vspace{-0cm}
    %%%%%%%%%%%%%%%%%%%
    \begin{equation}
        \sum 
        \limits_{\omega \in \Omega} 
        \bm{q}(\omega) 
        \left[ 
        u_{i}({\alpha}_i^*(\omega), \bm{\alpha}_{-i}^*(\omega)) - u_{i}(\alpha^\prime_{i}(\omega), \bm{\alpha}^*_{-i}(\omega)) 
        \right] 
        \geq 0  
        \; \text{ or } \;  (\alpha^\prime_{i}, \bm{\alpha}^*_{-i}) \notin \mathcal{R}_d
    \end{equation}
    %%%%%%%%%%%%%%%%%%%
    %where ''$\lor$'' denotes the logical inclusive "or" that means that the formula is true when either or both of the arguments are true.
\end{frame}

% new slide ===========================================================
\begin{frame}
\frametitle{Constraints induced by a set of feasible probability distributions}

\pause

\begin{itemize}
\setlength\itemsep{-0.3em}
\setlength{\itemindent}{1em}
    \item Let $\mathcal{C} \subseteq \Delta(\mathcal{A})$ be a set of probability distributions. For each correlation device $d$,
    \begin{align}
        \mathcal{R}_d = \{  \bm{\alpha}  \in \mathcal{S}_d \mid \bm{p}_{\bm{\alpha}} \in \mathcal{C} \} 
    \end{align} \pause
    \item Performance measures in many applications can be expressed in terms of probability distribution  over action profiles. Examples include applications in smart grids, wireless networks, etc.  \pause
    \begin{itemize}
    \setlength\itemsep{-0.3em}
    \setlength{\itemindent}{2em}
    \item  Constraints on the social welfare : $\sum_{i \in \mathcal{N}}\sum_{\bm{a} \in \mathcal{A}} \bm{p}(\bm{a})u_i(\bm{a}) \geq D_1$ \pause
    \item Constraints on the Nash product :
    $\prod_{i \in \mathcal{N}}\left( \sum_{\bm{a} \in \mathcal{A}} \bm{p}(\bm{a})u_i(\bm{a}) \right) \geq D_2$ \pause
    \end{itemize} 
\end{itemize}

\textcolor{aaltoblue}{\large Theorem 1 -- Characterization of the Set of Constrained Correlated Equilibrium Distributions}\\
    Let $G$ be a finite non-cooperative game and $\mathcal{C}$ a set of  feasible probability distributions. 
    The distribution $\bm{p}\in\Delta(\mathcal{A})$ is a constrained correlated equilibrium distribution 
    if and only if 
    for any player $i\in\mathcal{N}$,
    for any strategy $\beta_i : {\mathcal{A}_i} \rightarrow {\mathcal{A}_i}$, 
    if 
    $\bm{z}_{\beta_i,\bm{p}} \in \mathcal{C}$, 
    then
    \vspace{-0.1cm}
    %%%%%%%%%%%%%%%%%%%
    \begin{align}
        \sum\limits_{\bm{a} \in \mathcal{A}} \bm{p}(\bm{a})
        \left[
            u_{i}(a_i, \bm{a}_{-i}) - u_{i}(\beta_i(a_i), \bm{a}_{-i}) 
        \right] 
        \geq 0    
    \end{align}
    %%%%%%%%%%%%%%%%%%%
where $\bm{z}_{\beta_i, \bm{p}}(\bm{a}) 
    = 
    \Sigma_{b_i \in \mathcal{A}_i} \bm{p}(b_i, \bm{a}_{-i}) \mathds{1}_{\beta_i(b_i) = a_i}$ for any $\bm{a} \in \mathcal{A}$ is the distribution resulting from player $i$'s unilateral deviation $\beta_i$.
\pause

%\vspace{0.1cm}
%$\rightarrow$ Constrained correlated equilibrium distributions are obtained using the class of canonical correlation devices.

\end{frame}

% new slide ===========================================================
\begin{frame}
\frametitle{Existence of Constrained Correlated Equilibria}
\pause
\vspace{0.2cm}
Consider the following two-player game: \pause
\vspace{-0.2cm}
%%%%%%%%%%%%%%%%%%%
\begin{figure}[H]
    \normalsize
    \centering
        \scalebox{1.2}{
        \begin{tabular}{c|c|c}
                  & \multicolumn{1}{c|}{{$L$}} & {$R$}  \\ \hline
            {$\,\,\, U \,\,\,$} & \multicolumn{1}{c|}{$(2, 2)$} & $(1, 1)$  \\ \hline
            {$\,\,\, D \,\,\,$} & \multicolumn{1}{c|}{$(3, 0)$} & $(0, 5)$
        \end{tabular}}
    \caption{Two-player game in matrix form.}
    \label{tab:2x2game}
\end{figure}
%%%%%%%%%%%%%%%%%%%
\pause

Let $\mathcal{C} \subset \Delta(\mathcal{A})$ be a feasible set of probability distributions such that, \pause
\begin{align}
    \mathcal{C} = \{ \bm{p} \in \Delta(\mathcal{A}) \mid \bm{p}(U,L) = 1 \text{ or } \bm{p}(U,R) = 1 \text{ or } \bm{p}(D,L) = 1 \text{ or } \bm{p}(D,R) = 1\} 
\end{align}
\pause
%%%%%%%%%%%%%%%%%
\begin{itemize}\setlength{\itemsep}{-0.1cm}
    \setlength{\itemindent}{1em}
    \item The set of feasible strategies is the set of pure action profiles  \pause
    \item The players must play a correlated strategy profile inducing a pure action profile in $G$ \pause
    \item There does not exist a constrained correlated equilibrium for this game \pause
\end{itemize}
%%%%%%%%%%%%%%%%%
\vspace{0.3cm}

\textcolor{aaltoblue}{\large Theorem 2 - Existence of Constrained Correlated Equilibria}\\
Let $G$ be a finite non-cooperative game and $\mathcal{C}$ a feasible set of probability distributions.
If the intersection of $\mathcal{C}$ with the set of correlated equilibrium distributions of the game $G$ is non-empty, then a constrained correlated equilibrium of $G$ exists.


\end{frame}

% new slide ===========================================================
\begin{frame}
\frametitle{Constrained Correlated Equilibria of the Mixed Extension}
\pause

\vspace{0.5cm}

Define the mixed extension of $G$ by the game $\Delta G = (\mathcal{N}, (\Delta(\mathcal{A}_i))_{i \in \mathcal{N}}, (u_i)_{i \in \mathcal{N}})$ where $\Delta(\mathcal{A}_i) = \{\bm{p}\in \mathbb{R}^{|\mathcal{A}_i|}_+ \mid \sum_{a_i\in\mathcal{A}_i}\bm{p}(a_i) = 1\}$ is the set of probability distributions on $\mathcal{A}_i$. \pause
\vspace{0.15cm}
 
Let $d$ be a correlation device and $\Delta G_d$ be the extension of $\Delta G$ by $d$. The set of strategies of player $i$ in $\Delta G_d$ is
%%%%%%%%%%%%%%%%%%%%%
\begin{equation}
    \tilde{\mathcal{S}}_{i, d} = \{ \gamma_i : \Omega \rightarrow \Delta(\mathcal{A}_i)
    \mid 
    \gamma_i
    \text{  is }
    \mathcal{P}_i\text{-measurable} \} \label{eq:mixedCorrelatedStrategy}
\end{equation}
%%%%%%%%%%%%%%%%%%%%%
\pause

\vspace{0.1cm}

For any correlated strategy profile 
$\bm{\gamma} \in \tilde{\mathcal{S}}_d = \times_{i\in\mathcal{N}}\tilde{\mathcal{S}}_{i, d}$, the utility function of player 
$i$ is 
$
    \tilde{u}_i:\tilde{\mathcal{S}}_d \rightarrow \mathbb{R}
$ 
such that,
%%%%%%%%%%%%%%%%%%%
\begin{align}
     \tilde{u}_i(\gamma_i, \bm{\gamma}_{-i})
     &=
     \sum_{\omega \in \Omega}
     \bm{q}(\omega)
     u_i(\gamma_i(\omega),\bm{\alpha}_{-i}(\omega))
\end{align}
%%%%%%%%%%%%%%%%%%%
\pause
\vspace{0.1cm}

%%%%%%%%%%%%%%%%%%%%%%%%%%%%
\textcolor{aaltoblue}{\large Theorem 2} :
    %Let $G$ be a finite non-cooperative game and 
    Let $G$ be a finite non-cooperative game, $\mathcal{C}$ a feasible set of probability distributions, $d$ a correlation device and $\bm{\gamma}^*\in\tilde{\mathcal{S}}_d$ a correlated strategy profile.  
    If $(d,\tilde{\mathcal{R}}_d,\bm{\gamma}^*)$ is a constrained correlated equilibrium of $\Delta G$, then it exists a constrained correlated equilibrium $(d^\prime,\mathcal{R}_{d^\prime},\bm{\alpha}^*)$ of $G$ such that $\bm{p}_{\bm{\gamma}^*} = \bm{p}_{\bm{\alpha}^*}$.
%%%%%%%%%%%%%%%%%%%%%%%%%%%%
\pause
\vspace{0.3cm}

$\rightarrow$ The set of constrained correlated equilibrium distributions of the game $\Delta G$ is included in the set of constrained correlated equilibrium distributions of $G$.

\end{frame}

% new slide ===========================================================
\begin{frame}
\frametitle{Computation of Constrained Correlated Equilibria}

\begin{columns}
\column{0.5\textwidth}
%%%%%%%%%%%%%%%%%%
\begin{align}
    & \text {maximize } 0 \nonumber \\
    \text { s.t. } & \bm{p}(\bm{a}) \geq 0 \quad \forall \bm{a} \in \mathcal{A}, \quad \Sigma_{\bm{a} \in \mathcal{A}} \bm{p}(\bm{a}) = 1 \\
                    & \forall i \in \mathcal{N}, \forall \beta_i : \mathcal{A}_i \rightarrow \mathcal{A}_i \nonumber \\
                    & \sum\limits_{\bm{a} \in \mathcal{A}} \bm{p}(\bm{a}) \left[u_{i}(\bm{a}) - u_{i}(\beta_i(a_i), \bm{a}_{-i}) \right] \geq 0. \\ \nonumber
    \end{align}  
%%%%%%%%%%%%%%%%%%
\column{0.5\textwidth}
%%%%%%%%%%%%%%%%%%
\pause

\begin{align}
    & \text{maximize } 0 \nonumber \\
    \text { s.t. } & \bm{p}(\bm{a}) \geq 0 \quad \forall \bm{a} \in \mathcal{A}, \quad \Sigma_{\bm{a} \in \mathcal{A}} \bm{p}(\bm{a}) = 1 \\
                   & \forall i \in \mathcal{N}, \forall \beta_i : \mathcal{A}_i \rightarrow \mathcal{A}_i \textcolor{red}{\text{ s.t. } \bm{z}_{\beta_i, \bm{p}} \in \mathcal{C}} \nonumber \\
                   & \sum\limits_{\bm{a} \in \mathcal{A}} \bm{p}(\bm{a}) \left[u_{i}(\bm{a}) - u_{i}(\beta_i(a_i), \bm{a}_{-i}) \right] \geq 0,\\
                   & \textcolor{red}{\bm{p} \in \mathcal{C}}
\end{align}
%%%%%%%%%%%%%%%%%%
\end{columns}

\vspace{0.6cm}
\pause

Assuming linear constraints e.g., $\mathcal{C} = \{ \bm{p} \in \Delta(\mathcal{A}) \mid F\bm{p} \leq 0 \}$, we show that the problem of computing a constrained correlated equilibrium distribution \textbf{can be formulated as a Mixed-Integer Linear Program}. 

\end{frame}

% new slide ===========================================================
\begin{frame}
\frametitle{Simulation Example}

\begin{columns}
\begin{column}{0.3\textwidth}
    \begin{table}
    \resizebox{3cm}{!}{%
    \begin{tabular}{c|c|c}
     & \multicolumn{1}{c|}{$P$}  & $A$ \\ \hline
    $P$ & \multicolumn{1}{c|}{8, 8} & 3, 10  \\ \hline
    $A$ & \multicolumn{1}{c|}{10, 3} & 0, 0
    \end{tabular}}
    \caption*{\textcolor{aaltoblue}{Table 3:} Game of Chicken}
    \end{table}
    \vspace{0.1cm}
    \begin{align}
    	SW(\bm{q}) = \sum_{i\in\mathcal{N}} u_i(\bm{q}) 
    \end{align}
    where $\scriptstyle u_i(\bm{q}) = \sum_{\bm{a} \in \mathcal{A}} \bm{q}(\bm{a}) u_i(\bm{a})$
    \vspace{0.3cm}
    \begin{itemize}\setlength{\itemsep}{-0.2cm}
    \setlength{\itemindent}{.2in}
        \item Feasible utilities in \textcolor{ForestGreen}{green}    
        \item CE utilities in \textcolor{yellow}{yellow}
        \item \textbf{CCE} utilities in \textcolor{red}{red}        
    \end{itemize}
\end{column}
\begin{column}{0.6\textwidth}  %%<--- here
\vspace{0.25cm}
%%%%%%%%%%%%%%%%%%%%%%%%%%%%%%%%%%%%%%%%%%%%%%
%%%%%%%%%%%%%%%%%%%%%%%%%%%%%%%%
\begin{figure}
      \includegraphics[scale=0.43]{figures/P12.png}
      \caption{Sets of Utilities of the two players with $SW \geq 12$}
\end{figure}
%%%%%%%%%%%%%%%%%%%%%%%%%%%%%%%%
%%%%%%%%%%%%%%%%%%%%%%%%%%%%%%%%%%%%%%%%%%%%%%
\end{column}
\end{columns}
\end{frame}

% new slide ===========================================================
\begin{frame}
\frametitle{Simulation Results}

\begin{figure}
    \centering
    \begin{minipage}{.33\textwidth}
      \centering
      \includegraphics[scale=0.21]{figures/CCE_figures/P13.png}
      \caption{\centering Sets of Utilities of the two players with $SW \geq 13$}
      \label{fig:test1}
    \end{minipage}%
    \begin{minipage}{.33\textwidth}
      \centering
      \includegraphics[scale=0.21]{figures/CCE_figures/P14.png}
      \caption{\centering Sets of Utilities of the two players with $SW \geq 14$}
      \label{fig:test2}
    \end{minipage}
    \begin{minipage}{.33\textwidth}
      \centering
      \includegraphics[scale=0.21]{figures/CCE_figures/P15.png}
      \caption{\centering Sets of Utilities of the two players with $SW \geq 15$}
      \label{fig:test1}
    \end{minipage}%
\end{figure}
\pause

\begin{itemize}\setlength{\itemsep}{-0.1cm}
    \item The set may not be convex \pause
    \item There are constrained correlated equilibria outside the set of correlated equilibrium distributions
\end{itemize}

\end{frame}



% new slide ===========================================================
\begin{frame}
\frametitle{Conclusion \& Highlight of Contributions}

\begin{itemize}\setlength{\itemsep}{-0.1cm}
    \item \underline{\textbf{A learning rule to converge towards a correlated equilibrium distribution}}
    \begin{itemize}\setlength{\itemsep}{-0.05cm}\setlength{\itemindent}{1em}
        \item Empirical evidence of convergence of the distribution to a constrained correlated equilibrium
        \item A Markov chain models the learning algorithm 
        \item The algorithm is adaptive to changes in games e.g., payoff functions or number of players
    \end{itemize} \pause
    \item \underline{\textbf{Arbitrary constraints and arbitrary correlation device}}
    \begin{itemize}\setlength{\itemsep}{-0.05cm}
    \item \textbf{Characterizations of Equilibrium Strategies}
    \begin{itemize}\setlength{\itemsep}{-0.05cm}\setlength{\itemindent}{1em}
        \item Definition through generalized Nash equilibrium and correlated equilibrium with alternative characterizations
    \end{itemize}
    \item \textbf{Properties and relationship to (unconstrained) Correlated Equilibria}
    \begin{itemize}\setlength{\itemsep}{-0.05cm}\setlength{\itemindent}{1em}
        \item A feasible correlated equilibrium is a constrained correlated equilibrium 
        \item The set of constrained correlated equilibrium distributions may not be convex
        \item There exist constrained correlated equilibrium distributions outside the set of correlated equilibrium distributions
    \end{itemize}\pause
\end{itemize}
\item \underline{\textbf{Constraints on probability distributions}} \setlength{\itemsep}{-0.05cm}
\begin{itemize}\setlength{\itemsep}{-0.05cm}
    \item \textbf{Characterization of constrained correlated equilibrium distributions}
    \begin{itemize}\setlength{\itemsep}{-0.05cm}\setlength{\itemindent}{1em}
        \item Closed-form expression of the set of constrained correlated equilibrium distribution
    \end{itemize}
    \item \textbf{Conditions of Existence}
    \begin{itemize}\setlength{\itemsep}{-0.05cm}\setlength{\itemindent}{1em}
        \item There does not always exist a constrained correlated equilibrium in finite games
    \end{itemize}
    \item \textbf{Constrained correlated equilibrium distributions of the mixed extension of a game}
    \begin{itemize}\setlength{\itemsep}{-0.05cm}\setlength{\itemindent}{1em}
        \item Every constrained correlated equilibrium distribution of $\Delta G$ is a constrained correlated equilibrium distribution of $G$
    \end{itemize}
    \item \textbf{Computation of linearly constrained correlated equilibria}
\end{itemize}

\end{itemize}

\end{frame}

% new slide ===========================================================
\begin{frame}[t]
\frametitle{Open Perspectives}
\pause
\begin{itemize}
	\item \textbf{Research Questions}
	\begin{itemize}\setlength{\itemsep}{-0.05cm}
	        \item Are there any theoretical guarantees of convergence to the a correlated equilibrium distribution?
		\item Connection to Bayesian rationality still holds in the presence of constraints ?
		\item What happens in the case of infinite games ?
		\item Existence problem should be studied for weaker or alternative assumptions
		\item Study of the problem of computing constrained correlated equilibria (complexity, feasbility)
	\end{itemize} \pause
	\item \textbf{Some Unexplored Topics}
	\begin{itemize}\setlength{\itemsep}{-0.05cm}
		\item Learning rule leading to a specific correlated equilibrium (e.g., social welfare maximizing) 
		\item Learning with constraints
		\item Subjective correlated equilibrium
	\end{itemize} \pause
	\item \textbf{Potential applications of constrained correlated equilibria and learning correlated equilibria in various fields}
	\begin{itemize}\setlength{\itemsep}{-0.05cm}
		\item Mobile Edge Computing Systems
		\item Wireless Networks with constraints
		\item Engineering and beyond
	\end{itemize}
\end{itemize}

\end{frame}

% new slide ===========================================================
\frame{\frametitle{Thank you!}
\vspace{-0.31cm}
\begin{columns}
\begin{column}{4cm}
\begin{center}
\includegraphics[scale=0.299]{figures/questions} \\[-0.6cm]
\textbf{Questions?} 
\end{center}
\end{column}
\begin{column}{6.1cm}
  \begin{flushright}
    \textbf{For more information:}\\[0.1cm]
    {\footnotesize
    \textcolor{blue}{\url{omar.t.boufous@gmail.com}}} 
   
 \vskip0.5cm
  \end{flushright}
\end{column}
\end{columns}
}


% new slide ===========================================================
\begin{frame}[t]
\frametitle{Bibliography}

{ \small R. J. Aumann. \emph{Subjectivity and correlation in randomized strategies}. Journal of Mathematical Economics, 1974.}
\vspace{0.15cm}

{\small R. J. Aumann. \emph{Correlated equilibrium as an expression of bayesian rationality}. Econometrica, 1987. }
\vspace{0.15cm}

{ \small S. Hart \& D. Schmeidler. \emph{Existence of correlated equilibria}. Mathematics of Operations Research, 1989.}
\vspace{0.15cm}

{ \small B. von Stengel \& F. Forges. \emph{Extensive-form correlated equilibrium: Definition and computational complexity}. Mathematics of Operations Research, 2008.}
\vspace{0.15cm}

{ \small F. Forges. \emph{Correlated equilibria and communication in games}. Complex Social and Behavioral Systems: Game Theory and Agent-Based Models, 2020.}
\vspace{0.15cm}

{\small A. Brandenburger, E. Dekel, and John Geanakoplos. \emph{Correlated equilibrium with generalized information structures}. Games and Economic Behavior, 1992.}
\vspace{0.15cm}

{\small S. Grant and R. Stauber. \emph{Delegation and ambiguity in correlated equilibrium}. Games and Economic Behavior, 2022.}
\vspace{0.15cm}

{\small G. Debreu. \emph{A social equilibrium existence theorem}. Proceedings of the National Academy of Sciences, 1952.}
\vspace{0.15cm}

{\small K. J. Arrow and G Debreu. \emph{Existence of an equilibrium for a competitive economy}. Econometrica: Journal of the Econometric Society, 1954.}
\vspace{0.15cm}

{\small J. B. Rosen. \emph{Existence and Uniqueness of Equilibrium Points for Concave N-Person Games}. Econometrica, 1965.}
\vspace{0.15cm}

{\small Young, H. P. (2009). \emph{Learning by trial and error}. Games and economic behavior.}
\vspace{0.15cm}



\end{frame}

% new slide ===========================================================
\begin{frame}[t]
\frametitle{Bibliography}


{\small Ankur A Kulkarni. \emph{Games and teams with shared constraints}. Philosophical Transactions of the Royal Society A: Mathematical, Physical and Engineering Sciences, 375(2100):20160302, 2017.}
\vspace{0.15cm}

{\small Jong-Shi Pang, Gesualdo Scutari, Francisco Facchinei, and Chaoxiong Wang. \emph{Distributed power allocation with rate constraints in gaussian parallel interference channels}. IEEE Transactions on Information Theory, 54(8):3471–3489, 2008.}
\vspace{0.15cm}

{\small Daniel Nowak, Tobias Mahn, Hussein Al-Shatri, Alexandra Schwartz, and Anja Klein. \emph{A generalized nash game for mobile edge computation offloading}. In 2018 6th IEEE International Conference on Mobile Cloud Computing, Services, and Engineering (MobileCloud), pages 95–102. IEEE, 2018.}
\vspace{0.15cm}



\end{frame}

%Blanc, L et Deschamps, D (2015), Giroud, la solution au problème Benzema, France Football, Paris.


%% new slide ===========================================================
%\begin{frame}[t]
%\frametitle{State-of-the-art literature}
%
%%Copy pasting from my stored viva report from a medical science PhD viva in the UK:
%
%%The examiners further report that they have satisfied themselves that the thesis:
%
%Please tick relevant boxes
%
%\begin{itemize}\setlength{\itemsep}{-0.1cm}
%	\item is genuinely the work of the student
%	\item forms a distinct contribution to knowledge of the subject
%	\item affords evidence of originality: 1) by the discovery if new facts and/or 2) by the exercise of independent critical power
%	\item is an integrated whole and presents a coherent argument
%	\item gives critical assessment of the relevant literature
%	\item describes the method of research and its findings
%	\item includes discussion of those findings and how they advance the study
%	\item demonstrates a deep and synoptic understanding of the field of study, objectivity and the capacity for judgment in complex situations and autonomous work in that field
%	\item is satisfactory as regards literary presentation
%	\item includes a full bibliography and references
%	\item demonstrates research skills relevant to the thesis
%	\item is of a standard to merit publication in whole, in part or in a revised form
%\end{itemize}
%
%\end{frame}
%







% new slide ===========================================================
\begin{frame}
\frametitle{Computation of Constrained Correlated Equilibria}

\begin{center}
    \textbf{Mixed-Integer Linear Program}
\end{center}

\vspace{0.3cm}

\begin{align}
    \boldsymbol{p} \geq 0, \Sigma_{\bm{a} \in \mathcal{A}} \bm{p}(\bm{a}) = 1, F \boldsymbol{p} \leq 0 & & \\
    {\left[U_i-U_i B_{\beta_i}\right] \boldsymbol{p} \leq M_i \times \boldsymbol{y}_{\beta_i}} & \quad \forall i \in \mathcal{N}, \quad \forall \beta_i \\
    F B_{\beta_i} \boldsymbol{p} \geq-K\left(1-\boldsymbol{y}_{\beta_i}\right)+\delta & \quad \forall i \in \mathcal{N}, \quad \forall \beta_i \\
    F B_{\beta_i} \bm{p} \leq K y_{\beta_i} & \quad \forall i \in \mathcal{N}, \quad \forall \beta_i\\
    y_{\beta_i} \in \{ 0,1 \} & \quad \forall i \in \mathcal{N}, \quad \forall \beta_i
\end{align}
\end{frame}





% ===================================================================
%
% That's all folks! 
%
% ===================================================================
\end{document}
