\documentclass[handout,first=blue,second=blue,logo=yellowexcELEC,usenames,dvipsnames, aspectratio=169]{aaltoslides169}
%\documentclass[first=blue,second=blue,logo=blueexc]{aaltoslides}
%\documentclass[handout]{beamer}        % to compile 2x2 handouts

%\usepackage{beamerthemesplit}

\usepackage[orientation=landscape,size=custom,width=16,height=9,debug]{beamerposter} 

\usefonttheme[onlymath]{serif}

\setbeamercovered{transparent}
\setbeamertemplate{caption}[numbered]

\usepackage{amsmath,amssymb}
\usepackage{biblatex}
\addbibresource{bibliography_admission_ecc2018.bib}
\usepackage[ruled,vlined,linesnumbered]{algorithm2e}
\usepackage{arydshln}
\usepackage{geometry}
\usepackage{graphicx}
\usepackage{lmodern}
\usepackage{pifont}
\usepackage{scrextend}
\usepackage{bm}
\usepackage{amsfonts}
\usepackage{booktabs}
\usepackage{siunitx}
\usepackage{gensymb}
\usepackage{subfig}
\usepackage{makecell}
\usepackage{xfrac}
\usepackage{cancel}

\usepackage{environ}
\usepackage{tikz}
\usetikzlibrary{arrows}
\usetikzlibrary{fadings}
\usetikzlibrary{matrix}
\usetikzlibrary{chains}
\usetikzlibrary{positioning}
\usetikzlibrary{fit,shapes.geometric}
\usetikzlibrary{arrows, patterns}
\usepackage{tcolorbox}
\usepackage{multimedia}
\usepackage{hyperref}
\usepackage{tcolorbox}


\usepackage[utf8]{inputenc}
%Load useful packages
\usepackage{graphicx} % Allows including images
\usepackage{booktabs} % Allows the use of \toprule, \midrule and \bottomrule in tables
%\usepackage{subcaption}
\usepackage{subfiles}
\usepackage{url}
\usepackage{amsmath,ulem}
\usepackage{amssymb}
\usepackage{mathtools}
\newtheorem{proposition}{Proposition}
%\usetheme{AnnArbor}
\setbeamertemplate{caption}[numbered]
\usepackage{lipsum}
\newcommand{\ie}{\emph{i.e., }}
\newcommand{\blue}{\textcolor{blue}}
\newcommand\xdownarrow[1][2ex]{%
   \mathrel{\rotatebox{90}{$\xleftarrow{\rule{#1}{0pt}}$}}
}
%\usepackage[usenames,dvipsnames]{color}
%\pgfplotsset{compat=1.13}

\usepackage{makecell}
\usepackage{colortbl} % for cell coloring 
\usepackage{multirow}





% packages necessary for the danger sign
\usepackage{stackengine}
\usepackage{scalerel}
\newcommand\dangersign[1][2ex]{%
  \renewcommand\stacktype{L}%
  \scaleto{\stackon[1.3pt]{\color{red}$\triangle$}{\tiny\bfseries !}}{#1}%
}


\tikzstyle{observation} = [rectangle, rounded corners, minimum width=3cm, minimum height=0.5cm,text centered]
\tikzstyle{binG} = [rectangle, rounded corners, minimum width=3cm, minimum height=0.5cm,text centered, draw=green,thick, fill=green!10]
\tikzstyle{binB} = [rectangle, rounded corners, minimum width=3cm, minimum height=0.5cm,text centered, draw=red,thick, fill=red!10]
\tikzstyle{arrow} = [thick,->,>=stealth]
\tikzstyle{crossout} = [rectangle, rounded corners, minimum width=3cm, minimum height=0.5cm,text centered, pattern=north west lines]

\tikzset{
	invisible/.style={opacity=0},
	visible on/.style={alt={#1{}{invisible}}},
	alt/.code args={<#1>#2#3}{%
		\alt<#1>{\pgfkeysalso{#2}}{\pgfkeysalso{#3}} % \pgfkeysalso doesn't change the path
	},
}

\usetikzlibrary{shapes.arrows}

\tikzset{
    myarrow/.style={
        draw=aaltoblue,
        fill=aaltoblue!70!white,
        single arrow,
        minimum height=5.5ex,
        single arrow head extend=1ex
    }
}
\newcommand{\arrowup}{%
\tikz [baseline=-0.5ex]{\node [myarrow,rotate=90] {};}
}
\newcommand{\arrowdown}{%
\tikz [baseline=-1ex]{\node [myarrow,rotate=-90] {};}
}

\newcounter{nodemarkers}
\newcommand<>\circletext[1]{%
	\tikz[overlay,remember picture] 
	\node (marker-\arabic{nodemarkers}-a) at (0,1.5ex) {};%
	#1%
	\tikz[overlay,remember picture]
	\node (marker-\arabic{nodemarkers}-b) at (0,0){};%
	\tikz[overlay,remember picture,inner sep=2pt]
	\node#2[draw,green,ellipse,fit=(marker-\arabic{nodemarkers}-a.center) (marker-\arabic{nodemarkers}-b.center)] {};%
	\stepcounter{nodemarkers}%
}

\everymath{\displaystyle}

\changefontsizes{8pt}

\usepackage{mathtools}
\DeclarePairedDelimiter\ceil{\lceil}{\rceil}
\DeclareMathOperator*{\argmax}{arg\,max}

\setlength{\leftmargini}{0.45cm}
\setlength{\leftmarginii}{0.35cm}

 \def\tr{\mathop{\mathrm{tr}}}
  
\setbeamertemplate{section in toc}[square]
\setbeamertemplate{subsection in toc}[square]
\setbeamerfont{section number projected}{size=\large}
\setbeamercolor{section number projected}{bg=aaltoblue,fg=white}

\newcommand\aaltofootertext[3]{\def\footfrow{#1}\def\footsrow{#2}\def\foottrow{#3}}
\aaltofootertext{Omar Boufous}{\today}{\insertframenumber/\inserttotalframenumber}

% Syntax: \colorboxed[<color model>]{<color specification>}{<math formula>}
\newcommand*{\colorboxed}{}
\def\colorboxed#1#{%
  \colorboxedAux{#1}%
}
\newcommand*{\colorboxedAux}[3]{%
  % #1: optional argument for color model
  % #2: color specification
  % #3: formula
  \begingroup
    \colorlet{cb@saved}{.}%
    \color#1{#2}%
    \boxed{%
      \color{cb@saved}%
      #3%
    }%
  \endgroup
}

\newtcolorbox{adbox}[1][\hspace{-0.3cm} \textbf{Advertisement}]{
colback=white,
colbacktitle=aaltoblue!10!white,
coltitle=aaltoblue,
colframe=aaltoblue,
boxrule=1pt,
titlerule=0pt,
arc=5pt,
title={\strut#1}
}

\usepackage{pgfplots}

%\mode<presentation>{\usetheme{Warsaw}}

%\makeatletter
%\def\th@mystyle{%
%    \normalfont % body font
%    \setbeamercolor{block title example}{bg=orange,fg=white}
%    \setbeamercolor{block body example}{bg=orange!20,fg=black}
%    \def\inserttheoremblockenv{exampleblock}
%  }
%\makeatother
%\theoremstyle{mystyle}
\newtheorem*{remark}{Remark}




%List of packages
%\usepackage{amsmath}

%%%%%%%%%%%%%%%%%%%%%%%%%%%%%% Metadata %%%%%%%%%%%%%%%%%%%%%%%%%%%%%
\hypersetup
{
	%Separate multiple authors by comma
	pdfauthor={Omar Boufous},
	pdftitle={ECC presentation},
	pdfsubject={ECC presentation},
	pdfkeywords={ECC, presentation},
	colorlinks=false
}

%%%%%%%%%%%%%%%%%%%%%%%%%%%%%% Title related %%%%%%%%%%%%%%%%%%%%%%%%%%%%%%
\setbeamertemplate{subsection in toc}[default]

%The contact for one of the authors MUST be embedded on the title (see below)
\title[]{Correlated Equilibria \& Learning}
%Subtitle (if needed)
\subtitle{}
%For LICENSE, we suggest CC-BY-SA, but you are free to choose your own as long
%as the LICENSE you choose is AT LEAST as permissive as CC-BY-SA
\date[2020]{\today\\ \vspace{1cm} {PhD Defense\\}}
\author[Boufous \hspace{0.2cm} {\includegraphics[height=0.2cm, keepaspectratio]{8015.png}}]{\texorpdfstring{Omar Boufous}{}}
\institute[Orange]{Orange, Châtillon, France\\CERI/LIA, Université d’Avignon, Avignon, France\\INRIA Sophia Antipolis, France}







%%%%%%%%%%%%%%%%%%%%%%%%% Presentation begins here %%%%%%%%%%%%%%%%%%%%%%%%%
\begin{document}

\begin{frame}
	\titlepage
\end{frame}






% new slide ===========================================================
\begin{frame}[t]
\frametitle{Table of Content}

\begin{itemize}
	\item \textbf{Introduction}
	\item \textbf{Related Work}
	\item \textbf{Part 1 : Learning Correlated Equilibria}
	\begin{itemize}
		\item The Learning Problem
		\item CPRM Algorithm
		\item Convergence Properties
		\item Simulation Results
	\end{itemize}
	\item \textbf{Part 2 : Constrained Correlated Equilibria}
	\begin{itemize}
		\item Definition of Constrained Correlated Equilibria
		\item Properties of Constrained Correlated Equilibrium Strategies
		\item Constrained Correlated Equilibrium Distributions
		\item Existence of Constrained Correlated Equilibria
		\item Simulation Results
	\end{itemize}
	\item \textbf{Conclusion}
\end{itemize}

\end{frame}


% new slide ===========================================================
\begin{frame}
\frametitle{Outline}

\begin{enumerate}
    \item \textbf{Background \& Preliminaries}
    \begin{enumerate}
        \item Related work \& problem definition
        \item Highlight of contributions
        \item Notations \& model
    \end{enumerate}
    \item \textbf{Correlated Equilibria \& Properties}
    \begin{enumerate}
        \item Extended Game \& Correlated Strategies
        \item Coupled Constraints in the Extended Game
    \end{enumerate}
    \item \textbf{Constrained Correlated Equilibrium Strategies}
    \begin{enumerate}
        \item Definition \& Example
        \item Alternative Characterizations
    \end{enumerate}
    \item \textbf{Existence Conditions \& Constrained Equilibria of the Mixed Extension}
    \item \textbf{Example of an Application of Constrained Correlated Equilibria} 
    \item \textbf{Conclusions \& Perspectives}
\end{enumerate}

\end{frame}


% new slide ===========================================================
\begin{frame}
\frametitle{Outline}

\begin{enumerate}
    \item \textbf{Background \& Preliminaries}
    \begin{enumerate}
        \item Related work \& problem definition
        \item Highlight of contributions
        \item Notations \& model
    \end{enumerate}
    \item \textbf{Correlated Equilibria \& Properties}
    \begin{enumerate}
        \item Extended Game \& Correlated Strategies
        \item Coupled Constraints in the Extended Game
    \end{enumerate}
    \item \textbf{Constrained Correlated Equilibrium Strategies}
    \begin{enumerate}
        \item Definition \& Example
        \item Alternative Characterizations
    \end{enumerate}
    \item \textbf{Existence Conditions \& Constrained Equilibria of the Mixed Extension}
    \item \textbf{Example of an Application of Constrained Correlated Equilibria} 
    \item \textbf{Conclusions \& Perspectives}
\end{enumerate}

\end{frame}
% new slide ===========================================================
\begin{frame}
\frametitle{Problem \& Related work}

\textbf{Correlated Equilibria}
%%%%%%%%%%%%%%%%%%%%%%%%%%%%%%%%%%%%%%%%%%%%%%%%%%%%%%%%%%%
\begin{itemize}\setlength{\itemsep}{-0.1cm}
    \setlength{\itemindent}{.2in}
    \item Defined in (Aumann, 1974) and (Aumann, 1987)
    \item A second proof of existence and a generalization to infinite games in (Hart \& Schmeidler, 1989)
    \item Defined for the extensive form in (Von Stengel \& Forges, 2008)   
    \item Other extensions in (Forges, 2020), (Brandenburger \& Dekel, 1992) and (Grant \& Stauber, 2022)
\end{itemize}

%%%%%%%%%%%%%%%%%%%%%%%%%%%%%%%%%%%%%%%%%%%%%%%%%%%%%%%%%%%
\textbf{Constraints \& Generalized Equilibria}
\begin{itemize}\setlength{\itemsep}{-0.1cm}
    \setlength{\itemindent}{.2in}
    \item (Debreu, 1952) defines the concept of generalized equilibrium
    \item (Arrow \& Debreu, 1954) shows a proof of existence
    \item (Rosen, 1965) considers coupled constraints and shows the existence and uniqueness of equilibria in concave games
    \item Many other theoretical and applied contributions.
    %under appropriate assumptions on utilities and the feasible set of joint strategies
\end{itemize}

\vspace{0.5cm}

\noindent $\Rightarrow$ A solution concept combining \textbf{correlation and constraints has not yet been studied} in the literature \\
\noindent $\Rightarrow$ We consider this problem for \textbf{finite non-cooperative games} and \textbf{propose a solution}

\end{frame}
% new slide ===========================================================
\begin{frame}
\frametitle{Related work}

%%%%%%%%%%%%%%%%%%
\begin{tabular}{cc}
    \dangersign[7ex] & While working on this topic, another paper dealing with the same issue came out \\
    \quad & The author contacted us
\end{tabular}
%%%%%%%%%%%%%%%%%%
\vspace{0.5cm}


%{ \huge \setfont{frc}{Dear authors,\\ We came across your paper on Constrained Correlated Equilibria.\\ We are also working on the same topic!} }


\end{frame}
% new slide ===========================================================
\begin{frame}
\frametitle{Highlights of the Contributions}

\begin{enumerate}
    \item \underline{\textbf{Arbitrary constraints and arbitrary correlation device}}
    \begin{enumerate}
    \item \textbf{Characterizations of Equilibrium Strategies}
    \begin{itemize}
    \setlength{\itemindent}{1em}
        \item Definition derived from the generalized Nash equilibrium and correlated equilibrium
        \item Alternative characterizations of constrained correlated strategies
    \end{itemize}
    \item \textbf{Properties and relationship to (unconstrained) Correlated Equilibria}
    \begin{itemize}
    \setlength{\itemindent}{1em}
        \item A feasible correlated equilibrium is a constrained correlated equilibrium 
        \item The set of constrained correlated equilibrium distributions may not be convex
        \item There exist constrained correlated equilibrium distributions outside the set of correlated equilibrium distributions
    \end{itemize}
\end{enumerate}
\item \underline{\textbf{Constraints on probability distributions}}
\begin{enumerate}
    \item \textbf{Characterization of constrained correlated equilibrium distributions}
    \begin{itemize}
    \setlength{\itemindent}{1em}
        \item Closed-form expression of the set of constrained correlated equilibrium distribution
    \end{itemize}
    \item \textbf{Sufficient Conditions of Existence}
    \begin{itemize}
    \setlength{\itemindent}{1em}
        \item There does not always exist a constrained correlated equilibrium in finite games
        \item The existence of a constrained correlated equilibrium 
    \end{itemize}
    \item \textbf{Constrained correlated equilibrium distributions of the mixed extension of a game}
    \begin{itemize}
    \setlength{\itemindent}{1em}
        \item The set of constrained correlated equilibrium distribution of $\Delta G$ is included in the set of constrained correlated equilibrium of $G$
    \end{itemize}
\end{enumerate}

\end{enumerate}

\end{frame}

% new slide ===========================================================
\begin{frame}
\frametitle{Problem \& Related work}

\textbf{Correlated Equilibria}
%%%%%%%%%%%%%%%%%%%%%%%%%%%%%%%%%%%%%%%%%%%%%%%%%%%%%%%%%%%
\begin{itemize}\setlength{\itemsep}{-0.1cm}
    \setlength{\itemindent}{.2in}
    \item Defined in (Aumann, 1974) and (Aumann, 1987)
    \item A second proof of existence and a generalization to infinite games in (Hart \& Schmeidler, 1989)
    \item Defined for the extensive form in (Von Stengel \& Forges, 2008)   
    \item Other extensions in (Forges, 2020), (Brandenburger \& Dekel, 1992) and (Grant \& Stauber, 2022)
\end{itemize}

%%%%%%%%%%%%%%%%%%%%%%%%%%%%%%%%%%%%%%%%%%%%%%%%%%%%%%%%%%%
\textbf{Constraints \& Generalized Equilibria}
\begin{itemize}\setlength{\itemsep}{-0.1cm}
    \setlength{\itemindent}{.2in}
    \item (Debreu, 1952) defines the concept of generalized equilibrium
    \item (Arrow \& Debreu, 1954) shows a proof of existence
    \item (Rosen, 1965) considers coupled constraints and shows the existence and uniqueness of equilibria in concave games
    \item Many other theoretical and applied contributions.
    %under appropriate assumptions on utilities and the feasible set of joint strategies
\end{itemize}

\vspace{0.5cm}

\noindent $\Rightarrow$ A solution concept combining \textbf{correlation and constraints has not yet been studied} in the literature \\
\noindent $\Rightarrow$ We consider this problem for \textbf{finite non-cooperative games} and \textbf{propose a solution}

\end{frame}

% new slide ===========================================================
\begin{frame}[t]
\frametitle{State-of-the-art}

X

\end{frame}

% new slide ===========================================================
\begin{frame}[t]
\frametitle{Motivation}
\begin{itemize}\itemsep 0.3cm
\item Planning, on the fly, a path from a starting position such that the \textbf{robot covers every point in an initially unknown spatial environment}.
\item Currently,
\begin{itemize}\itemsep 0.1cm
\item Finding an optimal path that visits every node in a graph exactly once is \textbf{NP-hard problem}.
\item \textbf{Approximate and heuristic solutions} are usually used for the complete coverage path planning task.
\item Most methods rely on the \textbf{a priori knowledge of the map of the environment} and cope with unknown obstacles detected by range sensors.
\end{itemize}
\item Objectives:
\begin{itemize}\itemsep 0.1cm
\item \textbf{Partially or completely unknown environments} (i.e. exploration task).
\item Cover as close to 100\% of the land as possible.
\item Avoid double coverage of areas.
\item Avoid obstacles and impassable areas.
\item Be as efficient as possible, i.e., keep costs to a minimum to prevent unnecessary, wastage of time and resources
\end{itemize}
\end{itemize}
\end{frame}

% new slide ===========================================================
\begin{frame}[t]
\frametitle{Contributions}

\end{frame}

% new slide ===========================================================
\begin{frame}
\frametitle{Background \& Context}
 
\textbf{Correlated Equilibria}
%%%%%%%%%%%%%%%%%%%%%%%%%%%%%%%%%%%%%%%%%%%%%%%%%%%%%%%%%%%
\begin{itemize}\setlength{\itemsep}{-0.1cm}
    \setlength{\itemindent}{.2in}
    \item A \textbf{generalization} of Nash equilibria (Aumann, 74)(Aumann, 87) resulting from Bayesian rationality 
    \item A solution concept with appealing \textbf{computational properties} 
    \item \textbf{Regret-learning} dynamics naturally lead to correlated equilibria (Hart \& Mas-Colell, 00) 
    \item Many other theoretical and applied contributions in engineering, economics, etc. 
    %\item A second proof of existence and a generalization to infinite games in 
    \item Extensions to extensive form (Von Stengel \& Forges, 08) and infinite games (Hart \& Schmeidler, 89) 
    \item Other generalizations in (Forges, 20), (Brandenburger \& Dekel, 92) and (Grant \& Stauber, 22) 
\end{itemize}

\textbf{Constraints in Games}

\begin{itemize}\setlength{\itemsep}{-0.1cm}
    \setlength{\itemindent}{.2in}
    \item Generalized Nash equilibrium problem (Facchinei, et al., 07) (Fischer, et al., 14) 
    \item (Debreu, 52) defines the concept of generalized equilibrium 
    \item A proof of existence is presented in (Arrow \& Debreu, 54) 
    \item (Rosen, 65) shows existence and uniqueness of equilibria in concave games with coupled constraints 
    %under appropriate assumptions on utilities and the feasible set of joint strategies
    \item (Bernasconi et al., 23) introduces the concept of constrained Phi-equilibrium 
\end{itemize}
\vspace{0.2cm}

In this work,
\vspace{-0.1cm}
\begin{itemize}\setlength{\itemsep}{-0.1cm}
    \setlength{\itemindent}{.2in}
    \item We \textbf{define} and \textbf{characterize} a new solution concept called \textbf{constrained correlated equilibrium} 
    \item Several characterizations of equilibrium strategies and study their properties
    \begin{itemize}\setlength{\itemsep}{-0.1cm}
    \setlength{\itemindent}{.4in}
        \item sufficient \textbf{conditions} of \textbf{existence} in case of \textbf{constraints on probability distributions over action profiles} 
        \item characterization by \textbf{canonical correlation devices} 
        \item linearly constrained correlated equilibrium problem $\rightarrow $ \textbf{MILP}
    \end{itemize}
    %and characterize the equilibrium distributions 
    %\item study the equilibrium distributions when allowing randomization over correlated strategy profiles
\end{itemize}


\end{frame}

% new slide ===========================================================
\begin{frame}
\frametitle{Model \& Definitions}
\vspace{0.3cm}

\begin{columns}
\begin{column}{0.56\textwidth}
\; \quad Finite non-cooperative game  $G  = (\mathcal{N}, (\mathcal{A}_i)_{i \in \mathcal{N}}, (u_i)_{i \in \mathcal{N}})$ 
\begin{itemize}
\setlength{\itemindent}{2.5em}
\setlength\itemsep{-0.35em}
\item Set of players $\mathcal{N}$  
\item Action set $\mathcal{A}_i$ for each $i \in \mathcal{N}$  
\item Utility function $u_i : {\scriptscriptstyle \prod\limits_{i \in \mathcal{N}}}\mathcal{A}_i = \mathcal{A} \rightarrow \mathbb{R}$  
\end{itemize}
\end{column}
\begin{column}{0.46\textwidth}  %%<--- here
Correlation device  $d = (\Omega, (\mathcal{P}_i)_{i \in \mathcal{N}}, \bm{q})$ 
\begin{itemize}
\setlength{\itemindent}{1.2em}
\setlength\itemsep{-0.35em}
\item A sample space $\Omega$  
\item Partition $\mathcal{P}_i$ of $\Omega$  for each $i \in \mathcal{N}$   
\item Probability distribution $\bm{q}$ over $\Omega$  $\textcolor{white}{{\scriptscriptstyle \prod\limits_{i \in \mathcal{N}}}}$  
\end{itemize}
\end{column}
\end{columns}
\vspace{0.2cm}
Player \textbf{$i$'s set of strategies} is $\mathcal{S}_{i, d} = \{ f_i : \Omega \rightarrow \mathcal{A}_i \text{  s.t. }  f_i \text{ is } \mathcal{P}_i\text{-measurable} \}$ and the \textbf{set of strategy profiles} is $\mathcal{S}_{d} = \mathcal{S}_{1, d} \times  ... \times \mathcal{S}_{n, d}$. 
\vspace{-0cm}
\begin{block}{Definition 1 -- Correlated equilibrium (Aumann, 1974)} 
\vspace{-0.1cm}
A correlated equilibrium of $G$ is a pair $(d, \bm{\alpha}^*)$ where $\bm{\alpha}^* : \Omega \rightarrow \mathcal{A}$ is a correlated strategy profile such that  
\vspace{-0.2cm}
\begin{align}
\forall i \in \mathcal{N}, \forall \alpha_i^\prime : \Omega \rightarrow \mathcal{A}_i \quad  \sum_{\omega\in \Omega} \bm{q}(\omega) u_i(\bm{\alpha}^*_i(\omega), \bm{\alpha}^*_{-i}(\omega)) \geq  \sum_{\omega\in \Omega} \bm{q}(\omega) u_i(\alpha^\prime_i(\omega), \bm{\alpha}^*_{-i}(\omega))
\end{align}
\end{block}
\vspace{-0.1cm} 
Given a \textbf{constraint set} $\mathcal{R}_d \subseteq \mathcal{S}_d$, we define a constrained correlated equilibrium,
\vspace{-0cm}
\begin{block}{Definition 2 -- Constrained correlated equilibrium (Boufous et al., 2024)}%\textsuperscript{$2$}  
\vspace{-0.1cm}
A constrained correlated equilibrium of $G$ is a triplet $(d, \mathcal{R}_d, \bm{\alpha}^*)$ where $\bm{\alpha}^* : \Omega \rightarrow \mathcal{A}$ is a correlated strategy profile such that $\bm{\alpha}^*\in \mathcal{R}_d$ and 
\vspace{-0.3cm}
\begin{align}
\forall i \in \mathcal{N}, \forall \alpha_i^\prime : \Omega \rightarrow \mathcal{A}_i \text{ s.t. } (\alpha_i^\prime, \bm{\alpha}^*_{-i}) \in \mathcal{R}_d \quad  \sum_{\omega\in \Omega} \bm{q}(\omega) u_i(\bm{\alpha}^*_i(\omega), \bm{\alpha}^*_{-i}(\omega)) \geq  \sum_{\omega\in \Omega} \bm{q}(\omega) u_i(\alpha^\prime_i(\omega), \bm{\alpha}^*_{-i}(\omega))
\end{align} 
\vspace{-0.1cm}
\end{block} 
\vspace{-0.1cm}
%\customfootnotetext{$1$}{Aumann, R. J. (1974). {''\emph{Subjectivity and correlation in randomized strategies}''}. Journal of mathematical Economics.}
%\customfootnotetext{$2$}{Boufous, O., El-Azouzi, R., Touati, M., Altman, E., Bouhtou, M., (2023). {''\emph{Constrained Correlated Equilibria}''}. ArXiv (to appear).}
\end{frame}

% new slide ===========================================================
%===============================================================================
\begin{frame}
\frametitle{Example : a 2-by-2 Matrix Game}

\begin{columns}
\begin{column}{0.5\textwidth}
\begin{center}
{\textbf{Two-player game $G$}}
\end{center}
\end{column}
\begin{column}{0.5\textwidth}  %%<--- here
\begin{center}
{\textbf{Correlation device $d$}}
\end{center}
\end{column}
\end{columns}
\begin{columns}
\begin{column}{0.45\textwidth}
\begin{table}
\resizebox{3cm}{!}{%
\begin{tabular}{c|c|c}
 & \multicolumn{1}{c|}{$P$}  & $A$ \\ \hline
$P$ & \multicolumn{1}{c|}{8, 8} & 3, 10  \\ \hline
$A$ & \multicolumn{1}{c|}{10, 3} & 0, 0
\end{tabular}}
%\caption*{\textcolor{aaltoblue}{Table:} Utility matrix of the game of Chicken.}
\end{table}
\end{column}  
\begin{column}{0.55\textwidth}  %%<--- here
\vspace{-0.55cm}
\begin{itemize}
\setlength{\itemindent}{3em}
\setlength\itemsep{-0.2em}
    \item $\Omega = \{ H, M, L\}$
    \item $\mathcal{P}_1 = \{ \{H\}, \{M, L\}\}$ and $\mathcal{P}_2 = \{ \{H, M\}, \{L\}\}$
    \item $\bm{q}(H) = \bm{q}(M) = \bm{q}(L) = \sfrac{1}{3}$
\end{itemize}

\end{column}
\end{columns} 
\vspace{-0cm}

\begin{columns}
\begin{column}{0.5\textwidth}
\begin{center}
{\textbf{Game $G$ extended with correlation device $d$}}
\end{center}
\end{column}
\begin{column}{0.5\textwidth}  %%<--- here
\begin{center}
\visible<3->{\textbf{Constrained extended game}}
\end{center}
\end{column}
\end{columns}
\vspace{-0.cm}
\begin{columns}
\begin{column}{0.5\textwidth}
\begin{center}
    \begin{figure}[H]
    \centering
    \scalebox{0.8}{
    \begin{tabular}{c|c|c|c|c}
                & \makecell{${L\textcolor{white}{^c} \mapsto P}$\\${L^c\mapsto P}$}
                & \makecell{${L\textcolor{white}{^c} \mapsto A}$\\${L^c \mapsto A}$}
                & \makecell{${L\textcolor{white}{^c} \mapsto A}$\\${L^c \mapsto P}$}
                & \makecell{${L\textcolor{white}{^c} \mapsto P}$\\${L^c \mapsto A}$}\\ \hline
    \makecell{${H\textcolor{white}{^c} \mapsto P}$\\${H^c \mapsto P}$} &       $8, 8$            &           $3, 10$            &    $6.33, 8.67$        &      $4.67, 9.33$       \\ \hline
    \makecell{${H\textcolor{white}{^c} \mapsto A}$\\${H^c \mapsto A}$} &   $10, 3$           &           $0, 0$           &       $6.67, 2$        &     $3.33, 1$ \\ \hline
    \makecell{${H\textcolor{white}{^c} \mapsto A}$\\${H^c \mapsto P}$} &   $8.67, 6.33$        &      $2, 6.67$                &       \cellcolor{green!15}$7,7$             &       $3.67, 6$        \\ \hline
    \makecell{${H\textcolor{white}{^c} \mapsto P}$\\${H^c \mapsto A}$}  &   $9.33, 4.67$        &       $1, 3.33$               &     $6, 3.67$          &     $4.33, 4.33$        \\
    %$\begin{cases} \bm{H \mapsto A}\\ \bm{H^c \mapsto P}\end{cases}$ & b & c & d & e
    \end{tabular}}
    %\caption{Constrained extension of the game of Chicken.}
  \label{fig:ConstrainedExtendedPayoff}
\end{figure}
\end{center}
 
\end{column}
\begin{column}{0.5\textwidth}  %%<--- here
\begin{center}
\begin{figure}[H]
    \centering
    \scalebox{0.8}{
    \begin{tabular}{c|c|c|c|c}
                & \makecell{${L\textcolor{white}{^c} \mapsto P}$\\${L^c\mapsto P}$}
                & \makecell{${L\textcolor{white}{^c} \mapsto A}$\\${L^c \mapsto A}$}
                & \makecell{${L\textcolor{white}{^c} \mapsto A}$\\${L^c \mapsto P}$}
                & \makecell{${L\textcolor{white}{^c} \mapsto P}$\\${L^c \mapsto A}$}\\ \hline
    \makecell{${H\textcolor{white}{^c} \mapsto P}$\\${H^c \mapsto P}$} &       $8, 8$            &           \cellcolor{green!15}$3, 10$            &    \cellcolor{gray!20}\textcolor{gray}{\xcancel{$6.33, 8.67$}}        &      \cellcolor{gray!20}\textcolor{gray}{\xcancel{$4.67, 9.33$}}       \\ \hline
    \makecell{${H\textcolor{white}{^c} \mapsto A}$\\${H^c \mapsto A}$} &   \cellcolor{gray!20}\textcolor{gray}{\xcancel{$10, 3$}}           &           $0, 0$           &       $6.67, 2$        &     $3.33, 1$ \\ \hline
    \makecell{${H\textcolor{white}{^c} \mapsto A}$\\${H^c \mapsto P}$} &   \cellcolor{gray!20}\textcolor{gray}{\xcancel{$8.67, 6.33$}}        &       \cellcolor{gray!20}\textcolor{gray}{\xcancel{$2, 6.67$}}                &       \cellcolor{green!15}$7,7$             &       $3.67, 6$        \\ \hline
    \makecell{${H\textcolor{white}{^c} \mapsto P}$\\${H^c \mapsto A}$}  &   \cellcolor{gray!20}\textcolor{gray}{\xcancel{$9.33, 4.67$}}        &       $1, 3.33$               &     \cellcolor{gray!20}\textcolor{gray}{\xcancel{$6, 3.67$}}          &     \cellcolor{green!15}$4.33, 4.33$        \\
    %$\begin{cases} \bm{H \mapsto A}\\ \bm{H^c \mapsto P}\end{cases}$ & b & c & d & e
    \end{tabular}}
    \end{figure}
\end{center}
\end{column}
\end{columns}
\end{frame}


% new slide ===========================================================
\begin{frame}[t]

\vspace{1.7cm}
\vspace*{\fill}
\begin{center}
	{ \huge Part 1 : Learning Correlated Equilibria }
\end{center}
\vspace*{\fill}

\end{frame}

% new slide ===========================================================
\begin{frame}[t]
\frametitle{Motivation}


\end{frame}

% new slide ===========================================================
\begin{frame}[t]


\vspace{1.7cm}
\vspace*{\fill}
\begin{center}
	{ \huge Part 2 : Constrained Correlated Equilibria }
\end{center}
\vspace*{\fill}


\end{frame}

% new slide ===========================================================
\begin{frame}
\frametitle{Properties of Constrained Correlated Equilibrium Strategies}

%Given a finite non-cooperative game $G$, a correlation device $d$ and a set of feasible correlated strategies $\mathcal{R}_d$, we have,

\vspace{0.3cm}

Consider a finite non-cooperative game ${G}=(\mathcal{N},(\mathcal{A}_{i})_{i \in \mathcal{N}},(u_{i})_{i \in \mathcal{N}})$, a correlation device $d = (\Omega, (\mathcal{P}_i)_{i \in \mathcal{N}}, \bm{q})$ and a constraint set $\mathcal{R}_d \subset \mathcal{S}_d$.

\vspace{0.8cm}

\textcolor{aaltoblue}{\large Proposition 1} : If $(d, \bm{\alpha}^*)$ is a correlated equilibrium and $\bm{\alpha}^* \in \mathcal{R}_d$, then $(d,\mathcal{R}_d,\bm{\alpha}^*)$ is a constrained correlated equilibrium.

\vspace{0.8cm}


\textcolor{aaltoblue}{\large Proposition 2} :  If $\bm{\alpha}^* \in \mathcal{R}_d$ and for any $i \in \mathcal{N}$, for any $\alpha_i^\prime$ s.t. $(\alpha_i^\prime, \bm{\alpha}_{-i}) \in \mathcal{R}_d$, for any $\omega \in \Omega$
    \vspace{0cm}
    %%%%%%%%%%%%%%%%%%%%%
    \begin{equation}\label{eq:perOutcomeStabilityCondition}
        \sum_{\omega^{\prime} \in P_{i}(\omega)} \bm{q}(\omega^{\prime}) \left[u_{i}({\alpha}_i^*(\omega), \bm{\alpha}_{-i}^*(\omega)) - u_{i}(\alpha_i^\prime(\omega^\prime), \bm{\alpha}^*_{-i}(\omega^{\prime}))\right] \geq 0
        \vspace{-0.1cm}
    \end{equation}
    %%%%%%%%%%%%%%%%%%%%%
    then $(d,\mathcal{R}_d,\bm{\alpha}^*)$ is a constrained correlated equilibrium.

\vspace{0.8cm}

\textcolor{aaltoblue}{\large Proposition 3} :
    The triplet $(d,\mathcal{R}_d,\bm{\alpha}^*)$ is a constrained correlated equilibrium if and only if $\bm{\alpha}^* \in \mathcal{R}_d$ and for any 
    $i \in \mathcal{N}$, for any $\alpha^\prime_{i} \in \mathcal{S}_{i,d}$,
    \vspace{-0cm}
    %%%%%%%%%%%%%%%%%%%
    \begin{equation}
        \sum 
        \limits_{\omega \in \Omega} 
        \bm{q}(\omega) 
        \left[ 
        u_{i}({\alpha}_i^*(\omega), \bm{\alpha}_{-i}^*(\omega)) - u_{i}(\alpha^\prime_{i}(\omega), \bm{\alpha}^*_{-i}(\omega)) 
        \right] 
        \geq 0  
        \; \text{ or } \;  (\alpha^\prime_{i}, \bm{\alpha}^*_{-i}) \notin \mathcal{R}_d
    \end{equation}
    %%%%%%%%%%%%%%%%%%%
    %where ''$\lor$'' denotes the logical inclusive "or" that means that the formula is true when either or both of the arguments are true.

\end{frame}

% new slide ===========================================================
\begin{frame}[t]
\frametitle{Motivation}


\end{frame}

% new slide ===========================================================
\begin{frame}[t]
\frametitle{Open Perspectives}

X

\end{frame}


% new slide ===========================================================
\frame{\frametitle{Thank you!}
\vspace{-0.31cm}
\begin{columns}
\begin{column}{4cm}
\begin{center}
\includegraphics[scale=0.299]{figures/questions} \\[-0.6cm]
\textbf{Questions?} 
\end{center}
\end{column}
\begin{column}{6.1cm}
  \begin{flushright}
    \textbf{For more information:}\\[0.1cm]
    {\footnotesize
    \textcolor{blue}{\url{omar.boufous@alumni.univ-avignon.fr}}} 
   
 \vskip0.5cm
  \end{flushright}
\end{column}
\end{columns}
}

% ===================================================================
%
% That's all folks! 
%
% ===================================================================
\end{document}
