\documentclass[first=blue,second=blue,logo=yellowexcELEC,usenames,dvipsnames, aspectratio=169]{aaltoslides169}
%\documentclass[first=blue,second=blue,logo=blueexc]{aaltoslides}
% \documentclass[handout]{beamer}        % to compile 2x2 handouts

%\usepackage{beamerthemesplit}
\usepackage[orientation=landscape,size=custom,width=16,height=9,debug]{beamerposter} 


\usefonttheme[onlymath]{serif}

\setbeamercovered{transparent}
\setbeamertemplate{caption}[numbered]

\usepackage{amsmath,amssymb}
\usepackage{biblatex}
\addbibresource{bibliography_admission_ecc2018.bib}
\usepackage[ruled,vlined,linesnumbered]{algorithm2e}
\usepackage{arydshln}
\usepackage{geometry}
\usepackage{graphicx}
\usepackage{lmodern}
\usepackage{pifont}
\usepackage{scrextend}
\usepackage{bm}
\usepackage{amsfonts}
\usepackage{booktabs}
\usepackage{siunitx}
\usepackage{gensymb}
\usepackage{subfig}

\usepackage{environ}
\usepackage{tikz}
\usetikzlibrary{arrows}
\usetikzlibrary{fadings}
\usetikzlibrary{matrix}
\usetikzlibrary{chains}
\usetikzlibrary{positioning}
\usetikzlibrary{fit,shapes.geometric}
\usetikzlibrary{arrows, patterns}
\usepackage{tcolorbox}
\usepackage{multimedia}
\usepackage{hyperref}


\tikzstyle{observation} = [rectangle, rounded corners, minimum width=3cm, minimum height=0.5cm,text centered]
\tikzstyle{binG} = [rectangle, rounded corners, minimum width=3cm, minimum height=0.5cm,text centered, draw=green,thick, fill=green!10]
\tikzstyle{binB} = [rectangle, rounded corners, minimum width=3cm, minimum height=0.5cm,text centered, draw=red,thick, fill=red!10]
\tikzstyle{arrow} = [thick,->,>=stealth]
\tikzstyle{crossout} = [rectangle, rounded corners, minimum width=3cm, minimum height=0.5cm,text centered, pattern=north west lines]

\tikzset{
	invisible/.style={opacity=0},
	visible on/.style={alt={#1{}{invisible}}},
	alt/.code args={<#1>#2#3}{%
		\alt<#1>{\pgfkeysalso{#2}}{\pgfkeysalso{#3}} % \pgfkeysalso doesn't change the path
	},
}

\usetikzlibrary{shapes.arrows}


\tikzset{
    myarrow/.style={
        draw=aaltoblue,
        fill=aaltoblue!70!white,
        single arrow,
        minimum height=5.5ex,
        single arrow head extend=1ex
    }
}
\newcommand{\arrowup}{%
\tikz [baseline=-0.5ex]{\node [myarrow,rotate=90] {};}
}
\newcommand{\arrowdown}{%
\tikz [baseline=-1ex]{\node [myarrow,rotate=-90] {};}
}



\newcounter{nodemarkers}
\newcommand<>\circletext[1]{%
	\tikz[overlay,remember picture] 
	\node (marker-\arabic{nodemarkers}-a) at (0,1.5ex) {};%
	#1%
	\tikz[overlay,remember picture]
	\node (marker-\arabic{nodemarkers}-b) at (0,0){};%
	\tikz[overlay,remember picture,inner sep=2pt]
	\node#2[draw,green,ellipse,fit=(marker-\arabic{nodemarkers}-a.center) (marker-\arabic{nodemarkers}-b.center)] {};%
	\stepcounter{nodemarkers}%
}



\everymath{\displaystyle}


 
\changefontsizes{8pt}

\usepackage{mathtools}
\DeclarePairedDelimiter\ceil{\lceil}{\rceil}
\DeclareMathOperator*{\argmax}{arg\,max}

\setlength{\leftmargini}{0.45cm}
\setlength{\leftmarginii}{0.35cm}

 \def\tr{\mathop{\mathrm{tr}}}
 
\setbeamertemplate{section in toc}[square]
\setbeamertemplate{subsection in toc}[square]
\setbeamerfont{section number projected}{size=\large}
\setbeamercolor{section number projected}{bg=aaltoblue,fg=white}

\newcommand\aaltofootertext[3]{\def\footfrow{#1}\def\footsrow{#2}\def\foottrow{#3}}
\aaltofootertext{Omar Boufous}{September 2020}{\insertframenumber/\inserttotalframenumber}

% Syntax: \colorboxed[<color model>]{<color specification>}{<math formula>}
\newcommand*{\colorboxed}{}
\def\colorboxed#1#{%
  \colorboxedAux{#1}%
}
\newcommand*{\colorboxedAux}[3]{%
  % #1: optional argument for color model
  % #2: color specification
  % #3: formula
  \begingroup
    \colorlet{cb@saved}{.}%
    \color#1{#2}%
    \boxed{%
      \color{cb@saved}%
      #3%
    }%
  \endgroup
}

\newtcolorbox{adbox}[1][\hspace{-0.3cm} \textbf{Advertisement}]{
colback=white,
colbacktitle=aaltoblue!10!white,
coltitle=aaltoblue,
colframe=aaltoblue,
boxrule=1pt,
titlerule=0pt,
arc=5pt,
title={\strut#1}
}



\usepackage{pgfplots}

%\mode<presentation>{\usetheme{Warsaw}}

%\makeatletter
%\def\th@mystyle{%
%    \normalfont % body font
%    \setbeamercolor{block title example}{bg=orange,fg=white}
%    \setbeamercolor{block body example}{bg=orange!20,fg=black}
%    \def\inserttheoremblockenv{exampleblock}
%  }
%\makeatother
%\theoremstyle{mystyle}
\newtheorem*{remark}{Remark}




%List of packages
%\usepackage{amsmath}

%%%%%%%%%%%%%%%%%%%%%%%%%%%%%% Metadata %%%%%%%%%%%%%%%%%%%%%%%%%%%%%
\hypersetup
{
	%Separate multiple authors by comma
	pdfauthor={Omar Boufous},
	pdftitle={ECC presentation},
	pdfsubject={ECC presentation},
	pdfkeywords={ECC, presentation},
	colorlinks=false
}

%%%%%%%%%%%%%%%%%%%%%%%%%%%%%% Title related %%%%%%%%%%%%%%%%%%%%%%%%%%%%%%
\setbeamertemplate{subsection in toc}[default]

%The contact for one of the authors MUST be embedded on the title (see below)
\title[]{Deep Reinforcement Learning for Complete\\Coverage Path Planning in Unknown Environments}
%Subtitle (if needed)
\subtitle{}
%For LICENSE, we suggest CC-BY-SA, but you are free to choose your own as long
%as the LICENSE you choose is AT LEAST as permissive as CC-BY-SA
\date[2020]{September 10$^{th}$, 2020\\ \vspace{1cm} {Master of Science Thesis\\}}
\author[Boufous \hspace{0.2cm} {\includegraphics[height=0.2cm, keepaspectratio]{8015.png}}]{\texorpdfstring{Omar Boufous}{}}
\institute[Aalto University]{School of Electrical Engineering and Computer Science, KTH Royal Institute of Technology.\\School of Science, Aalto University.\\Institut Mines Telecom Atlantique.}

%%%%%%%%%%%%%%%%%%%%%%%%% Presentation begins here %%%%%%%%%%%%%%%%%%%%%%%%%
\begin{document}

\begin{frame}
	\titlepage
\end{frame}


% new slide ===========================================================
\begin{frame}[t]
\frametitle{State-of-the-art}

X

\end{frame}

% new slide ===========================================================
\begin{frame}[t]
\frametitle{Motivation}
\begin{itemize}\itemsep 0.3cm
\item Planning, on the fly, a path from a starting position such that the \textbf{robot covers every point in an initially unknown spatial environment}.
\item Currently,
\begin{itemize}\itemsep 0.1cm
\item Finding an optimal path that visits every node in a graph exactly once is \textbf{NP-hard problem}.
\item \textbf{Approximate and heuristic solutions} are usually used for the complete coverage path planning task.
\item Most methods rely on the \textbf{a priori knowledge of the map of the environment} and cope with unknown obstacles detected by range sensors.
\end{itemize}
\item Objectives:
\begin{itemize}\itemsep 0.1cm
\item \textbf{Partially or completely unknown environments} (i.e. exploration task).
\item Cover as close to 100\% of the land as possible.
\item Avoid double coverage of areas.
\item Avoid obstacles and impassable areas.
\item Be as efficient as possible, i.e., keep costs to a minimum to prevent unnecessary, wastage of time and resources
\end{itemize}
\end{itemize}
\end{frame}

% new slide ===========================================================
\begin{frame}[t]
\frametitle{Contributions}


\end{frame}

% new slide ===========================================================
\begin{frame}[t]

\begin{center}
	Part 1 : Learning Correlated Equilibria
\end{center}

\end{frame}

% new slide ===========================================================
\begin{frame}[t]
\frametitle{Motivation}


\end{frame}

% new slide ===========================================================
\begin{frame}[t]

\begin{center}
	Part 2 : Constrained Correlated Equilibria
\end{center}

\end{frame}

% new slide ===========================================================
\begin{frame}[t]
\frametitle{Motivation}


\end{frame}

% new slide ===========================================================
\frame{\frametitle{Thank you!}
\vspace{-0.31cm}
\begin{columns}
\begin{column}{4cm}
\begin{center}
\includegraphics[scale=0.299]{figures/questions} \\[-0.6cm]
\textbf{Questions?} 
\end{center}
\end{column}
\begin{column}{6.1cm}
  \begin{flushright}
    \textbf{For more information:}\\[0.1cm]
    {\footnotesize
    \textcolor{blue}{\url{omar.boufous@aalto.fi}}} 
   
 \vskip0.5cm
  \end{flushright}
\end{column}
\end{columns}
}

% ===================================================================
%
% That's all folks! 
%
% ===================================================================
\end{document}
