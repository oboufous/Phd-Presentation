\documentclass[first=blue,second=blue,logo=yellowexcELEC,usenames,dvipsnames, aspectratio=169, handout]{aaltoslides169}
%\documentclass[first=blue,second=blue,logo=blueexc]{aaltoslides}
% \documentclass[handout]{beamer}        % to compile 2x2 handouts

%\usepackage{beamerthemesplit}
\usepackage[orientation=landscape,size=custom,width=16,height=9,debug]{beamerposter}


\usefonttheme[onlymath]{serif}
\setbeamercovered{transparent}
\setbeamertemplate{caption}[numbered]
%\usepackage{caption}
\usepackage{xfrac} %for different types of fraction
%\usepackage{amsmath,amssymb}
\usepackage[ruled,vlined,linesnumbered]{algorithm2e}
\usepackage{arydshln}
%\usepackage{graphicx}
\usepackage{lmodern}
\usepackage{tabulary}
\usepackage{tabularray}
\usepackage{bbm}
\usepackage{pifont}
\setbeamercovered{dynamic}
\usepackage{dsfont}
\usepackage{scrextend}
\usepackage{appendixnumberbeamer}
\usepackage{bm}
\usepackage{gensymb}
\usepackage{subfig}
%\usepackage{colortbl} % for cell coloring
%\usepackage{subcaption}
%\usepackage{cancel}
%\usepackage{xcolor}
\usepackage{tikz}
\usepackage{mathrsfs}
\usepackage{extarrows}
\usepackage{frcursive} % hand writing font
%\usepackage[T1]{fontenc}
\usepackage{multirow}




\usepackage[utf8]{inputenc}
%Load useful packages
\usepackage{graphicx} % Allows including images
\usepackage{booktabs} % Allows the use of \toprule, \midrule and \bottomrule in tables
%\usepackage{subcaption}
\usepackage{subfiles}
\usepackage{cancel}
\usepackage{url}
\usepackage{amsmath,ulem}
\usepackage{amssymb}
\usepackage{mathtools}
\newtheorem{proposition}{Proposition}
%\usetheme{AnnArbor}
\setbeamertemplate{caption}[numbered]
\usepackage{tcolorbox}
\usepackage{lipsum}
\newcommand{\ie}{\emph{i.e., }}
\newcommand{\blue}{\textcolor{blue}}
\newcommand\xdownarrow[1][2ex]{%
   \mathrel{\rotatebox{90}{$\xleftarrow{\rule{#1}{0pt}}$}}
}
%\usepackage[usenames,dvipsnames]{color}
%\pgfplotsset{compat=1.13}

\usepackage{makecell}
\usepackage{colortbl} % for cell coloring 
\usepackage{multirow}



%\usepackage{ulem}
\newcommand{\soutthick}[1]{%
    \renewcommand{\ULthickness}{2.4pt}%
       \sout{#1}%
    \renewcommand{\ULthickness}{.4pt}% Resetting to ulem default
}

% ======= This is for custom footnotes
\newcommand{\customfootnotetext}[2]{{% Group to localize change to footnote
  \renewcommand{\thefootnote}{#1}% Update footnote counter representation
  \footnotetext[0]{#2}}}% Print footnote text



\colorlet{shadecolor}{gray!40}



\tikzstyle{observation} = [rectangle, rounded corners, minimum width=3cm, minimum height=0.5cm,text centered]
\tikzstyle{binG} = [rectangle, rounded corners, minimum width=3cm, minimum height=0.5cm,text centered, draw=green,thick, fill=green!10]
\tikzstyle{binB} = [rectangle, rounded corners, minimum width=3cm, minimum height=0.5cm,text centered, draw=red,thick, fill=red!10]
\tikzstyle{arrow} = [thick,->,>=stealth]
\tikzstyle{crossout} = [rectangle, rounded corners, minimum width=3cm, minimum height=0.5cm,text centered, pattern=north west lines]

\tikzset{
	invisible/.style={opacity=0},
	visible on/.style={alt={#1{}{invisible}}},
	alt/.code args={<#1>#2#3}{%
		\alt<#1>{\pgfkeysalso{#2}}{\pgfkeysalso{#3}} % \pgfkeysalso doesn't change the path
	},
}

\usetikzlibrary{shapes.arrows}


\tikzset{
    myarrow/.style={
        draw=aaltoblue,
        fill=aaltoblue!70!white,
        single arrow,
        minimum height=5.5ex,
        single arrow head extend=1ex
    }
}
\newcommand{\arrowup}{%
\tikz [baseline=-0.5ex]{\node [myarrow,rotate=90] {};}
}
\newcommand{\arrowdown}{%
\tikz [baseline=-1ex]{\node [myarrow,rotate=-90] {};}
}



\newcounter{nodemarkers}
\newcommand<>\circletext[1]{%
	\tikz[overlay,remember picture] 
	\node (marker-\arabic{nodemarkers}-a) at (0,1.5ex) {};%
	#1%
	\tikz[overlay,remember picture]
	\node (marker-\arabic{nodemarkers}-b) at (0,0){};%
	\tikz[overlay,remember picture,inner sep=2pt]
	\node#2[draw,green,ellipse,fit=(marker-\arabic{nodemarkers}-a.center) (marker-\arabic{nodemarkers}-b.center)] {};%
	\stepcounter{nodemarkers}%
}



\everymath{\displaystyle}


 
\changefontsizes{8pt}

\usepackage{mathtools}
\DeclarePairedDelimiter\ceil{\lceil}{\rceil}
\DeclareMathOperator*{\argmax}{arg\,max}

\setlength{\leftmargini}{0.45cm}
\setlength{\leftmarginii}{0.35cm}

 \def\tr{\mathop{\mathrm{tr}}}
 
\setbeamertemplate{section in toc}[square]
\setbeamertemplate{subsection in toc}[square]
\setbeamerfont{section number projected}{size=\large}
\setbeamercolor{section number projected}{bg=aaltoblue,fg=white}

\newcommand\aaltofootertext[3]{\def\footfrow{#1}\def\footsrow{#2}\def\foottrow{#3}}
\aaltofootertext{60th Annual Allerton Conference on Communication, Control, and Computing}{September 2024}{\insertframenumber/\inserttotalframenumber}

% Syntax: \colorboxed[<color model>]{<color specification>}{<math formula>}
\newcommand*{\colorboxed}{}
\def\colorboxed#1#{%
  \colorboxedAux{#1}%
}
\newcommand*{\colorboxedAux}[3]{%
  % #1: optional argument for color model
  % #2: color specification
  % #3: formula
  \begingroup
    \colorlet{cb@saved}{.}%
    \color#1{#2}%
    \boxed{%
      \color{cb@saved}%
      #3%
    }%
  \endgroup
}

\newtcolorbox{adbox}[1][\hspace{-0.3cm} \textbf{Advertisement}]{
colback=white,
colbacktitle=aaltoblue!10!white,
coltitle=aaltoblue,
colframe=aaltoblue,
boxrule=1pt,
titlerule=0pt,
arc=5pt,
title={\strut#1}
}



\usepackage{pgfplots}

%\mode<presentation>{\usetheme{Warsaw}}

%\makeatletter
%\def\th@mystyle{%
%    \normalfont % body font
%    \setbeamercolor{block title example}{bg=orange,fg=white}
%    \setbeamercolor{block body example}{bg=orange!20,fg=black}
%    \def\inserttheoremblockenv{exampleblock}
%  }
%\makeatother
%\theoremstyle{mystyle}
\newtheorem*{remark}{Remark}




%List of packages
%\usepackage{amsmath}

%%%%%%%%%%%%%%%%%%%%%%%%%%%%%% Metadata %%%%%%%%%%%%%%%%%%%%%%%%%%%%%
\hypersetup
{
	%Separate multiple authors by comma
	%pdfauthor={Omar Boufous},
	colorlinks=false
}

%%%%%%%%%%%%%%%%%%%%%%%%%%%%%% Title related %%%%%%%%%%%%%%%%%%%%%%%%%%%%%%
\setbeamertemplate{subsection in toc}[default]

%The contact for one of the authors MUST be embedded on the title (see below)
\title[]{Constrained Correlated Equilibria}
%Subtitle (if needed)
\subtitle{}
%For LICENSE, we suggest CC-BY-SA, but you are free to choose your own as long
%as the LICENSE you choose is AT LEAST as permissive as CC-BY-SA
\vspace{1cm}\date[2024]{60th Annual Allerton Conference on Communication, Control, and Computing\\[1cm]27$^{th}$ September 2024}
\author[Boufous\hspace{0.2cm} {\includegraphics[height=0.2cm, keepaspectratio]{8015.png}}]{\texorpdfstring{\textbf{Omar Boufous}$^{1, 2}$, Rachid El-Azouzi$^2$, Mikaël Touati$^1$, Eitan Altman$^{2,3}$ and Mustapha Bouhtou$^1$ {  }}{}}
\institute[Orange]{$^1$ Orange, Châtillon, France\\$^2$ CERI/LIA, Université d’Avignon, Avignon, France\\$^3$ INRIA Sophia Antipolis, France}

%%%%%%%%%%%%%%%%%%%%%%%%% Presentation begins here %%%%%%%%%%%%%%%%%%%%%%%%%
\begin{document}

\begin{frame}
	\titlepage
\end{frame}
% new slide ===========================================================
\begin{frame}
\frametitle{Background \& Context}
\pause 
\textbf{Correlated Equilibria}
%%%%%%%%%%%%%%%%%%%%%%%%%%%%%%%%%%%%%%%%%%%%%%%%%%%%%%%%%%%
\begin{itemize}\setlength{\itemsep}{-0.1cm}
    \setlength{\itemindent}{.2in}
    \item A \textbf{generalization} of Nash equilibria (Aumann, 74)(Aumann, 87) resulting from Bayesian rationality \pause
    \item A solution concept with appealing \textbf{computational properties} \pause
    \item \textbf{Regret-learning} dynamics naturally lead to correlated equilibria (Hart \& Mas-Colell, 00) \pause
    \item Many other theoretical and applied contributions in engineering, economics, etc. \pause
    %\item A second proof of existence and a generalization to infinite games in 
    \item Extensions to extensive form (Von Stengel \& Forges, 08) and infinite games (Hart \& Schmeidler, 89) \pause
    \item Other generalizations in (Forges, 20), (Brandenburger \& Dekel, 92) and (Grant \& Stauber, 22) \pause
\end{itemize}

\textbf{Constraints in Games}
\pause
\begin{itemize}\setlength{\itemsep}{-0.1cm}
    \setlength{\itemindent}{.2in}
    \item Generalized Nash equilibrium problem (Facchinei, et al., 07) (Fischer, et al., 14) \pause
    \item (Debreu, 52) defines the concept of generalized equilibrium \pause
    \item A proof of existence is presented in (Arrow \& Debreu, 54) \pause
    \item (Rosen, 65) shows existence and uniqueness of equilibria in concave games with coupled constraints \pause
    %under appropriate assumptions on utilities and the feasible set of joint strategies
    \item (Bernasconi et al., 23) introduces the concept of constrained Phi-equilibrium \pause
\end{itemize}
\vspace{0.2cm}

In this work,
\vspace{-0.1cm}
\begin{itemize}\setlength{\itemsep}{-0.1cm}
    \setlength{\itemindent}{.2in}
    \item We \textbf{define} and \textbf{characterize} a new solution concept called \textbf{constrained correlated equilibrium} 
    \item Several characterizations of equilibrium strategies and study their properties
    \begin{itemize}\setlength{\itemsep}{-0.1cm}
    \setlength{\itemindent}{.4in}
        \item sufficient \textbf{conditions} of \textbf{existence} in case of \textbf{constraints on probability distributions over action profiles} 
        \item characterization by \textbf{canonical correlation devices} 
        \item linearly constrained correlated equilibrium problem $\rightarrow $ \textbf{MILP}
    \end{itemize}
    %and characterize the equilibrium distributions 
    %\item study the equilibrium distributions when allowing randomization over correlated strategy profiles
\end{itemize}


\end{frame}
% new slide ===========================================================
\begin{frame}
\frametitle{Model \& Definitions}
\vspace{0.3cm}
\pause
\begin{columns}
\begin{column}{0.56\textwidth}
\; \quad Finite non-cooperative game  $G  = (\mathcal{N}, (\mathcal{A}_i)_{i \in \mathcal{N}}, (u_i)_{i \in \mathcal{N}})$ 
\begin{itemize}\pause
\setlength{\itemindent}{2.5em}
\setlength\itemsep{-0.35em}
\item Set of players $\mathcal{N}$  \pause
\item Action set $\mathcal{A}_i$ for each $i \in \mathcal{N}$  \pause
\item Utility function $u_i : {\scriptscriptstyle \prod\limits_{i \in \mathcal{N}}}\mathcal{A}_i = \mathcal{A} \rightarrow \mathbb{R}$  \pause
\end{itemize}
\end{column}
\begin{column}{0.46\textwidth}  %%<--- here
Correlation device  $d = (\Omega, (\mathcal{P}_i)_{i \in \mathcal{N}}, \bm{q})$ \pause
\begin{itemize}
\setlength{\itemindent}{1.2em}
\setlength\itemsep{-0.35em}
\item A sample space $\Omega$  \pause
\item Partition $\mathcal{P}_i$ of $\Omega$  for each $i \in \mathcal{N}$   \pause
\item Probability distribution $\bm{q}$ over $\Omega$  $\textcolor{white}{{\scriptscriptstyle \prod\limits_{i \in \mathcal{N}}}}$  \pause
\end{itemize}
\end{column}
\end{columns}
\vspace{0.2cm}
Player \textbf{$i$'s set of strategies} is $\mathcal{S}_{i, d} = \{ f_i : \Omega \rightarrow \mathcal{A}_i \text{  s.t. }  f_i \text{ is } \mathcal{P}_i\text{-measurable} \}$ and the \textbf{set of strategy profiles} is $\mathcal{S}_{d} = \mathcal{S}_{1, d} \times  ... \times \mathcal{S}_{n, d}$. \pause
\vspace{-0cm}
\begin{block}{Definition 1 -- Correlated equilibrium (Aumann, 1974)} 
\vspace{-0.1cm}
A correlated equilibrium of $G$ is a pair $(d, \bm{\alpha}^*)$ where $\bm{\alpha}^* : \Omega \rightarrow \mathcal{A}$ is a correlated strategy profile such that  
\vspace{-0.2cm}
\begin{align}
\forall i \in \mathcal{N}, \forall \alpha_i^\prime : \Omega \rightarrow \mathcal{A}_i \quad  \sum_{\omega\in \Omega} \bm{q}(\omega) u_i(\bm{\alpha}^*_i(\omega), \bm{\alpha}^*_{-i}(\omega)) \geq  \sum_{\omega\in \Omega} \bm{q}(\omega) u_i(\alpha^\prime_i(\omega), \bm{\alpha}^*_{-i}(\omega))
\end{align}
\end{block}\pause
\vspace{-0.1cm} 
Given a \textbf{constraint set} $\mathcal{R}_d \subseteq \mathcal{S}_d$, we define a constrained correlated equilibrium,\pause
\vspace{-0cm}
\begin{block}{Definition 2 -- Constrained correlated equilibrium (Boufous et al., 2024)}%\textsuperscript{$2$}  
\vspace{-0.1cm}
A constrained correlated equilibrium of $G$ is a triplet $(d, \mathcal{R}_d, \bm{\alpha}^*)$ where $\bm{\alpha}^* : \Omega \rightarrow \mathcal{A}$ is a correlated strategy profile such that $\textcolor<14->{red}{\bm{\alpha}^*\in \mathcal{R}_d}$ and 
\vspace{-0.3cm}
\begin{align}
\forall i \in \mathcal{N}, \forall \alpha_i^\prime : \Omega \rightarrow \mathcal{A}_i \text{ s.t. } \textcolor<15->{red}{(\alpha_i^\prime, \bm{\alpha}^*_{-i}) \in \mathcal{R}_d} \quad  \sum_{\omega\in \Omega} \bm{q}(\omega) u_i(\bm{\alpha}^*_i(\omega), \bm{\alpha}^*_{-i}(\omega)) \geq  \sum_{\omega\in \Omega} \bm{q}(\omega) u_i(\alpha^\prime_i(\omega), \bm{\alpha}^*_{-i}(\omega))
\end{align}\pause \pause
\vspace{-0.1cm}
\end{block} 
\vspace{-0.1cm}
%\customfootnotetext{$1$}{Aumann, R. J. (1974). {''\emph{Subjectivity and correlation in randomized strategies}''}. Journal of mathematical Economics.}
%\customfootnotetext{$2$}{Boufous, O., El-Azouzi, R., Touati, M., Altman, E., Bouhtou, M., (2023). {''\emph{Constrained Correlated Equilibria}''}. ArXiv (to appear).}
\end{frame}
%===============================================================================
\begin{frame}
\frametitle{Example : a 2-by-2 Matrix Game}

\begin{columns}
\begin{column}{0.5\textwidth}
\begin{center}
{\textbf{Two-player game $G$}}
\end{center}
\end{column}
\begin{column}{0.5\textwidth}  %%<--- here
\begin{center}
{\textbf{Correlation device $d$}}
\end{center}
\end{column}
\end{columns}
\begin{columns}
\begin{column}{0.45\textwidth}
\begin{table}
\resizebox{3cm}{!}{%
\begin{tabular}{c|c|c}
 & \multicolumn{1}{c|}{$P$}  & $A$ \\ \hline
$P$ & \multicolumn{1}{c|}{8, 8} & 3, 10  \\ \hline
$A$ & \multicolumn{1}{c|}{10, 3} & 0, 0
\end{tabular}}
%\caption*{\textcolor{aaltoblue}{Table:} Utility matrix of the game of Chicken.}
\end{table}
\end{column}  
\begin{column}{0.55\textwidth}  %%<--- here
\vspace{-0.55cm}
\begin{itemize}
\setlength{\itemindent}{3em}
\setlength\itemsep{-0.2em}
    \item $\Omega = \{ H, M, L\}$
    \item $\mathcal{P}_1 = \{ \{H\}, \{M, L\}\}$ and $\mathcal{P}_2 = \{ \{H, M\}, \{L\}\}$
    \item $\bm{q}(H) = \bm{q}(M) = \bm{q}(L) = \sfrac{1}{3}$
\end{itemize}
\pause
\end{column}
\end{columns} 
\vspace{-0cm}

\begin{columns}
\begin{column}{0.5\textwidth}
\begin{center}
{\textbf{Game $G$ extended with correlation device $d$}}
\end{center}
\end{column}
\begin{column}{0.5\textwidth}  %%<--- here
\begin{center}
\visible<3->{\textbf{Constrained extended game}}
\end{center}
\end{column}
\end{columns}
\vspace{-0.cm}
\begin{columns}
\begin{column}{0.5\textwidth}
\begin{center}
    \begin{figure}[H]
    \centering
    \scalebox{0.8}{
    \begin{tabular}{c|c|c|c|c}
                & \makecell{${L\textcolor{white}{^c} \mapsto P}$\\${L^c\mapsto P}$}
                & \makecell{${L\textcolor{white}{^c} \mapsto A}$\\${L^c \mapsto A}$}
                & \makecell{${L\textcolor{white}{^c} \mapsto A}$\\${L^c \mapsto P}$}
                & \makecell{${L\textcolor{white}{^c} \mapsto P}$\\${L^c \mapsto A}$}\\ \hline
    \makecell{${H\textcolor{white}{^c} \mapsto P}$\\${H^c \mapsto P}$} &       $8, 8$            &           $3, 10$            &    $6.33, 8.67$        &      $4.67, 9.33$       \\ \hline
    \makecell{${H\textcolor{white}{^c} \mapsto A}$\\${H^c \mapsto A}$} &   $10, 3$           &           $0, 0$           &       $6.67, 2$        &     $3.33, 1$ \\ \hline
    \makecell{${H\textcolor{white}{^c} \mapsto A}$\\${H^c \mapsto P}$} &   $8.67, 6.33$        &      $2, 6.67$                &       \cellcolor{green!15}$7,7$             &       $3.67, 6$        \\ \hline
    \makecell{${H\textcolor{white}{^c} \mapsto P}$\\${H^c \mapsto A}$}  &   $9.33, 4.67$        &       $1, 3.33$               &     $6, 3.67$          &     $4.33, 4.33$        \\
    %$\begin{cases} \bm{H \mapsto A}\\ \bm{H^c \mapsto P}\end{cases}$ & b & c & d & e
    \end{tabular}}
    %\caption{Constrained extension of the game of Chicken.}
  \label{fig:ConstrainedExtendedPayoff}
\end{figure}
\end{center}
\pause 
\end{column}
\begin{column}{0.5\textwidth}  %%<--- here
\begin{center}
\begin{figure}[H]
    \centering
    \scalebox{0.8}{
    \begin{tabular}{c|c|c|c|c}
                & \makecell{${L\textcolor{white}{^c} \mapsto P}$\\${L^c\mapsto P}$}
                & \makecell{${L\textcolor{white}{^c} \mapsto A}$\\${L^c \mapsto A}$}
                & \makecell{${L\textcolor{white}{^c} \mapsto A}$\\${L^c \mapsto P}$}
                & \makecell{${L\textcolor{white}{^c} \mapsto P}$\\${L^c \mapsto A}$}\\ \hline
    \makecell{${H\textcolor{white}{^c} \mapsto P}$\\${H^c \mapsto P}$} &       $8, 8$            &           \cellcolor{green!15}$3, 10$            &    \cellcolor{gray!20}\textcolor{gray}{\xcancel{$6.33, 8.67$}}        &      \cellcolor{gray!20}\textcolor{gray}{\xcancel{$4.67, 9.33$}}       \\ \hline
    \makecell{${H\textcolor{white}{^c} \mapsto A}$\\${H^c \mapsto A}$} &   \cellcolor{gray!20}\textcolor{gray}{\xcancel{$10, 3$}}           &           $0, 0$           &       $6.67, 2$        &     $3.33, 1$ \\ \hline
    \makecell{${H\textcolor{white}{^c} \mapsto A}$\\${H^c \mapsto P}$} &   \cellcolor{gray!20}\textcolor{gray}{\xcancel{$8.67, 6.33$}}        &       \cellcolor{gray!20}\textcolor{gray}{\xcancel{$2, 6.67$}}                &       \cellcolor{green!15}$7,7$             &       $3.67, 6$        \\ \hline
    \makecell{${H\textcolor{white}{^c} \mapsto P}$\\${H^c \mapsto A}$}  &   \cellcolor{gray!20}\textcolor{gray}{\xcancel{$9.33, 4.67$}}        &       $1, 3.33$               &     \cellcolor{gray!20}\textcolor{gray}{\xcancel{$6, 3.67$}}          &     \cellcolor{green!15}$4.33, 4.33$        \\
    %$\begin{cases} \bm{H \mapsto A}\\ \bm{H^c \mapsto P}\end{cases}$ & b & c & d & e
    \end{tabular}}
    \end{figure}
\end{center}
\end{column}
\end{columns}
\end{frame}
% new slide ===========================================================
\begin{frame}
\frametitle{Properties of Constrained Correlated Equilibrium Strategies}

%Given a finite non-cooperative game $G$, a correlation device $d$ and a set of feasible correlated strategies $\mathcal{R}_d$, we have,

\vspace{0.3cm}

Consider a finite non-cooperative game ${G}=(\mathcal{N},(\mathcal{A}_{i})_{i \in \mathcal{N}},(u_{i})_{i \in \mathcal{N}})$, a correlation device $d = (\Omega, (\mathcal{P}_i)_{i \in \mathcal{N}}, \bm{q})$ and a constraint set $\mathcal{R}_d \subset \mathcal{S}_d$.
\pause
\vspace{0.8cm}

\textcolor{aaltoblue}{\large Proposition 1} : If $(d, \bm{\alpha}^*)$ is a correlated equilibrium and $\bm{\alpha}^* \in \mathcal{R}_d$, then $(d,\mathcal{R}_d,\bm{\alpha}^*)$ is a constrained correlated equilibrium.

\vspace{0.8cm}
\pause

\textcolor{aaltoblue}{\large Proposition 2} :  If $\bm{\alpha}^* \in \mathcal{R}_d$ and for any $i \in \mathcal{N}$, for any $\alpha_i^\prime$ s.t. $(\alpha_i^\prime, \bm{\alpha}_{-i}) \in \mathcal{R}_d$, for any $\omega \in \Omega$
    \vspace{0cm}
    %%%%%%%%%%%%%%%%%%%%%
    \begin{equation}\label{eq:perOutcomeStabilityCondition}
        \sum_{\omega^{\prime} \in P_{i}(\omega)} \bm{q}(\omega^{\prime}) \left[u_{i}({\alpha}_i^*(\omega), \bm{\alpha}_{-i}^*(\omega)) - u_{i}(\alpha_i^\prime(\omega^\prime), \bm{\alpha}^*_{-i}(\omega^{\prime}))\right] \geq 0
        \vspace{-0.1cm}
    \end{equation}
    %%%%%%%%%%%%%%%%%%%%%
    then $(d,\mathcal{R}_d,\bm{\alpha}^*)$ is a constrained correlated equilibrium.
\pause
\vspace{0.8cm}

\textcolor{aaltoblue}{\large Proposition 3} :
    The triplet $(d,\mathcal{R}_d,\bm{\alpha}^*)$ is a constrained correlated equilibrium if and only if $\bm{\alpha}^* \in \mathcal{R}_d$ and for any 
    $i \in \mathcal{N}$, for any $\alpha^\prime_{i} \in \mathcal{S}_{i,d}$,
    \vspace{-0cm}
    %%%%%%%%%%%%%%%%%%%
    \begin{equation}
        \sum 
        \limits_{\omega \in \Omega} 
        \bm{q}(\omega) 
        \left[ 
        u_{i}({\alpha}_i^*(\omega), \bm{\alpha}_{-i}^*(\omega)) - u_{i}(\alpha^\prime_{i}(\omega), \bm{\alpha}^*_{-i}(\omega)) 
        \right] 
        \geq 0  
        \; \text{ or } \;  (\alpha^\prime_{i}, \bm{\alpha}^*_{-i}) \notin \mathcal{R}_d
    \end{equation}
    %%%%%%%%%%%%%%%%%%%
    %where ''$\lor$'' denotes the logical inclusive "or" that means that the formula is true when either or both of the arguments are true.

\end{frame}
% new slide ===========================================================
\begin{frame}
\frametitle{Constraints on Probability Distributions}

%Constraint sets induced by a set of feasible probability distributions over action profiles. 
\pause
\begin{itemize}
\setlength\itemsep{-0.2em}
\setlength{\itemindent}{1em}
    \item Let $\mathcal{C} \subseteq \Delta(\mathcal{A})$ be a set of probability distributions. For each correlation device $d$,
    \begin{align}
        \mathcal{R}_d = \{  \bm{\alpha}  \in \mathcal{S}_d \mid \bm{p}_{\bm{\alpha}} \in \mathcal{C} \} \pause
    \end{align}
    \item Performance measures in many applications can be expressed in terms of probability distribution  over action profiles. Examples include applications in smart grids, wireless networks, etc. \pause
    \begin{itemize}
    \setlength\itemsep{-0.2em}
    \setlength{\itemindent}{2em}
    \item  Constraints on the social welfare : $\sum_{i \in \mathcal{N}}\sum_{\bm{a} \in \mathcal{A}} \bm{p}(\bm{a})u_i(\bm{a}) \geq D_1$ \pause
    \item Constraints on the Nash product :
    $\prod_{i \in \mathcal{N}}\left( \sum_{\bm{a} \in \mathcal{A}} \bm{p}(\bm{a})u_i(\bm{a}) \right) \geq D_2$
    \end{itemize} 
\end{itemize}
\pause

\textcolor{aaltoblue}{\large Theorem 1 -- Characterization of the Set of Constrained Correlated Equilibrium Distributions}\\
    Let $G$ be a finite non-cooperative game and $\mathcal{C}$ a set of  feasible probability distributions. 
    The distribution $\bm{p}\in\Delta(\mathcal{A})$ is a constrained correlated equilibrium distribution 
    if and only if 
    for any player $i\in\mathcal{N}$,
    for any strategy $\beta_i : {\mathcal{A}_i} \rightarrow {\mathcal{A}_i}$, 
    if 
    $\bm{z}_{\beta_i,\bm{p}} \in \mathcal{C}$, 
    then
    \vspace{-0.1cm}
    %%%%%%%%%%%%%%%%%%%
    \begin{align}
        \sum\limits_{\bm{a} \in \mathcal{A}} \bm{p}(\bm{a})
        \left[
            u_{i}(a_i, \bm{a}_{-i}) - u_{i}(\beta_i(a_i), \bm{a}_{-i}) 
        \right] 
        \geq 0    
    \end{align}
    %%%%%%%%%%%%%%%%%%%
where $\bm{z}_{\beta_i, \bm{p}}(\bm{a}) 
    = 
    \Sigma_{b_i \in \mathcal{A}_i} \bm{p}(b_i, \bm{a}_{-i}) \mathds{1}_{\beta_i(b_i) = a_i}$ for any $\bm{a} \in \mathcal{A}$ is the distribution resulting from player $i$'s unilateral deviation $\beta_i$.
\end{frame}
% new slide ===========================================================
\begin{frame}
\frametitle{Existence of Constrained Correlated Equilibria}
\vspace{0.2cm}
Consider the following two-player game: 
\vspace{-0.2cm}
%%%%%%%%%%%%%%%%%%%
\begin{figure}[H]
    \normalsize
    \centering
        \scalebox{1.2}{
        \begin{tabular}{c|c|c}
                  & \multicolumn{1}{c|}{{$L$}} & {$R$}  \\ \hline
            {$\,\,\, U \,\,\,$} & \multicolumn{1}{c|}{$(2, 2)$} & $(1, 1)$  \\ \hline
            {$\,\,\, D \,\,\,$} & \multicolumn{1}{c|}{$(3, 0)$} & $(0, 5)$
        \end{tabular}}
    \caption{Two-player game in matrix form.}
    \label{tab:2x2game}
\end{figure}
%%%%%%%%%%%%%%%%%%%
\pause
Let $\mathcal{C} \subset \Delta(\mathcal{A})$ feasible set of probability distributions such that,
\begin{align}
    \mathcal{C} = \{ \bm{p} \in \Delta(\mathcal{A}) \mid \bm{p}(U,L) = 1 \text{ or } \bm{p}(U,R) = 1 \text{ or } \bm{p}(D,L) = 1 \text{ or } \bm{p}(D,R) = 1\}
\end{align}\pause
%%%%%%%%%%%%%%%%%
\begin{itemize}\setlength{\itemsep}{-0.1cm}
    \setlength{\itemindent}{1em}
    \item The set of feasible strategies is the set of pure action profiles \pause
    \item The players must play a correlated strategy profile inducing a pure action profile in $G$. \pause
    \item There does not exist a constrained correlated equilibrium for this game \pause
\end{itemize}
%%%%%%%%%%%%%%%%%
\vspace{0.3cm}

\textcolor{aaltoblue}{\large Theorem 2 - Existence of Constrained Correlated Equilibria}\\
    Let $G$ be a finite non-cooperative game and $\mathcal{C}$ a feasible set of probability distributions. 
    If $\mathcal{C}$ is \textbf{non-empty}, \textbf{compact} and \textbf{convex}, then a constrained correlated equilibrium of $G$ exists.


\end{frame}
% new slide ===========================================================
\begin{frame}
\frametitle{Sketch of the Proof}

The proof is similar to Nash's existence theorem using Brouwer's fixed-point theorem applied to an appropriate function. \pause
\begin{enumerate}
    \item We define a function $g : \mathcal{C} \mapsto \mathcal{C}$\\ \pause
        \begin{align}
        g(\bm{p}) =
            \left(
                1 - 
                \frac{
                        \sum\limits_{i\in\mathcal{N}}
                        \sum\limits_{\beta_i \in \mathcal{H}_i(\bm{p})} \varphi_{i,\beta_i}(\bm{p})
                    }
                    {
                        1+\sum\limits_{i\in\mathcal{N}}\sum\limits_{\alpha_i \in \mathcal{H}_i(\bm{p})} \varphi_{i,\alpha_i}(\bm{p})
                    }
            \right)
            \times
            \bm{p}
            +
            \sum_{i \in \mathcal{N}}
            \sum_{\beta_i\in \mathcal{H}_i(\bm{p})}
            \frac{\varphi_{i, \beta_i}(\bm{p})}{1+ \sum\limits_{i\in\mathcal{N}}\sum\limits_{\alpha_i\in \mathcal{H}_i(\bm{p})}\varphi_{i,\alpha_i}(\bm{p})}
            \times
            {\bm{z}_{\beta_i, \bm{p}}}
    \end{align}
    where $\varphi_{i, \beta_i}(\bm{p})
            =
            \max \{ 0, u_i({\bm{z}_{\beta_i, \bm{p}}})-u_i(\bm{p}) \}$ and $\mathcal{H}_i(\bm{p}) = \{ f_i : {\mathcal{A}_i} \rightarrow {\mathcal{A}_i} \mid  \bm{z}_{f_i,\bm{p}} \in \mathcal{C}\}$ \pause
    \item We show that $g$ is continuous \pause
    \item Since $\mathcal{C}$ is non-empty, compact and convex, by Brouwer's fixed point theorem, $g$ has a fixed point. \pause
    \item We show that a probability distribution is a constrained correlated equilibrium distribution if and only if it is a fixed point of $g$.
\end{enumerate}

\end{frame}
% new slide ===========================================================
\begin{frame}
\frametitle{Computation of Constrained Correlated Equilibria}



\begin{columns}
\column{0.5\textwidth}
%%%%%%%%%%%%%%%%%%
\begin{align}
    & \text {maximize } 0 \nonumber \\
    \text { s.t. } & \bm{p}(\bm{a}) \geq 0 \quad \forall \bm{a} \in \mathcal{A}, \quad \Sigma_{\bm{a} \in \mathcal{A}} \bm{p}(\bm{a}) = 1 \\
                    & \forall i \in \mathcal{N}, \forall \beta_i : \mathcal{A}_i \rightarrow \mathcal{A}_i \nonumber \\
                    & \sum\limits_{\bm{a} \in \mathcal{A}} \bm{p}(\bm{a}) \left[u_{i}(\bm{a}) - u_{i}(\beta_i(a_i), \bm{a}_{-i}) \right] \geq 0. \\ \nonumber
    \end{align}  \pause
%%%%%%%%%%%%%%%%%%
\column{0.5\textwidth}
%%%%%%%%%%%%%%%%%%
\begin{align}
    & \text{maximize } 0 \nonumber \\
    \text { s.t. } & \bm{p}(\bm{a}) \geq 0 \quad \forall \bm{a} \in \mathcal{A}, \quad \Sigma_{\bm{a} \in \mathcal{A}} \bm{p}(\bm{a}) = 1 \\
                   & \forall i \in \mathcal{N}, \forall \beta_i : \mathcal{A}_i \rightarrow \mathcal{A}_i \textcolor{red}{\text{ s.t. } \bm{z}_{\beta_i, \bm{p}} \in \mathcal{C}} \nonumber \\
                   & \sum\limits_{\bm{a} \in \mathcal{A}} \bm{p}(\bm{a}) \left[u_{i}(\bm{a}) - u_{i}(\beta_i(a_i), \bm{a}_{-i}) \right] \geq 0,\\
                   & \textcolor{red}{\bm{p} \in \mathcal{C}}
\end{align}
%%%%%%%%%%%%%%%%%%
\end{columns}
\pause
\vspace{0.6cm}

Assume linear constraints e.g., $\mathcal{C} = \{ \bm{p} \in \Delta(\mathcal{A}) \mid F\bm{p} \leq 0 \}$.

\end{frame}
% new slide ===========================================================
\begin{frame}
\frametitle{Computation of Constrained Correlated Equilibria}

\begin{center}
    \textbf{Mixed-Integer Linear Program}
\end{center}
\pause
\vspace{0.4cm}

\begin{align}
    \boldsymbol{p} \geq 0, \Sigma_{\bm{a} \in \mathcal{A}} \bm{p}(\bm{a}) = 1, F \boldsymbol{p} \leq 0 & & \\
    {\left[U_i-U_i B_{\beta_i}\right] \boldsymbol{p} \leq M_i \times \boldsymbol{y}_{\beta_i}} & \quad \forall i \in \mathcal{N}, \quad \forall \beta_i \\
    F B_{\beta_i} \boldsymbol{p} \geq-K\left(1-\boldsymbol{y}_{\beta_i}\right)+\delta & \quad \forall i \in \mathcal{N}, \quad \forall \beta_i \\
    F B_{\beta_i} \bm{p} \leq K y_{\beta_i} & \quad \forall i \in \mathcal{N}, \quad \forall \beta_i\\
    y_{\beta_i} \in \{ 0,1 \} & \quad \forall i \in \mathcal{N}, \quad \forall \beta_i
\end{align}

\end{frame}
% new slide ===========================================================
\begin{frame}
\frametitle{Simulation Example}

%Assume constraints on the social welfare $SW = \Sigma_{i\in\mathcal{N}} u_i$ for the game of chicken.\\
%\vspace{-0.3cm}

\begin{columns}
\begin{column}{0.3\textwidth}
    \begin{table}
    \resizebox{3cm}{!}{%
    \begin{tabular}{c|c|c}
     & \multicolumn{1}{c|}{$P$}  & $A$ \\ \hline
    $P$ & \multicolumn{1}{c|}{8, 8} & 3, 10  \\ \hline
    $A$ & \multicolumn{1}{c|}{10, 3} & 0, 0
    \end{tabular}}
    \caption*{Game of Chicken}
    \end{table}
    \vspace{0.4cm}
    \begin{itemize}\setlength{\itemsep}{-0.2cm}
    \setlength{\itemindent}{.2in}
        \item Feasible utilities in \textcolor{ForestGreen}{green}    
        \item CE utilities in \textcolor{yellow}{yellow}
        \item \textbf{CCE} utilities in \textcolor{red}{red}        
    \end{itemize}
\end{column}
\begin{column}{0.6\textwidth}  %%<--- here
\vspace{0.25cm}
%%%%%%%%%%%%%%%%%%%%%%%%%%%%%%%%%%%%%%%%%%%%%%
%%%%%%%%%%%%%%%%%%%%%%%%%%%%%%%%
\begin{figure}
      \includegraphics[scale=0.43]{figures/P12.png}
      \caption*{\centering Utilities\\ Constraint: Social Welfare $\geq 12$}
\end{figure}
%%%%%%%%%%%%%%%%%%%%%%%%%%%%%%%%
%%%%%%%%%%%%%%%%%%%%%%%%%%%%%%%%%%%%%%%%%%%%%%
\end{column}
\end{columns}
\end{frame}
% new slide ===========================================================
\begin{frame}
\frametitle{Simulation results}

\begin{figure}
    \centering
    \begin{minipage}{.33\textwidth}
      \centering
      \includegraphics[scale=0.21]{figures/CCE_figures/P13.png}
      \caption*{\centering Utilities\\ Constraint: Social Welfare $\geq 13$}
      \label{fig:test1}
    \end{minipage}%
    \begin{minipage}{.33\textwidth}
      \centering
      \includegraphics[scale=0.21]{figures/CCE_figures/P14.png}
      \caption*{\centering Utilities\\ Constraint: Social Welfare $\geq 14$}
      \label{fig:test2}
    \end{minipage}
    \begin{minipage}{.33\textwidth}
      \centering
      \includegraphics[scale=0.21]{figures/CCE_figures/P15.png}
      \caption*{\centering Utilities\\ Constraint: Social Welfare $\geq 15$}
      \label{fig:test1}
    \end{minipage}%
\end{figure}

\begin{itemize}\setlength{\itemsep}{-0.1cm}
    \item The set may not be convex
    \item There are constrained correlated equilibria outside the set of correlated equilibrium distributions
\end{itemize}

\end{frame}
% new slide ===========================================================
\begin{frame}
\frametitle{Simulation results}

\begin{figure}
    \centering
    \begin{minipage}{.33\textwidth}
      \centering
      \includegraphics[scale=0.075]{figures/CCE_figures/C12.png}
      \caption*{\centering Probabilities\\ Constraint: Social Welfare $\geq 12$}
      \label{fig:test1}
    \end{minipage}%
    \begin{minipage}{.33\textwidth}
      \centering
      \includegraphics[scale=0.075]{figures/CCE_figures/C13.png}
      \caption*{\centering Probabilities\\ Constraint: Social Welfare $\geq 13$}
      \label{fig:test2}
    \end{minipage}
    \begin{minipage}{.33\textwidth}
      \centering
      \includegraphics[scale=0.08]{figures/CCE_figures/C15.png}
      \caption*{\centering Probabilities\\ Constraint: Social Welfare $\geq 15$}
      \label{fig:test1}
    \end{minipage}%
\end{figure}

\begin{itemize}\setlength{\itemsep}{-0.1cm}
    \item The set may not be convex
    \item There are constrained correlated equilibria outside the set of correlated equilibrium distributions
\end{itemize}

\end{frame}
% new slide ===========================================================
\begin{frame}

\begin{figure}
    \centering
    \includegraphics[scale=0.22]{figures/paper.png}
    %\caption{Caption}
    \label{fig:enter-label}
\end{figure}

\end{frame}
% new slide ===========================================================
\begin{frame}
\frametitle{Thank you!}

\begin{columns}
\begin{column}{4cm}
\begin{center}
\includegraphics[scale=0.3]{figures/questions} \\[-0.6cm]
\textbf{Questions?} 
\end{center}
\end{column}
\begin{column}{6.1cm}
  \begin{flushright}
    \textbf{For more information:}\\[0.1cm]
    {\footnotesize
    \textcolor{blue}{\url{omar.boufous@alumni.univ-avignon.fr}} \\[0.06cm]} 
 \vskip0.5cm
  \end{flushright}
\end{column}
\end{columns}

\end{frame}
\end{document}