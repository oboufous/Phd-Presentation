\documentclass[handout,first=blue,second=blue,logo=yellowexcELEC,usenames,dvipsnames, aspectratio=169]{aaltoslides169}
%\documentclass[first=blue,second=blue,logo=blueexc]{aaltoslides}
% \documentclass[handout]{beamer}        % to compile 2x2 handouts

%\usepackage{beamerthemesplit}
\usepackage[orientation=landscape,size=custom,width=16,height=9,debug]{beamerposter}

\usefonttheme[onlymath]{serif}

\setbeamercovered{transparent}
\setbeamertemplate{caption}[numbered]
\usepackage{caption}
\usepackage{multirow, makecell}
\usepackage{xfrac} %for different types of fraction
\usepackage{amsmath,amssymb}
\usepackage[ruled,vlined,linesnumbered]{algorithm2e}
\usepackage{arydshln}
\usepackage{graphicx}
\usepackage{lmodern}
\usepackage{tabulary}
\usepackage{tabularray}
\usepackage{bbm}
\usepackage{pifont}
\setbeamercovered{dynamic}
\usepackage{dsfont}
\usepackage{scrextend}
 \usepackage{appendixnumberbeamer}
\usepackage{bm}
\usepackage{gensymb}
\usepackage{subfig}
\usepackage{colortbl}
%\usepackage{subcaption}
%\usepackage{makecell}
\usepackage{cancel}
\usepackage{xcolor}

%\usepackage{ulem}
\newcommand{\soutthick}[1]{%
    \renewcommand{\ULthickness}{2.4pt}%
       \sout{#1}%
    \renewcommand{\ULthickness}{.4pt}% Resetting to ulem default
}

% ======= This is for custom footnotes
\newcommand{\customfootnotetext}[2]{{% Group to localize change to footnote
  \renewcommand{\thefootnote}{#1}% Update footnote counter representation
  \footnotetext[0]{#2}}}% Print footnote text

\usepackage{environ}
\usepackage{tikz}
\usetikzlibrary{fadings}
\usetikzlibrary{matrix}
\usetikzlibrary{chains}
\usetikzlibrary{positioning}
\usetikzlibrary{fit,shapes.geometric}
\usetikzlibrary{arrows, patterns}
\usepackage{tcolorbox}
\usepackage{multimedia}
\usepackage{hyperref}


\colorlet{shadecolor}{gray!40}



\tikzstyle{observation} = [rectangle, rounded corners, minimum width=3cm, minimum height=0.5cm,text centered]
\tikzstyle{binG} = [rectangle, rounded corners, minimum width=3cm, minimum height=0.5cm,text centered, draw=green,thick, fill=green!10]
\tikzstyle{binB} = [rectangle, rounded corners, minimum width=3cm, minimum height=0.5cm,text centered, draw=red,thick, fill=red!10]
\tikzstyle{arrow} = [thick,->,>=stealth]
\tikzstyle{crossout} = [rectangle, rounded corners, minimum width=3cm, minimum height=0.5cm,text centered, pattern=north west lines]

\tikzset{
	invisible/.style={opacity=0},
	visible on/.style={alt={#1{}{invisible}}},
	alt/.code args={<#1>#2#3}{%
		\alt<#1>{\pgfkeysalso{#2}}{\pgfkeysalso{#3}} % \pgfkeysalso doesn't change the path
	},
}

\usetikzlibrary{shapes.arrows}


\tikzset{
    myarrow/.style={
        draw=aaltoblue,
        fill=aaltoblue!70!white,
        single arrow,
        minimum height=5.5ex,
        single arrow head extend=1ex
    }
}
\newcommand{\arrowup}{%
\tikz [baseline=-0.5ex]{\node [myarrow,rotate=90] {};}
}
\newcommand{\arrowdown}{%
\tikz [baseline=-1ex]{\node [myarrow,rotate=-90] {};}
}



\newcounter{nodemarkers}
\newcommand<>\circletext[1]{%
	\tikz[overlay,remember picture] 
	\node (marker-\arabic{nodemarkers}-a) at (0,1.5ex) {};%
	#1%
	\tikz[overlay,remember picture]
	\node (marker-\arabic{nodemarkers}-b) at (0,0){};%
	\tikz[overlay,remember picture,inner sep=2pt]
	\node#2[draw,green,ellipse,fit=(marker-\arabic{nodemarkers}-a.center) (marker-\arabic{nodemarkers}-b.center)] {};%
	\stepcounter{nodemarkers}%
}



\everymath{\displaystyle}


 
\changefontsizes{8pt}

\usepackage{mathtools}
\DeclarePairedDelimiter\ceil{\lceil}{\rceil}
\DeclareMathOperator*{\argmax}{arg\,max}

\setlength{\leftmargini}{0.45cm}
\setlength{\leftmarginii}{0.35cm}

 \def\tr{\mathop{\mathrm{tr}}}
 
\setbeamertemplate{section in toc}[square]
\setbeamertemplate{subsection in toc}[square]
\setbeamerfont{section number projected}{size=\large}
\setbeamercolor{section number projected}{bg=aaltoblue,fg=white}

\newcommand\aaltofootertext[3]{\def\footfrow{#1}\def\footsrow{#2}\def\foottrow{#3}}
\aaltofootertext{60th Annual Allerton Conference on Communication, Control, and Computing}{September 2024}{\insertframenumber/\inserttotalframenumber}

% Syntax: \colorboxed[<color model>]{<color specification>}{<math formula>}
\newcommand*{\colorboxed}{}
\def\colorboxed#1#{%
  \colorboxedAux{#1}%
}
\newcommand*{\colorboxedAux}[3]{%
  % #1: optional argument for color model
  % #2: color specification
  % #3: formula
  \begingroup
    \colorlet{cb@saved}{.}%
    \color#1{#2}%
    \boxed{%
      \color{cb@saved}%
      #3%
    }%
  \endgroup
}

\newtcolorbox{adbox}[1][\hspace{-0.3cm} \textbf{Advertisement}]{
colback=white,
colbacktitle=aaltoblue!10!white,
coltitle=aaltoblue,
colframe=aaltoblue,
boxrule=1pt,
titlerule=0pt,
arc=5pt,
title={\strut#1}
}



\usepackage{pgfplots}

%\mode<presentation>{\usetheme{Warsaw}}

%\makeatletter
%\def\th@mystyle{%
%    \normalfont % body font
%    \setbeamercolor{block title example}{bg=orange,fg=white}
%    \setbeamercolor{block body example}{bg=orange!20,fg=black}
%    \def\inserttheoremblockenv{exampleblock}
%  }
%\makeatother
%\theoremstyle{mystyle}
\newtheorem*{remark}{Remark}




%List of packages
%\usepackage{amsmath}

%%%%%%%%%%%%%%%%%%%%%%%%%%%%%% Metadata %%%%%%%%%%%%%%%%%%%%%%%%%%%%%
\hypersetup
{
	%Separate multiple authors by comma
	pdfauthor={Omar Boufous},
	colorlinks=false
}

%%%%%%%%%%%%%%%%%%%%%%%%%%%%%% Title related %%%%%%%%%%%%%%%%%%%%%%%%%%%%%%
\setbeamertemplate{subsection in toc}[default]

%The contact for one of the authors MUST be embedded on the title (see below)
\title[]{Constrained Correlated Equilibria}
%Subtitle (if needed)
\subtitle{}
%For LICENSE, we suggest CC-BY-SA, but you are free to choose your own as long
%as the LICENSE you choose is AT LEAST as permissive as CC-BY-SA
\date[2024]{September, 2024\\ \vspace{1cm} {{ \small 60th Annual Allerton Conference on Communication, Control, and Computing}}}
\author[Boufous\hspace{0.2cm} {\includegraphics[height=0.2cm, keepaspectratio]{8015.png}}]{\texorpdfstring{\textbf{Omar Boufous}$^{1,2}$, Rachid El Azouzi$^2$, Mikaël Touati$^1$, Eitan Altman$^{2,3}$ and Mustapha Bouhtou$^2$}{}}
\institute[Orange]{$^1$ Orange, Châtillon, France\\$^2$ CERI/LIA, Université d’Avignon, Avignon, France\\$^3$ INRIA Sophia Antipolis, France}

%%%%%%%%%%%%%%%%%%%%%%%%% Presentation begins here %%%%%%%%%%%%%%%%%%%%%%%%%
\begin{document}

\begin{frame}
	\titlepage
\end{frame}
% new slide ===========================================================
\begin{frame}
\frametitle{Model \& Definitions}
\vspace{0.3cm}

\begin{columns}
\begin{column}{0.56\textwidth}
\; \quad Finite non-cooperative game  $G  = (\mathcal{N}, (\mathcal{A}_i)_{i \in \mathcal{N}}, (u_i)_{i \in \mathcal{N}})$ 
\begin{itemize}
\setlength{\itemindent}{2.5em}
\setlength\itemsep{-0.35em}
\item Set of players $\mathcal{N}$  
\item Action set $\mathcal{A}_i$ for each $i \in \mathcal{N}$  
\item Utility function $u_i : {\scriptscriptstyle \prod\limits_{i \in \mathcal{N}}}\mathcal{A}_i = \mathcal{A} \rightarrow \mathbb{R}$  
\end{itemize}

\end{column}
\begin{column}{0.46\textwidth}  %%<--- here
Correlation device  $d = (\Omega, (\mathcal{P}_i)_{i \in \mathcal{N}}, \bm{q})$ 
\begin{itemize}
\setlength{\itemindent}{1.2em}
\setlength\itemsep{-0.35em}
\item A sample space $\Omega$  
\item Partition $\mathcal{P}_i$ of $\Omega$  for each $i \in \mathcal{N}$   
\item Probability distribution $\bm{q}$ over $\Omega$  $\textcolor{white}{{\scriptscriptstyle \prod\limits_{i \in \mathcal{N}}}}$  
\end{itemize}
\end{column}
\end{columns}
Player \textbf{$i$'s set of strategies} is $\mathcal{S}_{i, d} = \{ f_i : \Omega \rightarrow \mathcal{A}_i \text{  s.t. }  f_i \text{ is } \mathcal{P}_i\text{-measurable} \}$ and the \textbf{set of correlated strategy profiles} is $\mathcal{S}_{d} = \mathcal{S}_{1, d} \times  ... \times \mathcal{S}_{n, d}$.  We define a correlated equilibrium strategy, 
\vspace{-0.1cm}
\begin{block}{Definition -- Correlated equilibrium\textsuperscript{$1$}}  
\vspace{-0.1cm}
A correlated equilibrium of $G$ is a pair $(d, \bm{\alpha}^*)$ where $\bm{\alpha}^* : \Omega \rightarrow \mathcal{A}$ is a correlated strategy profile such that  
\vspace{-0.3cm}
\begin{align}
\forall i \in \mathcal{N}, \forall \alpha_i^\prime : \Omega \rightarrow \mathcal{A}_i \quad  \sum_{\omega\in \Omega} \bm{q}(\omega) u_i(\bm{\alpha}^*_i(\omega), \bm{\alpha}^*_{-i}(\omega)) \geq  \sum_{\omega\in \Omega} \bm{q}(\omega) u_i(\alpha^\prime_i(\omega), \bm{\alpha}^*_{-i}(\omega))
\end{align}
\end{block}
\vspace{-0.4cm} 
Given a \textbf{coupled constraint set} $\mathcal{R}_d \subseteq \mathcal{S}_d$, we define a constrained correlated equilibrium,
\vspace{-0cm}
\begin{block}{Definition -- \textcolor{red}{Constrained} correlated equilibrium}%\textsuperscript{$2$}  
\vspace{-0.1cm}
A constrained correlated equilibrium of $G$ is a triplet $(d, \mathcal{R}_d, \bm{\alpha}^*)$ where $\bm{\alpha}^* : \Omega \rightarrow \mathcal{A}$ is a correlated strategy profile such that $\bm{\alpha}^*\in \mathcal{R}_d$ and 
\vspace{-0.3cm}
\begin{align}
\forall i \in \mathcal{N}, \forall \alpha_i^\prime : \Omega \rightarrow \mathcal{A}_i \text{ s.t. } (\alpha_i^\prime, \bm{\alpha}^*_{-i}) \in \mathcal{R}_d \quad  \sum_{\omega\in \Omega} \bm{q}(\omega) u_i(\bm{\alpha}^*_i(\omega), \bm{\alpha}^*_{-i}(\omega)) \geq  \sum_{\omega\in \Omega} \bm{q}(\omega) u_i(\alpha^\prime_i(\omega), \bm{\alpha}^*_{-i}(\omega))
\end{align}
\vspace{-0.1cm}
\end{block} 
\vspace{-0.1cm}
\customfootnotetext{$1$}{Aumann, R. J. (1974). {''\emph{Subjectivity and correlation in randomized strategies}''}. Journal of mathematical Economics.}
%\customfootnotetext{$2$}{Boufous, O., El-Azouzi, R., Touati, M., Altman, E., Bouhtou, M., (2023). {''\emph{Constrained Correlated Equilibria}''}. ArXiv (to appear).}
\end{frame}
%===============================================================================
\begin{frame}
\frametitle{Example}

\begin{columns}
\begin{column}{0.5\textwidth}
\begin{center}
{\textbf{Two-player game}}
\end{center}
\end{column}
\begin{column}{0.5\textwidth}  %%<--- here
\begin{center}
{\textbf{Correlation device}}
\end{center}
\end{column}
\end{columns}
\begin{columns}
\begin{column}{0.45\textwidth}
\begin{table}
\resizebox{3cm}{!}{%
\begin{tabular}{c|c|c}
 & \multicolumn{1}{c|}{\textbf{P}}  & \textbf{A} \\ \hline
\textbf{P} & \multicolumn{1}{c|}{8, 8} & 3, 10  \\ \hline
\textbf{A} & \multicolumn{1}{c|}{10, 3} & 0, 0
\end{tabular}}
%\caption*{\textcolor{aaltoblue}{Table:} Utility matrix of the game of Chicken.}
\end{table}

\end{column}  
\begin{column}{0.55\textwidth}  %%<--- here
\vspace{-0.55cm}
\begin{itemize}
\setlength{\itemindent}{3em}
\setlength\itemsep{-0.2em}
    \item $\Omega = \{ H, M, L\}$
    \item $\mathcal{P}_1 = \{ \{H\}, \{M, L\}\}$ and $\mathcal{P}_2 = \{ \{H, M\}, \{L\}\}$
    \item $\bm{q}(H) = \bm{q}(M) = \bm{q}(L) = \sfrac{1}{3}$
\end{itemize}

\end{column}
\end{columns} 
\vspace{-0cm}

\begin{columns}
\begin{column}{0.5\textwidth}
\begin{center}
{\textbf{Constrained extended game}}
\end{center}
\end{column}
\begin{column}{0.5\textwidth}  %%<--- here
\begin{center}
\visible<8->{\underline{\textbf{Players' utilities}}}
\end{center}
\end{column}
\end{columns}
\vspace{-0.35cm}
\begin{columns}
\begin{column}{0.5\textwidth}
\begin{center}
    \begin{figure}[H]
    \centering
    \scalebox{0.8}{
    \begin{tabular}{c|c|c|c|c}
                & \makecell{${L\textcolor{white}{^c} \mapsto \textbf{P}}$\\${L^c\mapsto \textbf{P}}$}
                & \makecell{${L\textcolor{white}{^c} \mapsto \textbf{A}}$\\${L^c \mapsto \textbf{A}}$}
                & \makecell{${L\textcolor{white}{^c} \mapsto \textbf{A}}$\\${L^c \mapsto \textbf{P}}$}
                & \makecell{${L\textcolor{white}{^c} \mapsto \textbf{P}}$\\${L^c \mapsto \textbf{A}}$}\\ \hline
    \makecell{${H\textcolor{white}{^c} \mapsto \textbf{P}}$\\${H^c \mapsto \textbf{P}}$} &       \textcolor<4>{blue}{$8, 8$}            &           \textcolor<4>{blue}{\textcolor<7->{red}{$3, 10$}}            &    \textcolor<6->{shadecolor}{$6.33, 8.67$}        &      \textcolor<6->{shadecolor}{$4.67, 9.33$}       \\ \hline
    \makecell{${H\textcolor{white}{^c} \mapsto \textbf{A}}$\\${H^c \mapsto \textbf{A}}$} &   \textcolor<4>{blue}{\textcolor<6->{shadecolor}{$10, 3$}}           &           \textcolor<4>{blue}{$0, 0$}           &       $6.67, 2$        &     $3.33, 1$ \\ \hline
    \makecell{${H\textcolor{white}{^c} \mapsto \textbf{A}}$\\${H^c \mapsto \textbf{P}}$} &   \textcolor<6->{shadecolor}{$8.67, 6.33$}        &      \textcolor<6->{shadecolor}{$2, 6.67$}                &       \textcolor<7->{red}{$7,7$}             &       $3.67, 6$        \\ \hline
    \makecell{${H\textcolor{white}{^c} \mapsto \textbf{P}}$\\${H^c \mapsto \textbf{A}}$}  &   \textcolor<6->{shadecolor}{$9.33, 4.67$}        &       $1, 3.33$               &     \textcolor<6->{shadecolor}{$6, 3.67$}          &     \textcolor<7->{red}{$4.33, 4.33$}        \\
    %$\begin{cases} \bm{H \mapsto \textbf{A}}\\ \bm{H^c \mapsto \textbf{P}}\end{cases}$ & b & c & d & e
    \end{tabular}}
    %\caption{Constrained extension of the game of Chicken.}
  \label{fig:ConstrainedExtendedPayoff}
\end{figure}
\end{center}
    
\end{column}
\begin{column}{0.5\textwidth}  %%<--- here
\begin{center}
   \visible<8->{\begin{figure}
       \centering
       %\includegraphics[scale=0.35]{figures/Figure_10.png}
   \end{figure}}
\end{center}
\end{column}
\end{columns}
\end{frame}




% new slide ===========================================================
\frame{\frametitle{Thank you!}
\vspace{-0.31cm}
\begin{columns}
\begin{column}{4cm}
\begin{center}
%\includegraphics[scale=0.299]{figures/questions} \\[-0.6cm]
\textbf{Questions?} 
\end{center}
\end{column}
\begin{column}{6.1cm}
  \begin{flushright}
    \textbf{For more information:}\\[0.1cm]
    {\footnotesize
    \textcolor{blue}{\url{omar.boufous@orange.com}} \\[0.06cm]} 
 \vskip0.5cm
  \end{flushright}
\end{column}
\end{columns}

}

\end{document}