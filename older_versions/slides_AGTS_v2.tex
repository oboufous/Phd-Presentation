\documentclass[first=blue,second=blue,logo=yellowexcELEC,usenames,dvipsnames, aspectratio=169, handout]{aaltoslides169}
%\documentclass[first=blue,second=blue,logo=blueexc]{aaltoslides}
% \documentclass[handout]{beamer}        % to compile 2x2 handouts

%\usepackage{beamerthemesplit}
\usepackage[orientation=landscape,size=custom,width=16,height=9,debug]{beamerposter}

\usefonttheme[onlymath]{serif}

\setbeamercovered{transparent}
\setbeamertemplate{caption}[numbered]
\usepackage{caption}
\usepackage{multirow, makecell}
\usepackage{xfrac} %for different types of fraction
\usepackage{amsmath,amssymb}
\usepackage[ruled,vlined,linesnumbered]{algorithm2e}
\usepackage{arydshln}
\usepackage{graphicx}
\usepackage{lmodern}
\usepackage{tabulary}
\usepackage{tabularray}
\usepackage{bbm}
\usepackage{pifont}
\setbeamercovered{dynamic}
\usepackage{dsfont}
\usepackage{scrextend}
 \usepackage{appendixnumberbeamer}
\usepackage{bm}
\usepackage{gensymb}
\usepackage{subfig}
\usepackage{colortbl}
%\usepackage{subcaption}
%\usepackage{makecell}
\usepackage{cancel}
\usepackage{xcolor}

%\usepackage{ulem}
\newcommand{\soutthick}[1]{%
    \renewcommand{\ULthickness}{2.4pt}%
       \sout{#1}%
    \renewcommand{\ULthickness}{.4pt}% Resetting to ulem default
}

% ======= This is for custom footnotes
\newcommand{\customfootnotetext}[2]{{% Group to localize change to footnote
  \renewcommand{\thefootnote}{#1}% Update footnote counter representation
  \footnotetext[0]{#2}}}% Print footnote text

\usepackage{environ}
\usepackage{tikz}
\usetikzlibrary{fadings}
\usetikzlibrary{matrix}
\usetikzlibrary{chains}
\usetikzlibrary{positioning}
\usetikzlibrary{fit,shapes.geometric}
\usetikzlibrary{arrows, patterns}
\usepackage{tcolorbox}
\usepackage{multimedia}
\usepackage{hyperref}


\colorlet{shadecolor}{gray!40}



\tikzstyle{observation} = [rectangle, rounded corners, minimum width=3cm, minimum height=0.5cm,text centered]
\tikzstyle{binG} = [rectangle, rounded corners, minimum width=3cm, minimum height=0.5cm,text centered, draw=green,thick, fill=green!10]
\tikzstyle{binB} = [rectangle, rounded corners, minimum width=3cm, minimum height=0.5cm,text centered, draw=red,thick, fill=red!10]
\tikzstyle{arrow} = [thick,->,>=stealth]
\tikzstyle{crossout} = [rectangle, rounded corners, minimum width=3cm, minimum height=0.5cm,text centered, pattern=north west lines]

\tikzset{
	invisible/.style={opacity=0},
	visible on/.style={alt={#1{}{invisible}}},
	alt/.code args={<#1>#2#3}{%
		\alt<#1>{\pgfkeysalso{#2}}{\pgfkeysalso{#3}} % \pgfkeysalso doesn't change the path
	},
}

\usetikzlibrary{shapes.arrows}


\tikzset{
    myarrow/.style={
        draw=aaltoblue,
        fill=aaltoblue!70!white,
        single arrow,
        minimum height=5.5ex,
        single arrow head extend=1ex
    }
}
\newcommand{\arrowup}{%
\tikz [baseline=-0.5ex]{\node [myarrow,rotate=90] {};}
}
\newcommand{\arrowdown}{%
\tikz [baseline=-1ex]{\node [myarrow,rotate=-90] {};}
}



\newcounter{nodemarkers}
\newcommand<>\circletext[1]{%
	\tikz[overlay,remember picture] 
	\node (marker-\arabic{nodemarkers}-a) at (0,1.5ex) {};%
	#1%
	\tikz[overlay,remember picture]
	\node (marker-\arabic{nodemarkers}-b) at (0,0){};%
	\tikz[overlay,remember picture,inner sep=2pt]
	\node#2[draw,green,ellipse,fit=(marker-\arabic{nodemarkers}-a.center) (marker-\arabic{nodemarkers}-b.center)] {};%
	\stepcounter{nodemarkers}%
}



\everymath{\displaystyle}


 
\changefontsizes{8pt}

\usepackage{mathtools}
\DeclarePairedDelimiter\ceil{\lceil}{\rceil}
\DeclareMathOperator*{\argmax}{arg\,max}

\setlength{\leftmargini}{0.45cm}
\setlength{\leftmarginii}{0.35cm}

 \def\tr{\mathop{\mathrm{tr}}}
 
\setbeamertemplate{section in toc}[square]
\setbeamertemplate{subsection in toc}[square]
\setbeamerfont{section number projected}{size=\large}
\setbeamercolor{section number projected}{bg=aaltoblue,fg=white}

\newcommand\aaltofootertext[3]{\def\footfrow{#1}\def\footsrow{#2}\def\foottrow{#3}}
\aaltofootertext{Alpine Game Theory Symposium}{June 2023}{\insertframenumber/\inserttotalframenumber}

% Syntax: \colorboxed[<color model>]{<color specification>}{<math formula>}
\newcommand*{\colorboxed}{}
\def\colorboxed#1#{%
  \colorboxedAux{#1}%
}
\newcommand*{\colorboxedAux}[3]{%
  % #1: optional argument for color model
  % #2: color specification
  % #3: formula
  \begingroup
    \colorlet{cb@saved}{.}%
    \color#1{#2}%
    \boxed{%
      \color{cb@saved}%
      #3%
    }%
  \endgroup
}

\newtcolorbox{adbox}[1][\hspace{-0.3cm} \textbf{Advertisement}]{
colback=white,
colbacktitle=aaltoblue!10!white,
coltitle=aaltoblue,
colframe=aaltoblue,
boxrule=1pt,
titlerule=0pt,
arc=5pt,
title={\strut#1}
}



\usepackage{pgfplots}

%\mode<presentation>{\usetheme{Warsaw}}

%\makeatletter
%\def\th@mystyle{%
%    \normalfont % body font
%    \setbeamercolor{block title example}{bg=orange,fg=white}
%    \setbeamercolor{block body example}{bg=orange!20,fg=black}
%    \def\inserttheoremblockenv{exampleblock}
%  }
%\makeatother
%\theoremstyle{mystyle}
\newtheorem*{remark}{Remark}




%List of packages
%\usepackage{amsmath}

%%%%%%%%%%%%%%%%%%%%%%%%%%%%%% Metadata %%%%%%%%%%%%%%%%%%%%%%%%%%%%%
\hypersetup
{
	%Separate multiple authors by comma
	pdfauthor={Omar Boufous},
	colorlinks=false
}

%%%%%%%%%%%%%%%%%%%%%%%%%%%%%% Title related %%%%%%%%%%%%%%%%%%%%%%%%%%%%%%
\setbeamertemplate{subsection in toc}[default]

%The contact for one of the authors MUST be embedded on the title (see below)
\title[]{Constrained Correlated Equilibria}
%Subtitle (if needed)
\subtitle{}
%For LICENSE, we suggest CC-BY-SA, but you are free to choose your own as long
%as the LICENSE you choose is AT LEAST as permissive as CC-BY-SA
\date[2023]{\small June, 2023\\ \vspace{1cm} {{ \small 2023 Alpine Game Theory Symposium, Grenoble}}}
\author[Boufous\hspace{0.2cm} {\includegraphics[height=0.2cm, keepaspectratio]{8015.png}}]{\texorpdfstring{Omar Boufous$^{1,2}$, Rachid El Azouzi$^2$, Mikaël Touati$^1$, Eitan Altman$^{2,3}$ and Mustapha Bouhtou$^1$}{}}
\institute[Orange]{$^1$ Orange, Châtillon, France\\$^2$ CERI/LIA, Université d’Avignon, Avignon, France\\$^3$ INRIA Sophia Antipolis, France}

%%%%%%%%%%%%%%%%%%%%%%%%% Presentation begins here %%%%%%%%%%%%%%%%%%%%%%%%%
\begin{document}

\begin{frame}
	\titlepage
\end{frame}
% new slide ===========================================================
\begin{frame}
\frametitle{Related Work}

\begin{columns}
\begin{column}{0.5\textwidth}
\begin{center}
\textbf{Correlated Equilibria and Extensions} 
\end{center}
\begin{itemize}\setlength{\itemsep}{-0.1cm}
    \setlength{\itemindent}{.2in}
    \item Correlated equilibria defined in (Aumann, 1974) and (Aumann, 1987)
    \item A second proof of existence and a generalization to infinite games is in (Hart \& Schmeidler, 1989)
    \item The concept was extended to extensive games (Von Stengel \& Forges, 2008) and to allow communications between players (Forges, 2020)  
    \item Other variants are presented in (Brandenburger \& Dekel, 1992) and (Grant \& Stauber, 2022)
\end{itemize}
\end{column}
\begin{column}{0.5\textwidth}
\begin{center}
\textbf{Generaliazd Nash Equilibria} 
\end{center}
\begin{itemize}\setlength{\itemsep}{-0.1cm}
    \setlength{\itemindent}{.2in}
    \item (Debreu, 1952) defines the concept of generalized equilibrium 
    \item (Arrow \& Debreu, 1954) presents a proof of existence of the generalized equilibrium
    \item (Rosen, 1965) shows the existence and uniqueness of a generalized Nash equilibrium in games with shared or coupled constraints under appropriate assumptions on utilities and the feasible set of joint strategies
\end{itemize}
\end{column}
\end{columns}
\vspace{1cm}

\begin{center}
\noindent $\rightarrow$ The combination of constraints and correlation has not yet been studied in the literature    
\end{center}

\end{frame}
% new slide ===========================================================
\begin{frame}
\frametitle{Approach \& Contribution}
\vspace{0.3cm}

Consider constraints on the social welfare $SW = \Sigma_{i\in\mathcal{N}} u_i$ for the game of Chicken\\
\vspace{-0.3cm}
\begin{columns}
\begin{column}{0.37\textwidth}
    \begin{table}
    \resizebox{3cm}{!}{%
    \begin{tabular}{c|c|c}
     & \multicolumn{1}{c|}{\textbf{P}}  & \textbf{A} \\ \hline
    \textbf{P} & \multicolumn{1}{c|}{8, 8} & 3, 10  \\ \hline
    \textbf{A} & \multicolumn{1}{c|}{10, 3} & 0, 0
    \end{tabular}}
    \caption*{\textcolor{aaltoblue}{Table:} Game of Chicken.}
    \end{table}
\end{column}
\begin{column}{0.63\textwidth}  %%<--- here
\vspace{0.25cm}
%%%%%%%%%%%%%%%%%%%%%%%%%%%%%%%%%%%%%%%%%%%%%%
%%%%%%%%%%%%%%%%%%%%%%%%%%%%%%%%
\begin{figure}
\centering
\begin{minipage}{.5\textwidth}
  \centering
  \includegraphics[scale=0.27]{figures/P12.png}
  \caption*{\textcolor{aaltoblue}{Figure:} Utilities for $SW \geq 12$}
  \label{fig:test1}
\end{minipage}%
\begin{minipage}{.5\textwidth}
  \centering
  \includegraphics[scale=0.27]{figures/P13.png}
  \caption*{\textcolor{aaltoblue}{Figure:} Utilities for $SW \geq 13$}
  \label{fig:test2}
\end{minipage}
\end{figure}
%%%%%%%%%%%%%%%%%%%%%%%%%%%%%%%%
%%%%%%%%%%%%%%%%%%%%%%%%%%%%%%%%%%%%%%%%%%%%%%
\end{column}
\end{columns}
\vspace{-0.2cm}
In this work, we
\begin{itemize}\setlength{\itemsep}{-0.1cm}
    \setlength{\itemindent}{.2in}
    \item define and characterize a new solution concept called constrained correlated equilibrium
    \item show a relation between constrained and unconstrained correlated equilibria
    \item show sufficient conditions of existence of a constrained correlated equilibrium
    \item characterize the set of constrained correlated equilibrium distributions
    \item study the constrained equilibrium distributions of the mixed extension of the game
\end{itemize}

\end{frame}
% new slide ===========================================================
\begin{frame}
\frametitle{References}
\vspace{0.1cm}

{ \small R. J. Aumann. Subjectivity and correlation in randomized strategies. Journal of Mathematical Economics, 1(1):67–96, 1974.}
\vspace{0.1cm}

{\small R. J. Aumann. Correlated equilibrium as an expression of bayesian rationality. Econometrica: Journal of the Econometric Society, pages 1–18, 1987. }
\vspace{0.1cm}

{ \small S. Hart \& D. Schmeidler. Existence of correlated equilibria. Mathematics of Operations Research, 14(1):18–25, 1989.}
\vspace{0.1cm}

{ \small B. Von Stengel \& F. Forges. Extensive-form correlated equilibrium: Definition and computational complexity. Mathematics of Operations Research, 33(4):1002–1022, 2008.}
\vspace{0.1cm}

{ \small F. Forges. Correlated equilibria and communication in games. Complex Social and Behavioral Systems: Game Theory and Agent-Based Models, pages 107–
118, 2020.}
\vspace{0.1cm}

{\small A. Brandenburger, E. Dekel, and John Geanakoplos. Correlated equilibrium with generalized information structures. Games and Economic Behavior, 4(2):182–201, 1992.}
\vspace{0.1cm}

{\small S. Grant and R. Stauber. Delegation and ambiguity in correlated equilibrium. Games and Economic Behavior, 132:487–509, 2022.}
\vspace{0.1cm}

{\small G. Debreu. A social equilibrium existence theorem. Proceedings of the National Academy of Sciences, 38(10):886–893, 1952.}
\vspace{0.1cm}

{\small K. J. Arrow and G Debreu. Existence of an equilibrium for a competitive economy. Econometrica: Journal of the Econometric Society, pages 265–290, 1954.}
\vspace{0.1cm}

{\small J. B. Rosen. Existence and Uniqueness of Equilibrium Points for Concave N-Person Games. Econometrica, 33(3):520, 1965.}

\end{frame}
\end{document}